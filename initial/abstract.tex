\documentclass[10pt]{article}
\usepackage{nopageno}
\usepackage{geometry}
\usepackage{cite}
\setlength{\headsep}{5pt}
\begin{document}
\title{Incentives for cooperation in a peer-to-peer name-lookup service}
\author{Marc Lehmann}
\date{}
\maketitle

\section*{Introduction and Motivation}
I'm proposing, as my Master thesis, to look at ways to incentivize participants
of a peer-to-peer network to invest resources into providing a name-lookup
service to their peers.

As motivation, consider a network in which every participant is identified by an
ID. In order to communicate with someone with a certain ID, a participant needs
to know a current IP address belonging to that ID. It's similar to DNS, but
more general, since the ID's namespace is flat, not hierarchical. The service
provoding this ID-IP mapping is a decentralized storage system. Getting peers
who want to use this service to also contribute storage space and bandwidth to
it is the core of this thesis.

While there may be multiple reasons to prefer a decentralized system over a more
centralized system like DNS\footnote{Technically, DNS is decentralized, but
authoritative, i.e. for each domain, you know which of a (typically small) set
of name servers is responsible. And anyway, most private internet users will use
a centralized external resolver, usually provided by their ISP, through which
all of their lookups go.}, like failure resistance, the most interesting to me
is privacy. In a system like DNS, where most requests go through few dedicated
servers, these servers gain insight into the communication behavior of their
clients and can build profiles. They can also become high-value targets for
external attackers trying to create such profiles.

Spreading the mapping information and the queries among multiple nodes gives
each of the service nodes a much smaller data set to build profiles from, and
attackers would now have to infiltrate every service node to gain the same
amount of information.

\section*{Description of the System}
The lookup service should be entirely decentralized, with no authoritative
entity whatsoever. It will essentially be a structured distributed hash table
like Kademlia\cite{maymounkov2002kademlia} or P-Grid\cite{aberer2003p} with some
extra rules. Under the assumption that every user of the service
\begin{itemize}
\item has an unconditional need to use the service, i.e. there is no alternative
they would be willing to consider if the cost gets too high
\item is a rational, selfish agent
\end{itemize}
these extra rules should ensure that all, or at least most, users also
contribute to the DHT in such a way that everyone receives a reasonable level of
service. It is important that a large portion of the user base contribute,
because the privacy benefit comes from being able to spread out requests to
multiple nodes so that creation of profiles becomes difficult. If, for example,
only very few very generous users were to contribute, it would start to resemble
a centralized system again.

Since a user's need to use the service is the only leverage the system has over
him, his contribution can only be ensured under a threat of an unacceptably low
level of service. Users must be able to expect a good level of service if they
behave socially well (i.e. answer queries) and a bad level of service if they
don't\footnote{The latter part is also important: Users who behave badly
\emph{must} be deliberately given a bad level of service, e.g. through a delayed
response. If they weren't, they'd have no incentive to behave well.
Additionally, those nodes being more generous than they should be would again
move towards becoming central servers. Therefore, answering queries of badly
behaving users too quickly must itself be considered bad behavior.}.

In order to know what level of service to grant a user, some form of reputation
system must exist, in which e.g. users are awarded or deducted points for
behaving well or badly, respectively. These points need to lose their value over
time, so that users can't simply refuse service if they've got enough points
from earlier good deeds (after all, the system must keep working).

While it might be possible to store this reputation information within the same
DHT as the name mappings, and even to incorporate servicing those records into
the reputation system, a DHT would be too slow. Querying the DHT holding the
name mappings generally takes multiple hops through the peer ID namespace
already, using a DHT for reputation data would mean adding an entire DHT lookup
at potentially every hop, just to find out how quickly the request should be
answered. Since looking up a name mapping is a somewhat latency-sensitive
application, this wouldn't be good enough. On the other hand, storing all
reputation information locally doesn't scale to large networks.

For this reason, there are two types of peer groups any peer is a part of.
Firstly, the (subset) of peers with whom he has an overlap in the keys of the
DHT he's responsible for. This is a usual part of a DHT used to receive and send
updates about new data in the DHT. Secondly, at least one (and probably more)
small groups within which reputation of the members is tracked. In any DHT, any
peer needs to be able to reach other peers closer to being responsible for
records in the DHT he can't service directly in order to forward queries for
those records\footnote{Items in the DHT are identified via IDs. Each peer picks
a prefix and is then responsible for all items whose ID begins with it. For
every subprefix of his prefix, he needs knowledge of another peer whose prefix
has the same subprefix, but differs in the bit after that.}.  The second kind of
group\footnote{I hope I'll come up with a good name for them at some point.} is
made up of those peers. All queries a peer has himself, or those that he needs
to forward from others, he will send to members of this kind of group so that
reputation scores can be properly applied. These groups should be small enough
that reputation can be tracked locally (it can be constant in the total network
size), with reputation information being disseminated via rumor spreading. The
groups must also be short-lived\footnote{"Short" means long enough that some
trust can grow, but short enough that members of the group don't get the chance
to learn much about each other. If I had to guess, I'd say that means at least a
few hours, at most a few days.}, so that they don't become a "familiar
environment" where the peers know each other and can build profiles again.

A problem that arises in systems without an authority in which users can easily
change their identity is free-riding, through whitewashing or a Sybil attack. In
the former, a user whose reputation has been tarnished by not contributing
leaves the system and rejoins with a fresh identity to once again receive good
service. This can quite easily be prevented by requiring an entrance fee, i.e.
having newcomers start out with a bad reputation. In my system, this would mean
users who have just joined a group of the second type will receive bad service.
In order to still have a working system at all times, users would join a new
group and build up some reputation by answering queries before they leave an old
group.

In the Sybil attack, a user creates multiple identites and has them rate each
other up. Related to this is collusion, in which multiple users conspire to rate
each other up without warrant. This problem is not as easy to solve. It may be
some deterrent to prevent peers voting up someone with the same IP (if that can
be detected), but even then it may be legitimate and only due to NAT.

\section*{Related Work}
The main threat posed by a centralized system to a user's privacy is that the
server gets a very good picture of whom the client is communicating with. This
could be addressed by private information retrieval (PIR), on which much work
has already been done. In PIR, a client queries a database without the database
knowing which record the client is interested in. In the trivial case, the user
downloads the entire database and picks the correct record locally. But there
are more efficient techniques, the best I found is
\cite{cachin1999computationally}\footnote{This is actually computationally
private information retrieval (cPIR), which adds the assumption that the
database can't perform superpolynomial computations}, which has polylogarithmic
communication complexity. However, in any case the server needs to process the
entire database for every request, otherwise it would gain some insight into
which records the client is not interested in. This seems impractical, as it
would likely cause excessive load on the server\footnote{I also don't think it's
possible to do any preprocessing to reduce the work, since the cPIR schemes I've
seen work with a security parameter (product of two large primes) transmitted by
the client, which is used in the calculation of the response and is unique per
client.}.

Work has been done examining common social dilemmas in peer-to-peer networks and
proposing possible solutions. \cite{feldman2004robust} gives a good overview.
The points most relevant for my thesis are:
\begin{itemize}
\item In a sufficiently large network, any given pair of peers is unlikely to
know each other, which makes some form of reputation system necessary, so that
peers know whom to trust.
\item Without a central authority, nothing can stop peers from rejoining the
network with a new identity, making free-riding behavior through whitewashing
(abandoning an identity with a tarnished reputation in favor of a fresh one) and
collusion (having multiple identities that vouch for each other) possible.
\end{itemize}
The latter is examined more closely in
\cite{feldman2006free,friedman2001social,douceur2002sybil}.
However, they all seem to be written with file sharing in mind, which isn't very
latency-sensitive, i.e. the time it takes to decide what degree of service to
provide to someone isn't critical.

\cite{li2012collusion} examines behavior patterns of peers who are colluding in
order to cheat a reputation system and proposes measures to detect it, albeit in
a model with some central authority.

Finally, Kademlia\cite{maymounkov2002kademlia} and P-Grid\cite{aberer2003p} are
structured DHTs that differ in some details, but have the same basic structure I
want to use.

\section*{Content of the Thesis}
The central question I want to investigate is whether I can find a reputation
system with a set of rules that gets the selfish peers to create social value,
i.e. to collectively provide a practically useful name-lookup service. This
includes choosing a reputation system, such as a debit-credit or a credit-only
point system, specifying how and by whom reputation changes are awarded (e.g. by
the recipient or the creditor) and setting parameters, e.g. the amount of
reputation points for a specific action. In order for the system to be
successful, it must work under practical conditions, e.g. where peers have long
periods of inactivity with bursts of activity in between, and it must be stable,
e.g. there mustn't be an inflation or deflation of reputation scores.

I will start by implementing a network of peers in a discrete event simulation.
I am not primarily interested in the messages the peers exchange (although this
is of some interest to figure out the overhead of the system), but in the
development of the peer behavior and how it affects the level of service within
the system. Each peer is modeled as a selfish agent who needs to have some
queries answered, as well as a valuation for the resources he needs to
contribute. Based on the information available to him, he can then update his
behavior and choose whether or not he wants to follow the rules of the system
(this includes e.g. how many mappings he stores, and whether he properly checks
peers' reputations before answering a query). In the beginning I will use a very
simplified model where e.g. the reputation data is available to all as a
perfectly synchronized centralized storage and doesn't actually need to be
maintained via rumor spreading, where messages arrive instantaneously and all
clocks are perfectly in sync. If the system works under these circumstances, I
can then restrict these aspects to see if it still works.

To implement the simulation, I would like to use SimPy or OMNeT++. Since I will
need to modify the behavior of peers and the network a lot, being able to use
Python would be very convenient. OMNeT++ also has a lot of features concerning
the specification of a network topology that I don't need, and may actually have
to work around, because I'm mostly concerned with what happens with the behavior
of the individual nodes. For all I care, the network topology may be a full
mesh. On the other hand, SimPy is apparently a lot
slower\cite{oujezsky2016case,vom2008comparing} and may just not be practical for
simulations of large networks that could become necessary. OMNeT++ also has
built-in visualizations that should help in figuring out e.g. how busy any one
node gets. I think I won't get around using OMNeT++, but I may start with a
prototype in SimPy.

Besides a simulation, there is also the approach of finding an analytical
game-theoretic solution. This encompasses, given some assumptions about peers'
utility for receiving service and cost for providing it, proving or disproving
the existence of equilibria in the behavior of peers, whether they are stable in
the presence of defectors and what their social value is. This may or may not be
possible.

In case the simulations have a negative result, i.e. I find that my incentive
system doesn't lead to a practical system, it would be nice to either describe a
different approach that seems more likely to succeed, or to argue why it failed
or even prove that no system can achieve this given the model.

Here are some further interesting questions, some of which could be answered as
a secondary objective (in no particular order):
\begin{itemize}
\item Could the DHT become segmented into multiple parts that are either not
      connected at all, or connected so weakly that it becomes difficult to
      update DHT entries from one part to another? If so, can these parts find
      together again somehow?
\item Is there a way to prevent collusion or a Sybil attack, or to make them
      harder?
\item Could a determined attacker still gather enough data about users of the
      system by joining as many groups as he can? Could he gain a monopoly on
      hosting some subset of the DHT content?
\item Could a determined attacker "follow" a particular user by joining all the
      same groups as he? If that user notices he's being followed and elects not
      to issue requests in groups the attacker is also a part of, doesn't this
      already constitute a DoS attack?
\item How many peers behaving socially badly can the system tolerate?
\item How many peers behaving maliciously (i.e. actively investing resources
\item How many peers behaving too generously can the system tolerate?
\item Can the system prevent all traitorous behaviour, or does some level need
      to be tolerated?
\item Can users be incentivized to properly maintain the records they're
      responsible for? Otherwise they may just be lazy and respond with made-up
      or old records.
\item What's the overhead of it all compared to a plain DHT without incentives
      or a centralized system?
\item How annoying is the bootstrap phase for a new user who doesn't have any
      reputation anywhere?
\item Are there any solutions for peers that are only online sparsely? They
      would have a hard time building up a good reputation before the group
      disbands. Should they be allowed to "spend" some extra reputation on
      receiving a higher level of service for a short time?
\item How much storage/maintenance is required at each peer?
\item What's a sensible number of groups for a peer to be a member of?
\item How are groups connected, i.e. how many hops does it take to get from one
      group to another? How likely is it that a pair of peers shares more than
      one group?
\item What's a good group size?
\item Who decides when a group is "full"? Do peers just leave when it gets too
      crowded for their taste?
\item Can the rules of the reputation system (e.g. how many points a peer gains
      for answering a query) be changed at run time? Is this perhaps even
      necessary for the system to continue functioning?
\item Can the system tolerate some free-riding users who are ok with bad service
      (e.g. weak embedded devices)? Should all requests be served at some point,
      even if the requesting node seems unlikely to ever contribute?
\item How is the situation handled where A is required to provide good service
      to B but is unable because he needs to forward the query in a group where
      he has bad reputation?
\item Is it a viable strategy to send out lots of dummy requests when joining a
      new group to find out whether everyone plays by the rules?
\item Should the new player entrance fee be reduced for newly created groups?
\item Is there a problem where peers preferably store popular mappings to get
      more reward per storage/traffic? Should there be higher rewards for
      servicing less popular mappings?
\end{itemize}

\bibliography{abstract}{}
\bibliographystyle{plain}
\end{document}
