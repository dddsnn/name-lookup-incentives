\documentclass{beamer}

\beamertemplatenavigationsymbolsempty
\setbeamertemplate{footline}[frame number]

\title{Incentives for cooperation in a peer-to-peer name-lookup service}
\author{Marc Lehmann}
\date{}
\begin{document}
\frame{\titlepage}

\begin{frame}
  \frametitle{Introduction}
  \begin{itemize}
    \item Mapping names (IDs) to IPs
    \item Like DNS, but more general (flat namespace)
    \item Decentralized service (DHT)
    \item Goal: get everyone who uses it to contribute
  \end{itemize}
\end{frame}

\begin{frame}
  \frametitle{Motivation}
  \begin{itemize}
    \item DNS basically centralized: typical user's DNS lookups go through one
          server
    \item Privacy concerns
    \begin{itemize}
      \item DNS server gets a lot of information about what servers a user
            contacts
      \item Can build user profiles
      \item Can become target for others who want to build profiles
    \end{itemize}
  \end{itemize}
\end{frame}

\begin{frame}
  \frametitle{Motivation}
  \begin{itemize}
    \item Instead of trusting a central server, spread requests to many
          different nodes
    \item Each of them only gets a small amount of data, harder to build a
          profile
    \item Requires users of the service also offering service
    \item Assume people are selfish and rational: incentivize them with better
          service
  \end{itemize}
\end{frame}

\begin{frame}
  \frametitle{(c)PIR}
  \begin{itemize}
    \item Privacy on central servers with (computationally) private information
          retrieval
    \item Query a database without the DB knowing what record you're interested
          in
    \item Trivial solution: download the entire database, query locally
    \item More communication efficient approaches exist
    \item But server must always process the entire database: impractical
  \end{itemize}
\end{frame}

\begin{frame}
  \frametitle{Distributed Hash Table}
  \begin{itemize}
    \item Store ID-IP mappings in a structured DHT (like Kademlia or P-Grid)
    \item Every item has an ID
    \item Peers pick a prefix and are responsible for items with that prefix
    \item Peers are in contact with others responsible for the same items
  \end{itemize}
\end{frame}

\begin{frame}
  \frametitle{Distributed Hash Table}
  \begin{itemize}
    % TODO illustrate
    \item For other prefixes: know other peers responsible for them
    \item Or know someone who knows someone (who can route to someone
          responsible)
    \item Accessing an item: ask a peer whose prefix is closer to being
          responsible for it
  \end{itemize}
\end{frame}

\begin{frame}
  \frametitle{Incentives}
  \begin{itemize}
    \item Only leverage: users want good service
    \item Users who don't contribute need to be given lower level of service
          (responses deliberately delayed)
    \item Peers need to know what level of service to grant: reputation system
    \item Track how well peers contribute (abide by the rules)
    \item E.g. point system (credit-debit/credit-only)
  \end{itemize}
\end{frame}

\begin{frame}
  \frametitle{Reputation Tracking}
  \begin{itemize}
    \item If requests can come from anyone, need to be able to find out anyone's
          reputation
    \item Storing locally impractical for large networks
    \item Storing in a DHT is too slow
  \end{itemize}
\end{frame}

\begin{frame}
  \frametitle{Reputation Tracking}
  \begin{itemize}
    % TODO illustrate
    \item Peers form small groups in which they forward queries ("the second
          kind")
    \item Reputation tracked in those groups (locally)
    \item Short-lived for privacy, but long enough that trust can grow
  \end{itemize}
\end{frame}

\begin{frame}
  \frametitle{Goal of the Thesis}
  \begin{itemize}
    \item Is such an incentive system possible, assuming selfish peers?
    \item Is there inflation or deflation of reputation?
    \item Could the resulting system be practically useful?
    \item What's the overhead?
  \end{itemize}
\end{frame}

\begin{frame}
  \frametitle{Methodology}
  \begin{itemize}
    \item Simulate with SimPy/OMNeT++
    \item Focus on behavior of peers, not the network
    \item Let them update their strategy
    \item Observe how peers' strategies change and what level of service is
          achieved
  \end{itemize}
\end{frame}

\begin{frame}
  \frametitle{Methodology}
  \begin{itemize}
    \item Try an analytical, game-theoretic solution
    \item Are there equilibria?
    \item Are they always reached?
    \item What is their social value?
  \end{itemize}
\end{frame}

\begin{frame}
\end{frame}

\begin{frame}
  \frametitle{Why do we need incentives?}
  \begin{itemize}
    \item Maybe we don't
    \item "Just distribute software that contributes to the DHT, people will be
          too lazy to change it for a small benefit"
          \begin{itemize}
            \item Some alternative implementation may opt to leave it out
            \item Maybe for performance, maybe for security
          \end{itemize}
    \item Assume the worst case, be ready for reality
    \item Just nice to know if it would work
  \end{itemize}
\end{frame}

\begin{frame}
  \frametitle{Some more questions}
  \begin{itemize}
    \item Could an attacker still collect information by joining many groups (of
          the second kind)?
    \item How many defectors can the system tolerate?
    \item How many malicious peers can the system tolerate?
    \item How many peers behaving too generously can the system tolerate?
    \item How annoying is starting out with no reputation anywhere?
    \item What about peers that are rarely online?
  \end{itemize}
\end{frame}

\begin{frame}
  \frametitle{Some more questions}
  \begin{itemize}
    \item What's a sensible number of groups for a peer to be in?
    \item What's a good group size?
    \item How are groups connected?
    \item Can the rules of the reputation system be changed at run time?
    \item If some ID-IP mappings are more popular than other, won't less popular
          ones be neglected because they offer a smaller benefit?
  \end{itemize}
\end{frame}

\end{document}
