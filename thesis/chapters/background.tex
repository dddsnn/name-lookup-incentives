\chapter{Background}
\label{chap:background}
\section{Distributed Hash Tables}
Distributed hash tables (DHTs) are a method of distributing data storage to
nodes in a network. From an abstract point of view, they are just hash tables
that map keys to values. They are appealing because they offer robust data
storage without the need for a central authority.

The keys usually have fixed length, the values, or records, can be arbitrary
data. Some care should be taken to ensure there are no collisions in keys. Using
a cryptographic hash of the record itself is a possibility achieving that.

Each node in the DHT stores a number of the records. In the language of this
thesis, it is \emph{responsible for} them. The key of a record determines who is
responsible for it, and there is a distance metric measuring how far a node is
to being responsible for a record. In the DHT used in this thesis every node has
an ID that has the same format as the keys in the DHT. Nodes are responsible for
those records the keys of which have the same prefix as their ID>

A node in the network needs not know all other nodes, but for any key it must
know another node closer to that node. Then it is possible to route to any
record stored in the DHT. The node can ask the closer node for the record, which
in turn asks another node closer, eventually reaching a responsible node.

Records can be created, updated and deleted by contacting the node responsible
for the corresponding key.

A useful robustness feature is replication, in which more than one node is
responsible for each record. This way, if a node fails, the record is not lost.
All the nodes responsible for the same records must stay in contact to
synchronize their records, making each other aware of new and updated records as
well as deletions.

Load balancing is another advanced feature that's useful if some records are
much more popular, or much bigger than others. To achieve it, the mapping from
node to the records the node is responsible for must be mutable. More nodes are
then made responsible for the records that generate a lot of traffic.

- TODO explain routing in p-grid/kademlia style dht (picture with the tree)

\section{Discrete Event Simulation}
TODO

\section{Game Theory}
TODO
- trembles, cite that paper
- collusion/sybil
- free riding, whitewashing
\subsection{Reputation Management}
- credit-debit, credit-only
