\chapter{Introduction}
\section{Privacy in Name-Lookup}
- dns-like name-lookup, names to ips in mind, but could have other applications
- dns basically centralized, privacy issue, dns server can profile
- distribute service, make it harder for a single party to profile
- users can decide how far they spread out requests
- not a guaranteed privacy protection, just makes it a lot harder to track
  everyone
- need a lot of servers -> dht, user's should contribute resources
- have no inherent incentive, service still available if they're lazy
- paper "serving dns using p2p lookup" also states "but who's gonna run the
  nodes?"

\section{Necessity}
- why is this even necessary?
- maybe it isn't, best to be prepared
- user bases in other p2p usually small, maybe just the minority that's
  relatively selfless

\section{Goal}
This thesis introduces a system that is meant to extend an existing \ac{DHT}
such that peers wanting to access records stored in it in a timely manner have
an incentive to contribute to the operation of the \ac{DHT}, i.e. to store
records and to answer queries about them. Having many peers contributing is then
supposed to allow users to choose whom they query out of many options, thus
providing the privacy benefit described in the introduction (TODO ref).

The motivating use case is that of a distributed name-lookup service, similar to
DNS, which maps user identifiers to network addresses. Storing
\emph{reachability information} (e.g. IP address) about oneself, where one's own
user identifier is the key to the record is also the only use case
considered\footnote{Other ones may be possible, but are considered out of scope
for simplicity.}. Since this is a somewhat latency-sensitive
application\footnote{"Somewhat latency-sensitive" here means "it would be
annoying if it took longer than 500ms" rather than "it must happen within 20ms
or the entire system breaks".}, the system should cause as few additional delays
during lookup as possible. The records that are stored in the \ac{DHT} are
assumed to be fairly small, i.e. the network delay doesn't have a significant
impact on the total delay. Furthermore, it must be possible to update records,
and for these updates to be disseminated throughout the \ac{DHT} within a
reasonably short time.

An underlying assumption for the most part of this thesis is that the users of
the system are selfish agents who want to use the system and gain access to the
many peers serving the \ac{DHT}, and with good performance. This means they want
to expend as few resources as possible, including storage space, CPU time and
network bandwidth. They only want to use these resources in order to, at all
times, be very likely to be in a position to receive service quickly. There is
no explicit valuation determining what "very likely" means to a peer (TODO isn't
there?). Rather, the reputation buffer described in the implementation chapter
(TODO ref) acts as a proxy for it, determining the reliability of a peer's
ability to receive service.

While this is the usual assumption about peer motivation, other types of peers
are discussed (TODO ref types of peers in this chapter, also contented peers,
maybe something else?).

The system should not have a negative impact on cooperating peers, i.e. peers
who are contributing in the way they are supposed to. Ideally, such peers should
not notice the system being there at all, safe for some management overhead, and
experience the \ac{DHT} as if all peers were cooperating as well.

The system should prevent free riding and be resistant to whitewashing.
Additionally, it should offer some protection against colluding peers trying to
appear as cooperating peers when they are not, and against Sybil attacks (TODO
ref background section for explanation).

\section{Outline}

\section{Related Work}
- (computationally) private information retrieval
- peers with utility functions that have a price for privacy
- that paper that found that not all peers to be selfish
- that paper that looked at patterns of collusion
- that paper that found trusting peers initially, based on the trustworthiness
  of recent joiners, is sane
