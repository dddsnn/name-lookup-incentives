\chapter{Introduction}
- TODO update submission date?
- TODO spellcheck

This thesis proposes a system built on top of a distributed hash table (DHT)
forcing its users to become nodes contributing to it. It does so by punishing
those who don't with a bad quality of service. The motivation is to use the DHT
for a name-lookup service and give its users many nodes to query for privacy
reasons.

\section{Privacy in Name-Lookup}
The domain name system (DNS) is a service essential for the Internet, its
primary purpose being the mapping of domain names to IP addresses. In general, a
name mapping service that allows looking up the network address associated with
a name belonging to some entity (e.g. a server, or a user identity) is
indispensable for any large network. Using the network addresses themselves is
impractical because they are often not permanently assigned.

The way name resolution is typically handled for private end users is that they
send all their lookups to one DNS server that they were made aware of by their
internet service provider. When they communicate with another host for the first
time, e.g. to access a web site or send an email, they send a DNS request to
that server. The response is then likely cached for a while by their
operating system or their browser, but the DNS server still gets at least an
initial request and knows which host they are likely communicating with. Thus,
DNS requests are privacy-sensitive data that allow the server to build user
profiles.

This issue, the accumulation of sensitive name resolution queries at central
locations, is the motivation behind this thesis. Note that just encrypting
traffic with DNS servers doesn't help, the issue isn't eavesdroppers on the
connection, it's all of a user's DNS meta data heaping up at one central server.
The operator of the DNS server doesn't even have to have malicious intentions,
or in fact any intentions to use the data. Simply the fact that data from many
users accumulates there makes the server a worthwhile target for infiltration by
malicious agents.

The chosen approach to address the issue is to spread out name-lookup requests
to many servers rather than using just one resolver. This doesn't make
incursions on users' privacy impossible, but should significantly increase the
work needed to gather a relevant amount of data, especially when attempted on a
large scale targeting many users at once. The thinking is that a foreign party
having access to a little bit of privacy-sensitive data is relatively benign,
that they are only able to build a meaningful profile once they have acquired a
considerable volume. With many lookup servers available, users are able to
choose themselves how far they want to spread out their requests.

\section{Goal}
There needs to be a great number of servers for this to work. This thesis
introduces a system in which these servers are the nodes of a DHT, and the
mapping from name to network address the \emph{records} stored in it. The main
purpose of this system is to ensure that there is a great number of nodes. To
achieve this, its main focus lies on incentivizing every user of the lookup
service to not just use it, but to become a \emph{peer}, i.e. a node of the DHT
that also contributes resources to providing it.

A good motivating use case, and one that's useful to keep in mind, is a
peer-to-peer chat system. Every peer has a fixed \emph{ID} by which he is known
to the others, but the network addresses can change frequently. The DHT is used
to store the mapping from ID to network address. In fact, the rest of the thesis
assumes that each peer stores exactly one record containing his
\emph{reachability information} in the DHT, where the key is equal to his own
peer ID (the key of a record is also referred to as the ID of the record).

Since network addresses change, peers frequently need to look up the current
addresses of others they want to chat with. Sending all these lookups to one
central server would give that server a pretty good picture of what a peer is
doing. Spreading them throughout the DHT means a lot of people know a little,
but no single party knows enough to be useful.

Accessing a record is somewhat latency sensitive. While it is not a real-time
application, it becomes annoying if the lookup before sending a message takes
too long. Furthermore, it must be possible to update records, and for these
updates to be disseminated throughout the DHT quickly. But we can assume that
the records are fairly small, containing just an ID, an address, and maybe some
additional information.

The system is likely capable of supporting other applications than such a chat
system, notably ones with fewer of the limiting assumptions. But for the sake of
simplicity, and to focus on the more interesting aspects, they are kept in place
here.

Another underlying assumption is that the users are selfish. They want to expend
as little storage space, CPU time and network bandwidth as possible. Without any
addition to the DHT, they can do that whether or not they contribute themselves;
one more node is not going to make a big difference in the grand scheme, and
certainly not provide them a benefit. Using the DHT but not contributing is
called free-riding. The fact that peers want to use the service and enjoy the
benefit of having many peers to choose from to do their lookups is the leverage
the system exploits: peers who aren't contributing are given a worse quality of
service.

Whitewashing is a similar exploit in which peers use the system until the level
of service they are given is not satisfactory anymore, then leave and join again
with a new identity. The system addresses this by being suspicious of newcomers,
giving them a low quality of service that only improves once they have proven
themselves trustworthy.

\section{Necessity of Incentives}
The assumption that all peers are entirely selfish and need to be made to see a
benefit to themselves for every action required of them is very pessimistic. It
begs the question whether such a system is even necessary. Storing a few small
records containing reachability information and responding to queries concerning
them only takes little resources. Given that, it's not unreasonable to be
optimistic and believe that people wishing to use such a service are willing to
contribute to it.

In fact, McGee and Sk{\aa}geby\cite{mcgee2004gifting} survey the motivation of
users of a peer-to-peer file sharing service to \emph{gift}, i.e. to seed
without reward. They find that there are users motivated by ideology
("information wants to be free"), as well as ones acting altruistically, i.e.
for whom the act of sharing files itself is the motivation.

A later paper\cite{ripeanu2006gifting} related to this examines motivations for
gifting in BitTorrent. The authors find that social characteristics strongly
influence sharing behavior, a user base that has a culture of sharing outside
the file sharing application will gift more in file sharing. They also find that
a small subset of the most active uploaders is responsible for a majority of the
uploads (in 95\% of torrents, over half the file is uploaded by the top 10\% of
uploaders).

This suggests that, while gifting, or generous behavior, in peer-to-peer
applications exists, relying on it may not be suitable for the system proposed
in this thesis. It could be suggested to peers that participating in offering
the service makes it better, in hopes of inviting generous peers' contributions,
as proposed in a blog post of the Tor Project\cite{dingledine2009incentive_tor}.
The problem is that this may mostly just encourage the small subset of the most
generous peers. But the system absolutely needs many peers participating so
users can spread out their queries. Few peers doing most of the work is actually
counterproductive to the privacy goal.

One could also argue that it should be sufficient to write an implementation of
the DHT in which peers contribute by default, perhaps with no option of
disabling it. The overhead of this is small, so users wouldn't change the
software for a small benefit. Ripeanu et al.\cite{ripeanu2006gifting} even
recognize the difficulty of configuring a BitTorrent client to free-ride as a
contribution to the low levels of free-riding. But then a problem will arise if
an alternative implementation leaves out the contribution, e.g. out of laziness
or for security reasons (less code means less space for vulnerabilities).

While there is no certainty that incentives are absolutely necessary, there are
some indications that they are. Cox et al.\cite{cox2002serving}, who explore the
possibility of serving DNS via peer-to-peer also state "We need to find models
in which people have incentives to run servers rather than just take free rides
on others’ servers.". In the end, it may be best just to be prepared for the
worst.

Whitewashing is a peer leaving and then rejoining the system with a different ID
in order to shed previous bad reputation. This attack is viable only in systems
where new users are given some trust upon joining, which they can subsequently
lose. In the system proposed here, peers joining a query group start at the very
bottom, namely 0 reputation (and reputation is non-negative). Rejoining thus
doesn't afford them any benefit.

\section{Outline}
Chapters~\ref{chap:related_work} and~\ref{chap:background} briefly present some
work that has already been done related to this thesis, as well as some
techniques that are the basis of the work done here.

Chapter~\ref{chap:system_description} presents first the goal of the incentive
system itself in further detail, and then how the system achieves that goal. A
simplified implementation that was created to evaluate the system within a
simulation framework is described in Chapter~\ref{chap:implementation}. After
that, Chapter~\ref{chap:rep_avail} presents various results of the simulations
that illustrate the properties of the system. It principally examines the
ability of cooperative peers to use the system without impediment.

Chapter~\ref{chap:performance} estimates the latency that the system might be
able to achieve if implemented and compares it to other name-lookup systems.

Chapter~\ref{chap:future_work} collects a number of interesting open problems
concerning the system, alternative techniques and a few possible attacks on both
the privacy of the users, as well as the reputation system. Possible solutions
are discussed, before Chapter~\ref{chap:conclusion} briefly summarizes and
concludes the thesis.
