\chapter{Reputation Availability}
The incentive system's goal is to provide a good level of service to those peers
who abide by the rules of the system (TODO terminology: cooperating?). Since the
system measures this via a reputation point system, cooperating (TODO) peers
must be able to gain the required amount of reputation within a reasonable time
frame.

This chapter examines under which circumstances this is the case.

\section{Environment}
TODO CHOOSE A SEED
- mention and reference settings files for reproducing results
- query group forming etc., should be repeat from system
  description/implementation
- some peers are forced to open new groups that no other than the initial 2
  peers join: little traffic means little traffic means slow reputation gains.
  better implementation could let peers invite other useful peers, or make the
  no-penalty reputation relative to e.g. total reputation in the group.
- query group size 16
- request generation
- only one query at a time
- 8 random introductions + sync group
- finding peers for uncovered subprefixes
- 64 peers
- penalties, rewards
- decay
- simplified query group access and creation, reputation updates

\section{Peer Selection Strategy}
\subsection{Strategy: Overlap}
\subsection{Strategy: Overlap $\rightarrow$ Reputation Sorted}
- best
\subsection{Strategy: Overlap $\rightarrow$ Reputation Saturated Last}
\subsection{Strategy: Reputation Sorted}
\subsection{Strategy: Random}
- time it takes to get to no-penalty rep, dependent on who gets picked first
- impact of allowing to not use maximal overlap peers: hop count, timeouts (due
  to penalties, recursive queries), failures?
\subsection{Uneven Query Distribution}
\section{Reputation Decay}
- decay, necessary (see system description), but difficult
- decay doesn't really work if it's supposed to be the main incentive in
  continuing to participate: too difficult to balance even with the constant
  stream of requests; need timeout penalties as main measure
- possible solutions: vary decay rate with total reputation in group, or with
  average number of queries in a group or something
\section{Penalty Expectations}
- reputation oscillating
- peers need to expect penalties when they decide not to answer
- implementation flaw: peers don't expect penalties if they need to query
  further and it takes too long (recursive query problem). expecting a penalty
  for a fail response doesn't make sense if the response had already timed out.
- high buffer as workaround
\section{Reputation Attenuation}
- necessity
- different parameters
- the harder it is to get to max desired reputation, the fewer timeouts there
  are (calculate timeout rate)
- discuss whether lower\_bound should be set higher than no-penalty rep (10)
- possible improvements
    - peers expect rewards (not just penalties)
    - peers have a utility function that balances the investment in getting more
      reputation (more expensive the more they get) against the risk of going
      below the non-penalty rep
\subsection{Constant}
\subsection{Exponential}
\subsection{Harmonic}
\section{Query Group Reevaluation}
- incomplete implementation causes problems: peers first find their necessary
  query peers, but then they leave. potentially mitigated by running query peer
  discovery repeatedly now?
\section{Getting Started in a Running System (TODO section name)}
- number of introductions needed to get subprefix coverage
\section{Effect of Small Sync Groups}
- with small sync groups, particularly one peer per sync group, the system is
  very brittle, even if all sync groups exist (see 11/about)
- routing loops are a problem that can occur when overlap isn't the first
  selection criterion (particularly subprefix queries). this can lead to
  timeouts and lots of penalties (related to brittleness with small sync groups)
- the problem with each peer in his own sync group is apparently that, once the
  first peers reach 14, they let queries time out, and this causes problems to
  other peers who were relying on them, crashing reputations. with longer
  prefixes and deeper lookups (more hops), the likelihood of something timing
  out on the way rises.


-----------------------
- recursive-query-problem, observable in peer reputations (some configurations?
  see 8/about)
- impact of query group size
- uneven distribution of rep (effect of even sync groups)
- difference in expected delays because of difference in time between
  calculating; may have changed in the meantime
- peers can get more than their maximum desired reputation
- in the case of multiple shared groups, where do peers get rep?
- query peer selection: must send to peer with lowest rep
- uneven query distribution
- query group reevaluation mechanism, caveat that new groups can't be created,
  only when a new peer is introduced
- missing sync groups are a problem: no one can say for sure that there doesn't
  exist a peer in that sync group, so peers querying for missing subprefix
  coverage give out penalties to those they query because of the resulting fail
  responses. no real solution other than ensuring non-empty sync groups (linked
  to replication). this is confirmed by giving peers global knowledge of which
  sync groups exist and not having them query for those subprefixes, then
  everything works. even one missing sync group can screw everything up (see 64
  peers/6 bit prefix case).
- number of introductions relative to the amount of peers has an effect on how
  easy it is for the system to get started (see 11/about)
