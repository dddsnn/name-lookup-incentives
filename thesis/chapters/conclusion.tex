\chapter{Conclusion}
\label{chap:conclusion}
This thesis proposed a reputation system for a DHT storing reachability
information that is meant to increase participation so that queries can be
spread out among many recipients.

Broadly speaking, there are three elements of the system that are absolutely
integral and need to be investigated. They are firstly the availibility of
reputation to cooperative peers, which is necessary to make the system useful to
them. Secondly, a way of creating, joining and leaving query groups that allows
peers to effectively find others that they need to forward their queries. And
lastly, the complaint system, which is necessary to prevent abuse of the
reputation system.

The first of these has been examined in this thesis via simulation. With the
implementation as it is, the results are encouraging so far. They show, with
some caveats, that selfish peers are able to gain enough reputation to be
entitled to delay-free service, under the assumption that they can be motivated
to contribute by the promise of a good quality of service.

The other two have not been implemented, but some ideas presented. If those are
implemented, they are likely to influence each other, as well as the reputation
availability in ways that are difficult to predict. Further work is needed to
determine how these parts interact and whether they can be assembled into a
whole that is capable of fulfilling the design goal.

With regards to latency, the system compares very unfavorably to DNS. This is
not surprising, given that their purposes are not precisely the same. DNS first
and foremost provides resolutions from authoritative servers quickly, whereas
the DHT is focused on robustness and reliability. Queries in the DHT are also
not optimized for proximity between peers. When performance is compared agains
DNS queries done via Tor, however, the system could conceivably be faster.

Some possible attacks on the system, ones exploiting the reputation system, as
well as ones seeking to circumvent the privacy gains, have been described and
possible countermeasures discussed.
