\chapter{Future Work}
\label{chap:future_work}
There are quite a few parts necessary for the system to function that could not
be implemented and examined in this thesis. This chapter collects ideas about
these parts.

\section{Query Group Creation and Maintenance}
Query groups need to be created and disbanded, and peers need to be able to join
and leave them. This is an outline of a protocol, but there are likely to be
complex dynamics involved causing unforeseeable problems.

\subsection{Leaving and Disbanding}
Leaving a query group can be achieved by simply broadcasting a message
indicating this to all members. A query group is considered disbanded if it only
contains one peer, i.e. once the second-to-last member leaves.

\subsection{Creation}
A simple way to create a query group is for one peer to ask another to start
one, thus creating a query group containing 2 peers. This group would then grow
by new peers joining it. However, this assumes that it is reasonably easy for
one peer to find another with whom he is interested to share a query group and
who shares that interest. This may not be the case; a peer can send queries for
prefixes he is interested in, but the peers in the responses may not be
interested in him.

Peers are interested in others so that they can extend their subprefix coverage,
or get some redundancy into their coverage. For any pair of peers A and B, they
are either responsible for the same routing prefix, or A is responsible for
exactly one of B's subprefixes, namely the one of length $i$, where $i$ is the
one-based index of the first 1-bit in the bitwise XOR of the peer's respective
routing prefixes. So every peer can serve some subprefix for any other peer, and
the subprefix relationship between two peers is even symmetric, i.e. the
subprefix A can serve for B has the same length as the one B can serve for A
(they differ in the final bit). But that doesn't necessitate B being interested
in sharing a query group with A: he may already know enough peers for that
subprefix, or just not trust A for some reason.

A more complex way of creating a query group could involve the initiating peer
considering which subprefixes other peers are even interested in. This could be
announced in a peer's record that's stored in the DHT. The initiating peer would
check that the other peer is interested in extending his coverage to a subprefix
he is responsible for.

A query group could also be initiated with more than just 2 peers. The
initiating peer would contact all peers he wants to invite and propose to start
a group, attaching a list of members. If all agree, the initiating peer
broadcasts reachability information of all members. This allows for advanced
strategies in finding peers, where peers aren't pairwise interested in each
other, but circularly. E.g. A is interested in B, B in C, and C again in A.
Having more peers can also be helpful since it means more traffic (in small
groups, it can be hard to gain reputation for lack of queries).

\subsection{Usefulness}
\label{sec:desc_usefulness}
In this context it is convenient to introduce the notion of \emph{usefulness}:
it is the value peer A places on knowing peer B, with regards to his subprefix
coverage. It is a policy decision, not something the rules of the system can
dictate, and may depend on such things as the number of peers peer A already
knows for the subprefix, or whether he trusts peer B (maybe he feels like B is
following him and he doesn't want to deal with him at all).

These factors are not visible to peer B, the length of the subprefix he can
serve is, though. All else being equal, peers serving longer prefixes are more
useful than those serving shorter ones. That's because peers for shorter
subprefixes are easier to find: consider a subprefix of length 1, which can be
served by all peers whose routing prefix begins with that single bit. Assuming
evenly distributed IDs, that's half the population. Longer subprefixes are also
closer to a peer's own routing prefix, and the peer is likely to get more
queries for it, making more redundancy a handy thing to have.

The length of subprefix a peer can serve for another peer could be a heuristic
for the peer to estimate how useful he is to the other peer. And, more
crucially, how useful he would be in a query group, by calculating the sum of
the lengths of the subprefixes he can serve for each peer (or e.g. the sum of
squares, to give longer subprefixes a higher weight). This can serve as an
estimator for how many queries he is likely to receive in that group, and thus
how easily he can gain reputation.

\subsection{Inviting and Joining}
Joining a query group must be done via an \emph{entry peer}, a current member of
the group who handles the joining process. The membership candidate may ask the
entry peer to join, or receive a proactive invitation from him. Since a new
member affects everyone in the group, the entry peer should not be allowed to
make the decision by himself.

The simplest way is a majority vote: The entry peer presents the candidate to
the current members, and if more than half of them agree that he may join, he is
accepted. A modification is to require a unanimous vote. The members may use
their own judgment in this decision, e.g. considering how useful the candidate
would be to them, what impact he might have on the total traffic in the group
(especially with regards to reputation updates), or whether they distrust him
from previous experience.

If a candidate not all current members agree with is accepted, these members may
leave the group. Thus a candidate threatens the integrity of the group.
Therefore, it may be helpful to hold the vote in a way that asks the current
members whether they would leave the group if the candidate was added. Again,
this vote is up to peer policy. They may consider the peer untrustworthy, not
useful to them, or simply deem the group to big after his addition. This would
have to be done in multiple rounds: Peers announcing they'd leave in the first
round may prompt additional peers to want to leave, who were there for
connection to those peers. This is repeated until no additional peers announce
their intention to leave.

After all peer intentions have been collected, the decision to accept or reject
the candidate has to be made. It could either be made by the entry peer alone,
assuming that everyone who isn't going to leave is alright with the new peer
joining, or again by a vote (majority or unanimous) of the members who would
remain after the candidate joins.

This approach assumes peers always have the option to leave a query group. But
some may depend on the group because their subprefix coverage would become
incomplete if they left. These peers would be pressured to stay and have no real
say in accepting or refusing the candidate.

In a possible extension, those peers that leave the group after that candidate
joins may automatically form a new query group. Candidates looking to join
groups then act as a sort of catalyst to split groups into smaller parts.

The process of deciding on a candidate's membership is likely to entail complex
dynamics. Both the simple voting and the other approach with peers announcing
they'd leave is highly dependent on peer decisions, and neither is clear to
yield a system that allows a good distribution of peers in query groups of
useful size (i.e. not just 2 peers per group).

\subsection{Finding Groups}
In order to request to join a query group, a peer first needs to find one he
would like to join. After all, there is no public record of all available query
groups.

To do this, the query mechanism could be extended to allow queries not just for
peers (i.e. their record in the DHT), but for query groups. The peer looking for
a group would send a query to some of his current query peers stating he is
looking for a group. In that query, he lists one or more of his subprefixes for
which there should be responsible peers in the group. If the recipient of the
query is in a group matching these criteria, he may offer to act as the entry
peer and receive a reward in the current query group for a successfully answered
query.

In addition to listing subprefixes that should be covered, the peer looking for
a group could request that his usefulness in the prospective group be calculated
or estimated, and only groups offered where it exceeds a threshold. This may be
necessary to ensure the new peer even gets any queries and can gain reputation
in order to make use of his new group.

A peer completely new to the system needs some way of entering his first query
group. He needs to know at least one peer already in the system to get started,
who must act as entry peer to at least one query group.

It is not at all clear that this proposal would work in practice. If this
process is used alone, it may well happen that the network becomes fragmented,
with some parts of it being entirely unaware of others, and thus unable to
access records stored there. There is also no solution for the problem that a
peer new to the system is rejected in all query groups he is offered in the
beginning.

\section{Gossipping}
Instead of broadcasting the reputation updates, it may be possible to use
gossipping: The originator only tells a few query peers, who then tell a few
others, and so on. This spreads the traffic load out to more than one peer.
However, all the peers sending messages then have to have incentive to do so. If
they are given rewards, depending on the gossipping protocol, peers low on
reputation may opt to send many messages to gain lots of reputation, leading to
redundant traffic. On the other hand, if lots of peers already have enough
reputation, they may opt not to send anything and gossipping fails.

Gossipping could also potentially be used for the updates of records within sync
groups. In sync groups, it has the added benefit that sync peers don't all need
to know each other, there just needs to be a directed path between any pair of
them. This could make larger sync group sizes practical. But again there is no
incentive for the peers to participate in gossipping. Perhaps a separate
reputation system for sync groups could do the trick.

\section{Complaint System}
\label{sec:desc_complaints}
In certain situations, particularly regarding rewards or penalties being applied
or not, a peer may treat one of his query peers unfairly, but deny it. Usually,
no one but the two peers involved know what actually happened, and there is no
way for the wronged peer to prove it. The two most important examples are:
\begin{itemize}
\item Peer A queries peer B, B responds successfully. But then A fails to send
      the cooperation confirmation that allows B to claim a reward.
\item Peer A sends a reputation update containing a penalty for a peer B (thus
      receiving a small reward himself), but B didn't actually do anything
      wrong.
\end{itemize}

There may be multiple reasons for a peer to misbehave in such a way. In the
first example, it may be laziness (sending the confirmation yields no benefit to
the peer). The second example may be due to greed, since the peer receives a
small reward for broadcasting a reputation update. But it needn't necessarily be
selfishness, it could also be due to a tremble: A message may simply have been
lost or delayed in the network. Then the protocol implementation at both peers
may assume that it is in the right.

In both examples, only the two involved peers know what happened, and neither
can prove it to anyone else: In the first case, B has no way of proving that A
didn't send the cooperation confirmation, not even that A recently sent a query,
much less that he answered it successfully (and in time). In the second case,
peer B can't prove he didn't do anything wrong. Peer A may for example have
queried him with no reaction from B, and again B can't prove that he didn't
receive a query.

To resolve situations like these, there must be a complaint system within each
query group, via which one peer can complain about one of his query peers. Such
a complaint may be accepted, resulting in a penalty for the subject of the
complaint and a reward for the complaining peer (as compensation), or it may be
rejected. Of course, this system can't determine the truth either, but it should
not be continually exploitable, i.e. a peer should not be able to continually
misbehave without receiving a penalty through the complaint system, and a peer
should not be able to continually lodge false complaints against another peer
that are accepted.

\paragraph{Possible Solutions}
One possible solution is to hold a vote, and if the majority of query peers
believes the complainer, the offender is removed from the query group. This
should incentivize peers to not mistreat others provided they are interested in
staying in the group. When voting, the peers are free to use any criteria they
like. Experience from previous interactions with the accused is probably a good
choice, like whether he has mistreated the peer casting the vote before.

Of course, this makes it possible for a peer to continually mistreat a few of
his query peers. As long as he treats the others well, they will not believe the
complaints. Then again, the victims may broadcast false penalties for the
offender in retaliation, forcing him to complain about them. But if they have a
good standing with the rest of the group, those complaints will not be believed
either.

Whether this system is a successful deterrent, or whether there are situations
in which a peer deems it profitable to lie about someone anyway is not at all
clear.

Another possible way of handling a complaint is to make it impossible for the
two peers involved to penalize each other, removing their incentive to interact.
They would remain in the same group and interact with all the other members as
normally, just not with each other.

However, this could quite possibly lead to peers strategically severing their
connection to others they don't need for querying. It may get more difficult for
peers to complete their subprefix coverage, maybe to the point of the DHT
failing.

Finally, the paper by Papaioannou and Stamoulis\cite{papaioannou2005optimizing}
mentioned in Chapter~\ref{chap:related_work} claims that there is a way to
successfully detect liars involving both parties getting punished if their
assessment of a mutual transaction differs. But it is not clear how this
translates to a situation where there isn't even agreement that a transaction
took place, like in the case of a penalty for a timeout that may not have
happened.

\section{Load Balancing}
\label{sec:desc_load_balancing}
In the system as described, every peer is responsible for his routing prefix,
which is just a prefix of his own ID. Assuming IDs are evenly distributed (which
is the case if they are fingerprints of public keys and peers don't make
attempts to generate a particular key), this implies the expected value of the
number of peers responsible for a record is equal for all records. This is fine
under the assumption that all records are queried for with the same likelihood.

A more realistic record access pattern follows a power law distribution, with
very few records forming a large share of the total number of queries. Taking
this into consideration, distributing the load of responding to all the queries
to the popular records becomes necessary, or else those peers who happen to have
a routing prefix matching them receive an unfairly disproportionate amount of
queries. They may be inclined to generate a new ID that doesn't suffer from
this.

One of the central tenets of P-Grid involves separating a peer's routing prefix
from the records he is responsible for. Instead of the peer's ID determining
which records he has to store, each peer explicitly advertises one or more
prefixes which he is responsible for (this can become part of their record
stored in the DHT). These new prefixes don't even all have to be the same
length, allowing for finer grained control. Being responsible for more prefixes
means peers have to join more sync groups and completing their subprefix
coverage may require more connections.

With this system in place, it becomes possible for more peers to take over
responsibility for prefixes containing popular records, thus balancing the load.
In P-Grid, this is done because peers are well-intentioned and willing to
cooperate. Under the assumptions made by the system in this thesis, peers need
an incentive to do so. They may have such an incentive because they are unable
to gain reputation quickly enough because they aren't getting enough queries.
Taking on responsibility for a popular prefix can address this.

A new kind of query could help peers decide which new prefixes to cover, in
which they query for a prefix which receives a lot of traffic and the peers
responsible for it would like the load lightened. Ideally, this system would be
self-organizing, like P-Grid is, but via economical incentives. Whether this is
actually the case is an open question.

\section{Recursive vs. Iterative Queries}
\label{sec:desc_recursive_vs_iterative}
Section~\ref{sec:desc_querying} described recursive querying, in which the
recipient of a query is expected to respond with either a successful or a fail
response. If he isn't responsible for the target ID and thus doesn't have a copy
of the record, he has to query a suitable peer himself and wait for the response
in order to pass it on to the peer who queried him. This is the simplest way to
handle the situation, and the most convenient for the querying peer, since all
the work is done by the recipient. But they can also lead to the recursive query
problem.

Instead of recursive queries, iterative queries could be used. With this method,
the recipient is only expected to find a suitable next peer to query, i.e.  one
who is closer to the target ID, and respond with this peer's reachability
information. The querying peer must then query that peer himself. This solves
the recursive query problem, because the recipient of a query only needs to do a
local lookup and is not dependent on another peer.

It introduces a new problem, though. There is no guarantee, and with growing
network size it becomes increasingly unlikely, that the querying peer shares a
query group with the peer the first recipient said to contact next. That peer
has no incentive to respond to a peer with whom he shares no query group.

A possible solution to this problem is for the responder to pass on his level of
service. Say peer A has queried peer B, who responds with reachability
information for peer C. The response now also contains a \emph{query voucher},
which is a statement signed by B that allows A to send a query to C with the
quality of service that B would have gotten. A can then query C, and C must
treat him as though he were B. This entitles B to a reward for a successful
response, even if the level of service ultimately received by A is bad (because
B has low reputation in the query group he shares with C). After all, B
demonstrated his best effort to answer the query.

Query vouchers need to have a (relatively short) expiry time to prevent peers
from stockpiling vouchers and then leeching off of the issuer's good standing.
This requires an expiry time on the voucher, as well as clocks within the
network to be reasonably in sync.

Query vouchers also require the complaint system to be extended. Using peers A,
B and C from the example, A needs to have a threat of penalty for not sending a
cooperation confirmation to C after a successfully query. But C can only complain
about B, so there must be a new complaint type that allows B to complain about A
for having received a complaint after use of a voucher. Then C complains about
B, and B complains about A.

\section{Credit-only vs. Credit-Debit Reputation}
\label{sec:desc_credit_only_vs_credit_debit}
The system uses credit-only reputation, i.e. no reputation is deducted for using
the system (e.g. per query). Once peers have gathered at least the penalty
threshold reputation, they are entitled to delay-free service. They only lose
this privilege if they incur a penalty, most notably for failing to respond
properly to a query. As long as this doesn't happen, they can send as many
queries as they like. With this method, having reputation is seen as proof that
a peer is generally willing to contribute, to play a part of the whole. It is
unnecessary to deduct this reputation, as querying someone else doesn't change
that assessment.

In the alternative, credit-debit, each query a peer gets successfully answered
costs him a little reputation as a sort of payment. This could be incorporated
into the system quite easily, by incorporating this penalty into the reputation
update containing the reward for the responding peer. This method views
reputation more like currency, to trade work done for other peers against work
demanded of other peers.

Credit-only is the simpler of the two possibilities and works fine if we assume
peers have comparable query behavior. But if there are some power users sending
many more queries than other peers, the relationship of work put in vs. work
consumed becomes unfairly skewed towards those peers not querying very much.
After all, there are more queries in total that need to be answered (not
responding is not an option since it will yield penalties), but they are not
getting anything more out of it.

Credit-debit could address this: Peers using the service more heavily have to
contribute more, in effect taking load off of moderate consumers. But heavy
users must also be given the opportunity to earn more reputation by increasing
their contribution. Load balancing, described in
Section~\ref{sec:desc_load_balancing} has to be used to address this. Heavy
users have to take responsibility for more records in order to receive more
queries, thus gaining the additional reputation they need in order to use the
service to the extent that they wish.

\section{Going Offline}
The system so far doesn't consider that peers may want to go offline. If they
do, they are still expected to respond to queries. Consequently, their
reputation will crash once they do, and they have to work their way back up from
zero when they come back online.

What's missing to make going offline convenient is a way to mark oneself as
inactive, saving reputation until later. Such a mechanism must be proofed
against abuse, or peers would mark themselves inactive almost all the time, only
becoming active when they want to send a query. It is also not ideal for peers
whose connection may be disrupted unexpectedly, like ones on mobile connections.

\section{Peer Selection Overlap $\rightarrow$ Reputation Sorted}
There are a few possible issues with the peer selection strategy \emph{overlap
$\rightarrow$ reputation sorted} presented in
Section~\ref{sec:rep_avail_selection_overlap_rep_sorted}.

\subsection{Contented Peers}
\label{sec:rep_avail_selection_rep_sorted_contented}
In the simulation, all peers are willing to contribute for the best possible
service, i.e. they do their best to answer queries until they are reputation
saturated. Contented peers, who are satisfied with the lowest level of service
would not contribute and remain at 0 reputation. They would thus be strongly
favored for selection, but never actually answer any of the queries.

This could have disastrous results on overall performance for queries where the
group of peers with maximal overlap contains a contented peer. Firstly, in the
case where querying peers only send one query, the first query is always going
to fail, adding a timeout to the total lookup time.

Secondly, in case the querying peer sends two (or more) queries, the one
addressed to the contented peer is wasted; it was meant to allow the
lowest-reputation peer to gain reputation, but this opportunity is given to the
contented peer, who will not seize it. In this case, the purpose of this
selection strategy, to enable low-reputation peers to get started and in effect
to increase social value, is not served.

Possible solutions to this problem could be based on treating the contented peer
differently from the other peers. First, this requires detecting that the peer
is indeed contented. If he is nice, he could announce this upon joining the
group (he may still be tolerated in the group if he gives rewards for successful
queries). If he isn't, detection would have to be based on experience: a peer
who never responds to queries is potentially contented.

Once there is a suspicion that a peer is contented, he, and any other contented
peers, can be skipped when selecting peers. That is, when making the list of
viable potential recipients, contented peers are excluded. If the peer really is
contented, he shouldn't mind not receiving the query, and certainly not
complain, out of laziness. If he does complain, he would be better classified as
malicious, since he takes action that is not to his own benefit, but to the
detriment of the group.

Contented peers who are so lazy they don't even bother giving rewards for
successful responses they received could be dealt with via the complaint system
or the query group management. Peers who should have gotten a reward but didn't
can complain or just make an effort to have the peer removed from the group.

\subsection{Enforcement}
\label{sec:selection_overlap_rep_sorted_enforcement}
The downside of this strategy is that there is no inherent incentive for peers
to use it. It may provide good social value, but for any individual choice a
querying peer could decide that the lowest-reputation peer is less likely to
answer, or likely to take more time, than some other one. This could be from
previous experience, or from the assumption that a peer with low reputation in
one query group also has low reputation in other query groups, leading to the
recursive query problem and longer response times.

So in order for this strategy to be implemented in a real system, the rules of
the system must make querying the lowest-reputation peer mandatory. The querying
peer may send multiple queries, but for any query group he shares with the
recipient of a query, the lowest-reputation peer in that query group with
maximal overlap must also be sent one. A behavior conformant with this rule
would be for the querying peer to use whatever criteria he wishes to select the
recipient, and then to additionally query the lowest-reputation peer.

This rule has to be enforced to be useful. The enforcement process encompasses
two things: detecting that the querying peer failed to also send a query to the
lowest-reputation peer, and agreeing whether a penalty should be applied to the
querying peer and a compensation to the lowest-reputation peer.

The first of these problems may be tackled in one of two ways: the
lowest-reputation could monitor reputation updates in the query group. He then
knows a querying peer failed to send him a query if there is a reward or penalty
applied by the querying peer to some other peer for a query the
lowest-reputation peer should also have received but didn't. This requires
reputation updates regarding responses to include which target ID they're about.
However, this has privacy implications (see
Section~\ref{sec:selection_overlap_rep_sorted_privacy}), and there is neither
any incentive for the querying peer sending the reputation update to include
truthful information and thus risk a penalty, nor is there any incentive for the
subject of the reputation update (the one being given a reward or penalty) to
correct false information. Since no one else knows the truth about the target
ID, this approach is not promising.

The other approach to detecting missing queries to the lowest-reputation peer is
to let all recipients of queries check if there is another peer in the query
group with equally large overlap, but less reputation than them. The actual
recipient of a query has all the necessary information and so could raise this
issue. Then, if this leads to the lowest-reputation peer receiving compensation,
the actual recipient would receive a reward as well, as a sort of finder's fee
to incentivize the checking.

This approach can cause trouble in the presence of contented peers. The proposed
solution to dealing with those was to ignore them and not send them queries. But
if another peer then notices this and starts the enforcement process, the
querying peer may end up being penalized. This brings back all the problems of
contented peers blocking actually contributing peers from gaining reputation. A
possible modification of the enforcement could require the lowest-reputation
peer (the contented peer in this case) to make the complaint himself. Under the
assumption that the contented peer truly is contented, he would be too lazy to
even do that. In addition, if the actual recipient of the query who noticed the
lacking query to the lowest-reputation peer is aware that he is contented, he
need not bother informing him, since there is no hope of a finder's fee anyway.

The second stage of enforcement is coming to an agreement whether a penalty
should be applied to the querying peer and compensation awarded to the
lowest-reputation peer. This must be done via the complaint system, since no
other than the two peers involved know the truth about whether a query was sent.

\subsection{Privacy Implications}
\label{sec:selection_overlap_rep_sorted_privacy}
Employing this strategy has privacy implications. The entire point of the
reputation system is to give each peer many choices where to forward queries to,
spreading queries out and thus make profile building harder.

This strategy demands that a querying peer tell one other specific peer (the
lowest-reputation peer) which target ID he is interested in. But the querying
peer may not want to do that, for whatever reason. Maybe he doesn't trust the
peer (from previous experience, maybe he believes he is being followed), maybe
he has already sent him plenty of queries in the past and doesn't want him to
have too much information (see Section~\ref{sec:desc_stalking}).

Peers who are very sensitive to this could of course be in many more query
groups than is required for subprefix coverage, giving them a wider choice of
peers to query and making it less likely they feel forced into a bad situation.

A modification alleviating this issue is to only require a peer to query the
lowest-reputation peer if he has less than the penalty threshold reputation
(getting everyone above that threshold quickly is the purpose of the strategy,
after all). Assuming that peers generally do well, this would further reduce the
number of instances of having to send unwanted queries.

There is another issue that was mentioned in
\ref{sec:selection_overlap_rep_sorted_enforcement}: if reputation updates
regarding responses must include the target ID, this is equivalent to
broadcasting to the entire query group the target ID of each query, a much more
severe infringement than the one described.

\section{Possible Attacks}
This section describes a few vectors of attack on the system that have been
considered, but doesn't claim to be exhaustive. Some possible solutions are
discussed, but none have been implemented or examined rigorously. Two kinds of
attack can very broadly and clearly be distinguished:
\begin{itemize}
\item Trying to cheat the system into granting a better quality of service than
the peer is intended to receive based on his cooperation. The perpetrators are
selfish peers, their capabilities can be considered to be limited.
\item Attempting to nullify the privacy gains by tracking or building profiles
on either particular, multiple, or even all peers. The attackers are the
adversaries from the initial motivation for the system: data collectors ranging
from private companies trading in user data to government agencies. They should
be considered high-tech threats.
\end{itemize}

\subsection{Collusion/Sybil Attacks}
\label{sec:desc_collusion_sybil_attacks}
The system as is enables a very simple attack by a pair of colluding peers
(irrespective of whether they're actually separate peers or the same peer using
different IDs). The two peers join a query group together and give each other
rewards for successfully answered queries (they just need to pass each other
cooperation confirmations, no queries need to actually take place).

One quick fix to Sybil attacks is to require rewards to not just come from a
separate ID, but from a separate network address as well. However, this causes
problems for legitimately separate peers sharing a connection, and isn't
effective considering IPv6.

Possible solutions also targeting separate colluding peers are more complex. The
peers' ability to continually reward each other may be impeded by limiting the
volume of reward a peer can give any other given peer in a given time. The
reputation record would include who gave whom which rewards in the recent past
(this would require $\mathcal{O}(n^2 \cdot m)$ space in a query group with $n$
peers, each giving $m$ rewards to each other peer in this time frame). Instead
of completely disabling rewards, they could also just become smaller. After some
time of not applying rewards, peers would regain this ability.

With the time frame properly configured, this would already make it a good deal
harder for peers to collude. They can't simply ignore other peers' queries since
they don't have an unlimited supply of free reputation.

The colluders may then find more peers to collude with, to at least increase
that supply. But at some point, the colluders make up a large share of the peers
in the query group. The other peers, whom the colluders have come to exploit,
are likely to leave if they continue to not have their queries answered.

Implementing this solution would have side effects. The overhead of storing the
timeouts for every pair of peers has been mentioned already. Additionally, peers
receiving queries would lose their incentive to respond if they have received
the maximum number of rewards from the querying peer already. Everyone would
have to have more query peers so to be able to continue querying. This means
either larger query groups, or membership in more query groups than otherwise,
in either case a larger management overhead.

In a more extreme variant of the previous solution, the volume of reward any
peer can apply to each other peer could be restricted absolutely. Once it's been
used, that peer can't be rewarded anymore and thus has no incentive to respond.
If the relationship with that peer was important, the peer may consider the
query group to be "used up", and leave. Of course, there would have to be a way
to prevent a peer from leaving and immediately rejoining in order to reset the
counter.

This can even be combined with the first solution and should be even more
effective at combating collusion. It also brings the added side effect that all
peers are forced to switch query groups frequently.

Another approach could target colluding peers' ability to join the same query
group. Since there is no way of telling whether two peers are colluding, it
would have to be aimed at preventing pairs of peers from joining in close
temporal proximity. Whether a peer is accepted into the query group is a policy
decision by the group's members, so they have to consider it in their interest
to prevent collusion.

They could choose not to accept a peer into the group shortly after a new member
has been accepted, as would be the case with two colluding peers. With a more
complex policy, they could choose to only accept an applicant if the newest
member has already gained some reputation (and thus proven himself to be not
completely useless, as a peer waiting for his colluder would be). They could
also choose to refuse an applicant if the newest member of the group is acting
as entry peer, which is the easiest way for colluding peers to share a group.

None of these proposed solutions offer a way to detect or deter colluders with
certainty, but they may make it harder for them, possibly hard enough that it
isn't worth the effort. But whether this is the case is beyond the scope of this
thesis.

\subsection{Generous Peers}
\label{sec:desc_attacks_generous}
Generous peers have been described in Section~\ref{sec:desc_generous_peers}.
The motivation for their generosity may not be selfless. Firstly, responding
without delay regardless of the querying peer's reputation attracts traffic from
peers with low reputation. This counteracts one of the motivations for creating
the system, as the generous peer is better able to build a user profile of these
low-reputation peers sending him queries.

Secondly, it lowers the usefulness of having good reputation and thus lowers
peers' incentive to even respond to queries. While regarding the first point one
could argue that it's up to the peers to not query the same peer repeatedly,
this consequence makes just that harder. With all of those peers who don't care
very much about privacy using the generous peer's service but not contributing
themselves, those peers who do care have fewer options of peers to query.

Placing a large number of generous peers may act as part of a strategy by a
high-tech attacker aimed at attracting enough traffic to track users again.

\subsection{Stalking}
\label{sec:desc_stalking}
The motivation for using the system is that it makes tracking users and profile
building harder. Attackers may try to do it anyway by following peers. In order
to learn anything about a target peer's query behavior, the attacker must share
at least one, and should probably aim to share all, of the target peer's query
groups.

The attacker has to find and join query groups the target peer is in. He can do
this by querying for the target peer's routing prefix, pretending that he is
trying to complete his subprefix coverage. If he wants to avoid arousing
suspicion, he needs to have the right routing prefix, one which has the target
peer's routing prefix as a subprefix (i.e. it is that subprefix with the final
bit flipped). Assuming the routing prefix has a length of $p$, he can expect to
have to generate $2^{p-1}$ random IDs to get one that matches. This process
should eventually find at least some of the target's query groups.

A sufficiently paranoid peer should keep a list of query peers he has interacted
with recently, so to avoid them for a while and make it hard for anyone to track
him. So the attacker needs to generate a new ID for every query group the target
joins, and likely a new network address as well (these are known within a query
group and can also be temporarily avoided by the target).

\subsubsection{Passive Observation}
Once the attacker shares a query group with the target, his simplest option is
to passively listen. He knows when the target sends a query by observing the
reward the target applies to the query's recipient, or the complaint of the
recipient about the missing cooperation confirmation. By observing rewards
applied to the target in other query groups, he can infer whether that query was
for the target himself, or done on behalf of another peer. He doesn't know
exactly what the target queried for, but it's probably something the recipient
of the query is closer to.

\subsubsection{Actively Attracting Queries}
The attacker can also be more active and try to attract queries from the target.
To make this more likely, he must be useful to him, meaning he needs more
observation peers with different IDs (serving the target's subprefixes) in the
query group. The attacker's first peer can act as entry peer for them. If the
query peer selection strategy \emph{overlap $\rightarrow$ reputation sorted}
(see Section~\ref{sec:rep_avail_selection_overlap_rep_sorted}) is mandated, the
attacker should aim for the observation peers to always be the lowest reputation
peer for the given subprefix to make it likely to receive queries from the
target\footnote{It's still not guaranteed since there may be another peer with a
longer overlap. The attacker may choose to use even more observation peers, in
the extreme one mirroring the routing prefix for each existing peer in the
group. This would likely be suspicious to a paranoid peer, though.}. For all
queries sent to one of the observation peers, the attacker knows exactly what
the target is querying for.

Getting all of the observation peers into the query group can be difficult
already if the collusion countermeasures described in
Section~\ref{sec:desc_collusion_sybil_attacks} are implemented. They make it
harder to get multiple peers into a group at once. In that case, a paranoid
target could make the attacker's life harder by switching query groups quickly
so the attacker never gets a good foothold in any of them.

The target may also elect to never query a peer who joined the group after him.
This would disable that attack vector unless the attacker joins query groups
preemptively.

In another possible attack not vulnerable to that countermeasure, the attacker
could attract queries from the target's query peers instead of from the target
himself. Any query received from one of them that is followed by either a reward
from the target, or a complaint about the target because of a missing
cooperation confirmation, was likely done on behalf of the target. Thus the
attacker knows what the target queried for. This attack requires joining many
more query groups though, as each of the target's query peers will be in groups
he doesn't share with the target.

\subsubsection{Total Surveillance}
The most extreme attack is to strive to have a presence in every query group.
The attacker can then observe query chains through different query groups down
to sync group precision (i.e. know the routing prefix of the target ID of any
given query) just by correlating query rewards or complaints about missing
cooperation confirmations.

\subsubsection{Cost}
The cost for these attacks mostly comes down to generating the required IDs and
possibly acquiring the needed network addresses. Since the IDs must have
specific routing prefixes, some brute forcing is necessary.

For each of the generated IDs, the attacker needs a separate network address,
assuming the target is paranoid enough to check for this.

To make generating all these IDs more time consuming, the length of the routing
prefix could be extended, thus increasing the number of IDs required. However,
this causes problems in the system, as explained in
Section~\ref{sec:desc_empty_sync_groups}. It also means everyone needs more
query peers for complete subprefix coverage.

It could also be made harder to generate an ID itself. The ID is assumed to be
the hash of the peer's public key. This hash function could be chosen as one
with adjustable work factor. Paranoid peers could elect not to query peers whose
ID has a low work factor, or even leave query groups if such a peer joins to
avoid passive observation.

The attacks described here are likely too difficult to be practical for low-tech
attackers, given paranoid and vigilant targets, especially if they are not aimed
at just a few specific peers. But they are probably in reach of high-tech
attackers, especially if the attack is targeted at a specific peer, even though
they are likely not trivial.
