\chapter{Future Work}
\label{chap:future_work}
- going offline
- consider allowing to send a response immediately if it's a "no such peer"
  response, regardless of penalty delay. maybe even for fail responses. would
  speed up queries. those peers should be penalized, but it's not their fault
  the other peer didn't know the answer

\section{Query Group Creation and Maintenance}
- TODO mention here that this has not been implemented, is only a proposal with
  potentially unforeseeable dynamics, and considered out of scope?

Query groups need to be created and disbanded, and peers need to be able to join
and leave them. There is no mechanism actually implemented for this in the
simulation included in this thesis. This is an outline of a protocol, but there
are likely to be complex dynamics involved causing unforeseeable problems.

\subsection{Leaving and Disbanding}
Leaving a query group can be achieved by simply broadcasting a message
indicating this to all members (TODO similar concurrency problem as the one
described in rep management: if a new member joins and gets the old state of the
group where the one leaving was still included, requires special handling, or a
confirmation that the current state is correct).

A query group is considered disbanded if it only contains one peer, i.e. once
the second-to-last member leaves.

\subsection{Creation}
A simple way to create a query group is for one peer to ask another to start
one, thus creating a query group containing 2 peers. This group would then grow
by new peers joining it. However, this assumes that it is reasonably easy for
one peer to find another with whom he is interested to share a query group and
who shares that interest. This may not be the case; a peer can send queries for
prefixes he is interested in, but the peers in the responses may not be
interested in him.

Peers are interested in others so that they can extend their subprefix coverage,
or get some redundancy into their coverage. For any pair of peers A and B, they
are either responsible for the same routing prefix, or A is responsible for
exactly one of B's subprefixes, namely the one of length $i$, where $i$ is the
one-based index of the first 1 in the bitwise XOR of the peer's respective
routing prefixes. So every peer can serve some subprefix for any other peer, and
the subprefix relationship between two peers is even symmetric, i.e. the
subprefix A can serve for B has the same length as the one B can serve for A
(they differ in the final bit). But that doesn't necessitate B being interested
in sharing a query group with A: he may already know enough peers for that
subprefix, or just not trust A for some reason.

A more complex way of creating a query group could involve the initiating peer
considering which subprefixes other peers are even interested in. This could be
announced in a peer's record that's stored in the DHT. The initiating peer would
check that the other peer is interested in extending his coverage to a subprefix
he is responsible for.

A query group could also be initiated with more than just 2 peers. The
initiating peer would contact all peers he wants to invite and propose to start
a group, attaching a list of members. If all agree, the initiating peer
broadcasts reachability information of all members. This allows for advanced
strategies in finding peers, where peers aren't pairwise interested in each
other, but circularly. E.g. A is interested in B, B in C, and C again in A.
Having more peers can also be helpful since it means more traffic (in small
groups, it can be hard to gain reputation for lack of queries).

\subsection{Usefulness}
\label{sec:desc_usefulness}
In this context it is convenient to introduce the notion of \emph{usefulness}:
it is the value peer A places on knowing peer B, with regards to his subprefix
coverage. It is a policy decision, not something the rules of the system can
dictate, and may depend on such things as the number of peers peer A already
knows for the subprefix, or whether he trusts peer B (maybe he feels like he is
being followed).

These factors are not visible to peer B, the length of the subprefix he can
serve is, though. All else being equal, peers serving longer prefixes are more
useful than those serving shorter ones. That's because peers for shorter
subprefixes are easier to find: consider a subprefix of length 1, which can be
served by all peers whose routing prefix begins with that single bit. Assuming
evenly distributed IDs, that's half the population. Longer subprefixes are also
closer to a peer's own routing prefix, and the peer is likely to get more
queries for it, making more redundancy a handy thing to have.

The length of subprefix a peer can serve for another peer could be a heuristic
for the peer to estimate how useful he is to the other peer. And, more
crucially, how useful he would be in a query group, by calculating the sum of
the lengths of the subprefixes he can serve for each peer (or e.g. the sum of
squares, to give longer subprefixes a higher weight). This can serve as an
estimator for how many queries he is likely to receive in that group, and thus
how easily he can gain reputation (see chapter~\ref{chap:rep_avail},
particularly Section~\ref{sec:rep_avail_group_reeval}) (TODO is that section
still in?).

\subsection{Inviting and Joining}
Joining a query group must be done via an \emph{entry peer}, a current member of
the group who handles the joining process. The membership candidate may ask the
entry peer to join, or receive a proactive invitation from him. Since a new
member affects everyone in the group, the entry peer should not be allowed to
make the decision by himself.

The simplest way is a majority vote: The entry peer presents the candidate to
the current members, and if more than half of them agree that he may join, he is
accepted. A modification of this is to require a unanimous vote. The members may
use their own judgement in this decision, e.g. considering how useful the
candidate would be to them, what impact he might have on the total traffic in
the group (especially with regards to reputation updates), or whether they
distrust him from previous experience.

If a candidate is accepted that not all current members agree with, these
members may leave the group, thus the candidate threatens the integrity of the
group. Therefore, it may be helpful to hold the vote in a way that asks the
current members whether they would leave the group if the candidate was added.
Again, this vote is up to peer policy. They may consider the peer untrustworthy,
not useful to them, or simply deem the group to big after his addition.
This would have to be done in multiple rounds: Peers announcing they'd leave in
the first round may prompt additional peers to want to leave, who were there
for connection to those peers. This is repeated until no additional peers
announce their intention to leave.

After all peer intentions have been collected, the decision to accept or reject
the candidate has to be made. It could either be made by the entry peer alone,
assuming that everyone who isn't going to leave is alright with the new peer
joining, or again by a vote (majority or unanimous) of the members who would
remain after the cadidate joins.

This approach assumes peers always have the option to leave a query group. But
some may depend on the group because their subprefix coverage would become
incomplete if they left. These peers would be pressured to stay and have no real
say in accepting or refusing the candidate.

In a possible extension, those peers that leave the group after that candidate
joins may automatically form a new query group. Candidates looking to join
groups then act as a sort of catalyst to split groups into smaller parts.

The process of deciding on a candidate's membership is likely to entail complex
dynamics. Both the simple voting and the other approach with peers announcing
they'd leave is highly dependent on peer decisions, and neither is clear to
yield a system that allows a good distribution of peers in query groups of
useful size (i.e. not just 2 peers per group).

- TODO find and cite consensus algos?

\subsection{Finding Groups}
In order to request to join a query group, a peer first needs to find one he
would like to join. After all, there is no public record of all available query
groups.

To do this, the query mechanism could be extended to allow queries not just for
peers (i.e. their record in the \ac{DHT}), but for query groups. The peer
looking for a group would send a query to some of his current query peers
stating he is looking for a group. In that query, he lists one or more of his
subprefixes for which there should be responsible peers in the group. If the
recipient of the query is in a group matching these criteria, he may offer to
act as the entry peer and receive a reward in the current query group for a
successfully answered query.

In addition to listing subprefixes that should be covered, the peer looking for
a group could request that his usefulness in the prospective group be calculated
or estimated, and only groups offered where it exceeds a threshold. This may be
necessary to ensure the new peer even gets any queries and can gain reputation
in order to make use of his new group.

A peer completely new to the system needs some way of entering his first query
group. He needs to know at least one peer already in the system to get started
(see Section~\ref{sec:desc_bootstrapping}), who must act as entry peer to at
least one query group.

Again, it is not at all clear that this proposal would work in practice. If this
process is used alone, it may well happen that the network becomes fragmented,
with some parts of it being entirely unaware of others, and thus unable to
access records stored there. There is also no solution for the problem that a
peer new to the system is rejected in all query groups he is offered in the
beginning.

- TODO concurrency issues all over the place

\section{Gossipping}
Instead of broadcasting the reputation updates, it may be possible to use
gossipping to spread the traffic load. However, all the peers sending messages
then have to have incentive to do so. If they are given rewards, depending on
the gossipping protocol, peers low on reputation may opt to send many messages
to gain lots of reputation, leading to redundant traffic. On the other hand, if
lots of peers already have enough reputation, they may opt not to send anything
and gossipping fails. Therefore, this is considered out of scope.

- TODO gossipping in sync groups
In case gossipping should be used for the updates to records within sync groups,
finding incentives becomes more difficult. Gossipping spreads the load of (TODO)    
disseminating the update to multiple peers, and removes the need for all sync
peers to know one another. One might imagine a separate reputation system for
sync groups, but that's out of scope for this thesis.


\section{Complaint System}
\label{sec:desc_complaints}
In certain situations, particularly regarding rewards or penalties being applied
or not, a peer may treat one of his query peers unfairly, but deny it. Usually,
no one but the two peers involved know what actually happened, and there is no
way for the wronged peer to prove it. The two most important examples are:
\begin{itemize}
\item Peer A queries peer B, B responds successfully. But then A fails to send
      the cooperation confirmation that allows B to claim a reward.
\item Peer A sends a reputation update containing a penalty for a peer B (thus
      receiving a small reward himself), but B didn't actually do anything
      wrong.
\end{itemize}

There may be multiple reasons for a peer to misbehave in such a way. In the
first example, it may be laziness (sending the confirmationyields no benefit to
the peer). The second example may be due to greed, since the peer receives a
small reward for broadcasting a reputation update. But it needn't necessarily be
selfishness, it could also be due to a tremble: Some part of one peer's network
stack at the lower layers may have failed, or a message may have been lost or
delayed in the network. Then the protocol implementation at both peers may
rightfully assume that, given the information available to it, it is in the
right.

In both examples, only the two involved peers know what happened, and neither
can prove it to anyone else: In the first case, B has no way of proving that A
didn't send the cooperation confirmation, not even that A sent a query in the
first place, much less that he answered it successfully (and in time). In the
second case, peer B can't prove he didn't do anything wrong. Peer A may for
example have queried him with no reaction from B, and again B can't prove that
he didn't receive a query.

To resolve situations like these, there must be a complaint system within each
query group, via which one peer can \emph{complain about} one of his query
peers. Such a complaint may be accepted, resulting in a penalty for the subject
of the complaint and a reward for the complaining peer (as compensation), or it
may be rejected. Of course, this system can't determine the truth either, but it
should not be continually exploitable, i.e. a peer should not be able to
continually misbehave without receiving a penalty through the complaint system,
and a peer should not be able to continually lodge false complaints against
another peer that are accepted.

- needs to ensure that it's a dominant strategy not to lodge false complaints,
  and to cooperate

As with query group creation, no complaint system has been implemented in the
simulation covered in later chapters. These are proposals (TODO plural?) for a
protocol that may work, but needs closer examination and possibly modification
due to the complex dynamics that will likely arise.

- TODO possibly leave this out, just state that tit-for-tat is a dominant strat?
  but that doesn't take into account trembles, one of those and a peer
  relationship breaks down
- w/o the system, peers could just retaliate: A applies penalty to B, B to A
  etc., the query group would break down (even though the peers playing
  tit-for-tat should be dominant)
- complaint system must offer a better alternative to retaliation
- maybe a complaint means the involved peers don't have to do anything for each
  other, can't get rewards or penalties from each other?

- majority vote, could be abused up to a point
- automatically believe an indictment if enough (n/2+1) indictments heap up.
  believe all following indictments as well?
- no chance if colluding peers
- at some point, it's best for a peer whose complaints keep getting rejected, to
  leave the group
- complex dynamics

- TODO paper Optimizing an Incentives’ Mechanism for Truthful Feedback in
  Virtual Communities may be the solution

\section{Load Balancing}
\label{sec:desc_load_balancing}
In the system as described, every peer is responsible for his routing prefix,
which is just a prefix of his own ID. Assuming IDs are evenly distributed (which
is the case if they are fingerprints of public keys and peers don't make
attempts to generate a particular key), this implies the expected value of the
number of peers responsible for a record is equal for all records. This is fine
under the assumption that all records are queried for with the same likelihood
(which is in fact what's happening in the simulation, where no load balancing is
necessary).

A more realistic record access pattern follows a power law distribution, with
very few records forming a large share of the total number of queries (TODO cite
something). Taking this into consideration, distributing the load of responding
to all the queries to the popular record becomes necessary, or else those peers
who happen to have a routing prefix matching one of those popular records
receive an unfairly disproportionate amount of queries. They may be inclined to
generate a new ID that doesn't suffer from this.

One of the central tenets of P-Grid involves separating a peer's routing prefix
from the records he is responsible for. Instead of a peer's ID determining which
records he has to store, each peer explicitly advertises one or more prefixes
which he is responsible for (this can become part of their record stored in the
\ac{DHT}). These new prefixes don't even all have to be the same length,
allowing for finer grained control. Being responsible for more prefixes means
peers have to join more sync groups and completing their subprefix coverave may
require more connections.

With this system in place, it becomes possible for more peers to take over
responsibility for prefixes containing popular records, thus balancing the load.
In P-Grid, this is done because peers are well-intentioned and willing to
cooperate (TODO cite). Under the assumptions made by the system in this thesis,
peers need an incentive to do so. They may have such an incentive because they
are unable to gain reputation quickly enough because they aren't getting enough
queries. Taking on responsibility for a popular prefix can address this.

A new kind of query could help peers decide which new prefixes to cover, in
which they query for a prefix which receives a lot of traffic and the peers
responsible for it would like the load lightened. Ideally, this system would be
self-organizing, like P-Grid is, but via economical incentives. But this has not
been implemented for this thesis, and doing so is likely to bring to light
complex dynamics not foreseen here.

\section{Iterative Queries with Vouchers}
\label{sec:desc_iterative_vouchers}
Instead of recursive queries, iterative queries could be used. With this method,
instead of the recipient of a query handling all the work and returning the
final response, he is only expected to find a suitable next peer to query, i.e.
one who is closer to the target ID, and respond with this peer's reachability
information. The querying peer must then query that peer himself. This solves
the recursive query problem, because the recipient of a query only needs to do a
local lookup and is not dependent on another peer.

It introduces a new problem, though. There is no guarantee, and with growing
network size it becomes increasingly unlikely, that the querying peer shares a
query group with the peer the first recipient said to contact next. That peer
has no incentive to respond to a peer with whom he shares no query group.

A possible solution to this problem is for the responder to pass on his level of
service. Say peer A has queried peer B, who responds with reachability
information for peer C. The response now also contains a \emph{query voucher},
which is a statement signed by B that allows A to send a query to C with the
quality of service that B would have gotten. A can then query C, and C must
treat his as though he were B. This entitles B to a reward for a successful
response, even if the level of service ultimately received by A is bad (because
B has low reputation in the query group he shares with C). After all, B
demonstrated his best effort to answer the query.

Query vouchers must be single use to prevent peers from collecting vouchers for
all their subprefixes and then leeching off of another peer's good reputation.
First off all, this means they have to contain some unique ID. Secondly, it is
either required that the peer's with whom the voucher is redeemed keep a history
of all vouchers that have been redeemed, or the peer who issues the voucher must
authorize the particular voucher for one use.

They also need to have a (relatively short) expiry time to prevent peers from
stockpiling vouchers and then leeching off of the issuers good standing. This
again requires an expiry time on the voucher, as well as clocks within the
network to be reasonably in sync.

Query vouchers also require the complaint system to be extended. Using peers A,
B and C from the example, A needs to have a threat of penalty for not sending a
cooperation confirmation to C after a successfuly query. But C can only complain
about B, so there must be a new complaint type that allows B to complain about A
for having received a complaint after use of a voucher. Then C complains about
B, and B complains about A.

\section{Credit-only vs. Credit-Debit Reputation}
\label{sec:desc_credit_only_vs_credit_debit}
The system uses credit-only reputation, i.e. no reputation is deducted for using
the system (e.g. per query) (TODO ref background). Once peers have gathered at
least the penalty threshold reputation, they are entitled to delay-free service.
They only lose this privilege if they incur a penalty, most notably for failing
to respond properly to a query. As long as this doesn't happen, they can send as
many queries as they like. With this method, having reputation is seen as proof
that a peer is generally willing to contribute, to play a part of the whole. It
is unnecessary to deduct this reputation, as querying someone else doesn't
change that assessment.

In the alternative, credit-debit, each query a peer gets successfully answered
costs him a little reputation as a sort of payment. This could be incorporated
into the system quite easily, by incorporating this penalty into the reputation
update containing the reward for the responding peer. This method views
reputation more like currency, to trade work done for other peers against work
demanded of other peers.

Credit-only is the simpler of the two possibilities and works fine if we assume
peers have comparable query behavior, as the simulation does (TODO ref
implementation). But if there are some power users sending many more queries
than other peers, the relationship of work put in vs. work consumed becomes
unfairly skewed towards those peers not querying very much. After all, there are
more queries in total that need to be answered (not responding is not an option
since it will yield penalties), but they are not getting anything more out of
it.

Credit-debit could address this: Peers using the service more heavily have to
contribute more, in effect taking load off of moderate consumers. But heavy
users must also be given the opportunity to earn more reputation by increasing
their contribution. Load balancing, described in
Section~\ref{sec:desc_load_balancing} has to be used to address this. Heavy
users have to take responsibility for more records in order to receive more
queries, thus gaining the additional reputation they need in order to use the
service to the extent that they wish.

\subsection{Network-Optimized Queries}
There are no provisions in the system to choose recipients in a way that is
optimized with regards to network topology or geography. As it stands, every hop
in a chain necessary to resolve one initial query may be transmitted to another
continent, making queries potentially run very long.

Even if considering their own interest, peers may not have an incentive to query
a peer close to them in the network if they have a bad reputation in the query
group they share with him.

If such an extension is desired, peers need a way to find new query peers
(serving a specific subprefix) that are close to them in the network.

Then again, being able to spread out ones queries to a large pool of peers is a
design goal of the system, it actually happening a consequence.

\subsection{Going Offline}
- TODO future work

The system so far doesn't consider that peers may want to go offline. If they
do, they are still expected to respond to queries. Consequently, their
reputation will crash once they do, and they have to work their way back up from
0 when they come back online.

What's missing to make going offline convenient is a way to mark oneself as
inactive, saving reputation until later. Such a mechanism must be proofed
against abuse, or peers would mark themselves inactive almost all the time. It
is also not ideal for peers whose connection may be disrupted unexpectedly, like
ones on mobile connections.

\section{Possible Attacks}
- TODO these are the reasons for the system's existence
- TODO split between introduction and future work

This section describes a few vectors attacks on the system that have been
considered, but doesn't claim to be exhaustive. Some of them can be dealt with,
other less so with the system as it has been proposed. Two kinds of attack can
very broadly and clearly be distinguished:
\begin{itemize}
\item Trying to cheat the system into granting a better quality of service than
the peer is intended to receive based on his cooperation. This includes tricks
to avoid penalties after defecting behavior, and ones to receive rewards where
they're technically correctly applied, but not in the spirit of the system (TODO
do i even have examples of this?). The perpetrators are selfish peers, their
capabilities can be considered to be limited.
\item Attempting to nullify the privacy gains by tracking or building profiles
on either particular, multiple, or even all peers. The attackers are the
adversaries from the initial motivation for the system (TODO ref?): data
collectors ranging from private user behavior analysts to government agencies.
They should be considered high-tech threats.
\end{itemize}

In this thesis, the former are the primary focus. However, the latter should be
kept in mind, since they are ultimately the reason the system even exists.

- TODO make it clear that the first of the two points is a prerequisite of the
  second. thesis focuses on it because that's the basis for it working.

- TODO previous work has shown that it makes sense to trust new users to some
  degree, though
(TODO ref previous work, which should cite some stuff on this?)

\subsection{Collusion/Sybil Attacks}
\label{sec:desc_collusion_sybil_attacks}
- TODO future work

The system, as it is, enables a very simple attack by a pair of colluding peers
(irrespective of whether they're actually separate peers or the same peer using
different IDs). The two peers join a query group together and give each other
rewards for successfully answered queries (they just need to pass each other
cooperation confirmations, no queries need to actually take place).

One quick fix to Sybil attacks is to require rewards to not just come from a
separate ID, but from a separate network address as well. However, this causes
problems for legitimately separate peers sharing a connection, and isn't
effective considering IPv6.

Possible solutions also targeting separate colluding peers are more complex. The
peers' ability to continually reward each other may be impeded by limiting the
volume of reward a peer can give any other given peer in a given time. The
reputation record would include who gave whom which rewards in the recent past
(this would require $\mathcal{O}(n^2 \cdot m)$ space in a query group with $n$
peers, each giving $m$ rewards to each other peer in this time frame). Instead
of completely disabling rewards, they could also just become smaller. After some
time of not applying rewards, peers would regain this ability.

With the time frame properly configured, this would already make it a good deal
harder for peers to collude. They can't simply ignore other peers' queries since
they don't have an unlimited supply of free reputation.

The colluders may then find more peers to collude with, to at least increase
that supply. But at some point, the colluders make up a large shared of the
peers in the query group. The other peers, whom the colluders have come to
exploit, are likely to leave if they continue to not have their queries
answered.

Implementing this solution would have side effects. The overhead of storing the
timeouts for every pair of peers has been mentioned already. Additionally, peers
receiving queries would lose their incentive to respond if they have received
the maximum number of rewards from the querying peer already. Everyone would
have to have more query peers so to be able to continue querying. This means
either larger query groups, or membership in more query groups than otherwise,
in either case a larger management overhead.

In a more extreme variant of the previous solution, the volume of reward any
peer can apply to each other peer could be restricted absolutely. Once it's been
used, that peer can't be rewarded anymore and thus has no incentive to respond.
If the relationship with that peer was important, the peer may consider the
query group to be "used up", and leave. Of course, there would have to be a way
to prevent a peer from leaving and immediately rejoining in order to reset the
counter.

This can even be combined with the first solution and should be even more
effective at combating collusion. It also brings the added side effect that all
peers are forced to switch query groups frequently.

Another approach could target colluding peers' ability to join the same query
group. Since there is no way of telling whether two peers are colluding, it
would have to be aimed at preventing pairs of peers from joining in close
temporal proximity. Whether a peer is accepted into the query group is a policy
decision by the group's members, so they have to consider it in their interest
to prevent collusion.

They could choose not to accept a peer into the group shortly after a new member
has been accepted, as would be the case with two colluding peers. With a more
complex policy, they could choose to only accept an applicant if the newest
member has already gained some reputation (and thus proven himself to be not
completely useless, as a peer waiting for his colluder would be). They could
also choose to refuse an applicant if the newest member of the group is acting
as entry peer, which is the easiest way for colluding peers to share a group.

None of these proposed solutions offer a way to detect or deter colluders with
certainty, but they may make it harder for them, possibly hard enough that it
isn't worth the effort. Whether this is the case is beyond the scope of this
thesis.

\subsection{Limiting Responsibility}
If sync group sizes are very disparate, a peer may choose one that is very small
(by generating new IDs until the routing prefix matches). Thus, he is
responsible for only a small amount of records. While this makes it harder to
gain reputation initially, once he has done so, maintaining it requires less
effort.

If all peers are willing to generate IDs in order to join the smallest sync
group, this problem really fixes itself, performing load balancing. Otherwise, a
credit-debit system (see Section~\ref{sec:desc_credit_only_vs_credit_debit}) or
reputation decay (see Section~\ref{sec:desc_incentives_rep_mgmt}), together with
explicitly advertised prefixes (see Section~\ref{sec:desc_load_balancing}) may
help.

\subsection{Abusing the Complaint System}
- TODO selfishness: abusing the complaint system (TODO)

\subsection{Generous Peers}
\label{sec:desc_attacks_generous}
Generous peers have been described in Section~\ref{sec:desc_generous_peers}.
The motivation for their generosity may not be selfless. Firstly, responding
without delay regardless of the querying peer's reputation attracts traffic from
peers with low reputation. This counteracts one of the motivations for creating
the system, as the generous peer is better able to build a user profile of these
low-reputation peers sending him queries.

Secondly, it lowers the usefulness of having good reputation and thus lowers
peers' incentive to even respond to queries. While regarding the first point one
could argue that it's up to the peers to not query the same peer repeatedly,
this consequence makes just that harder. With all of those peers who don't care
very much about privacy using the generous peer's service but not contributing
themselves, those peers who do have fewer options of peers to query.

Placing a large number of generous peers may act as part of a strategy by a
high-tech attacker aimed at attracting enough traffic to track users again.

\subsection{Stalking}
\label{sec:desc_stalking}
The motivation for using the system is that it makes tracking users and profile
building harder. Attackers may try to do it anyway by following peers. In order
to learn anything about a target peer's query behavior, the attacker must share
at least one, and should probably aim to share all, of the target peer's query
groups.

The attacker has to find and join query groups the target peer is in. He can do
this by querying for the target peer's routing prefix, pretending that he is
trying to complete his subprefix coverage. If he wants to avoid arousing
suspicion, he needs to have the right routing prefix, one which has the target
peer's routing prefix as a subprefix (i.e. it is that subprefix with the final
bit flipped). Assuming the routing prefix has a length of $p$, he can expect to
have to generate $2^{p-1}$ random IDs to get one that matches. This process
should eventually find at least some of the target's query groups.

A sufficiently paranoid peer should keep a list of query peers he has interacted
with recently, so to avoid them for a while and make it hard for anyone to track
him. So the attacker needs to generate a new ID for every query group the target
joins, and likely a new network address as well (these are known within a query
group and can also be temporarily avoided by the target).

\subsubsection{Passive Observation}
Once the attacker shares a query group with the target, his simplest option is
to passively listen. He knows when the target sends a query by observing the
reward the target applies to the query's recipient, or the complaint of the
recipient about the missing cooperation confirmation. By observing rewards
applied to the target in other query groups, he can infer whether that query was
for the target himself, or done on behalf of another peer. He doesn't know
exactly what the target queried for, but it's probably something the recipient
of the query is closer to.

\subsubsection{Actively Attracting Queries}
The attacker can also be more active and try to attract queries from the target.
To make this more likely, he must be useful to him, meaning he needs more
observation peers with different IDs (serving the target's subprefixes) in the
query group. The attacker's first peer can act as entry peer for them. If the
query peer selection strategy \emph{overlap $\rightarrow$ reputation sorted}
(see Section~\ref{sec:rep_avail_selection_overlap_rep_sorted}) (TODO briefly
explain if this references a future section) is mandated, the attacker should
aim for the observation peers to always be the lowest reputation peer for the
given subprefix to make it likely to receive queries from the
target\footnote{It's still not guaranteed since there may be another peer with a
longer overlap. The attacker may choose to use even more observation peers, in
the extreme one mirroring the routing prefix for each existing peer in the
group. This would likely be suspicious to a paranoid peer, though.}. For all
queries sent to one of the observation peers, the attacker knows exactly what
the target is querying for.

Getting all of the observation peers into the query group can be difficult
already if the collusion countermeasures described in
Section~\ref{sec:desc_collusion_sybil_attacks} are implemented which make it
harder to get multiple peers into a group at once. In that case, a paranoid
target could make the attacker's life harder by switching query groups quickly
so the attacker never gets a good foothold in any of them.

The target may also elect to never query a peer who joined the group after him.
This would disable that attack vector unless the attacker joins query groups
preemptively.

In another possibile attack not vulnerable to that countermeasure, the attacker
could attract queries from the target's query peers instead of from the target
himself. Any query received from one of them that is followed by either a reward
from the target, or a complaint about the target because of a missing
cooperation confirmation, was likely done on behalf of the target. Thus the
attacker knows what the target queried for. This attack requires joining many
more query groups though, as each of the target's query peers will be in groups
he doesn't share with the target.

\subsubsection{Total Surveillance}
The most extreme attack is to strive to have a presence in every query group.
The attacker can then observe query chains through different query groups down
to sync group precision (i.e. know the routing prefix of the target ID of any
given query) by correlating query rewards or complaints about missing
cooperation confirmations.

\subsubsection{Cost}
The cost for these attacks mostly comes down to generating the required IDs and
possibly acquiring the needed network addresses. In the following, $p$ denotes
the (fixed) length of the routing prefix, $m$ the number of members in each
query group, $g$ the average number of query groups each peer is in, and $n$ the
total number of query groups in the system.

To join all of the target's query groups, the attacker must generate
$\mathcal{O}(g)$ specific IDs.

To attract queries from the target directly, with one observation peer per
subprefix, the attacker must generate $\mathcal{O}(g \cdot p)$ specific IDs.

To do this indirectly, attracting queries from the target's query peers, with
one observation peer per subprefix, the attacker must generate $\mathcal{O}(g
\cdot p \cdot m)$ specific IDs.

For total surveillance, the attacker must generate $\mathcal{O}(n \cdot m)$
specific IDs.

- TODO calculate how many random IDs in total until the attacker can expect to
  have all the needed IDs

For each of the generated IDs, the attacker needs a separate network address,
assuming the target is paranoid enough to check for this.

To make generating all these IDs more time consuming, the length of the routing
prefix could be extended, thus increasing the number of IDs required. However,
this causes problems in the system, as explained in
Section~\ref{sec:desc_empty_sync_groups}. It also means everyone needs more
query peers for a complete subprefix coverage.

It could also be made harder to generate an ID itself. The ID is assumed to be
the hash of the peer's public key. This hash function could be chosen as one
with adjustable work factor. Paranoid peers could elect not to query peers whose
ID has a low work factor, or even leave query groups if such a peer joins to
avoid passive observation.

The attacks described here are likely too difficult to be practical for low-tech
attackers, given paranoid and vigilant targets, especially if they are not aimed
at just a few specific peers. But they are probably in reach of high-tech
attackers, especially if the attack is targeted on a specific peer, even though
they are likely not trivial.
