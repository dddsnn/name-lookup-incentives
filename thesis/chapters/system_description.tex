\chapter{System Description}
\section{Goal}
- TODO for all the stuff that's out of scope, should i mention it in this
  chapter, or only in the implementation?

- TODO choose a style for when to use footnotes, and when parentheses

This thesis introduces a system that is meant to extend an existing \ac{DHT}
such that peers wanting to access records stored in it in a timely manner have
an incentive to contribute to the operation of the \ac{DHT}, i.e. to store
records and to answer queries about them. Having many peers contributing is then
supposed to allow users to choose whom they query out of many options, thus
providing the privacy benefit described in the introduction (TODO ref).

The motivating use case is that of a distributed name-lookup service, similar to
DNS, which maps user identifiers to network addresses. Storing
\emph{reachability information} (e.g. IP address) about oneself, where one's own
user identifier is the key to the record is also the only use case
considered\footnote{Other ones may be possible, but are considered out of scope
for simplicity.}. Since this is a somewhat latency-sensitive
application\footnote{"Somewhat latency-sensitive" here means "it would be
annoying if it took longer than 500ms" rather than "it must happen within 20ms
or the entire system breaks".}, the system should cause as few additional delays
during lookup as possible. The records that are stored in the \ac{DHT} are
assumed to be fairly small, i.e. the network delay doesn't have a significant
impact on the total delay. Furthermore, it must be possible to update records,
and for these updates to be disseminated throughout the \ac{DHT} within a
reasonably short time.

An underlying assumption for the most part of this thesis is that the users of
the system are selfish agents who want to use the system and gain access to the
many peers serving the \ac{DHT}, and with good performance. This means they want
to expend as few resources as possible, including storage space, CPU time and
network bandwidth. They only want to use these resources in order to, at all
times, be very likely to be in a position to receive service quickly. There is
no explicit valuation determining what "very likely" means to a peer (TODO isn't
there?). Rather, the reputation buffer described in the implementation chapter
(TODO ref) acts as a proxy for it, determining the reliability of a peer's
ability to receive service.

While this is the usual assumption about peer motivation, other types of peers
are discussed (TODO ref types of peers in this chapter, also contented peers,
maybe something else?).

The system should not have a negative impact on cooperating peers, i.e. peers
who are contributing in the way they are supposed to. Ideally, such peers should
not notice the system being there at all, safe for some management overhead, and
experience the \ac{DHT} as if all peers were cooperating as well.

The system should prevent free riding and be resistant to whitewashing.
Additionally, it should offer some protection against colluding peers trying to
appear as cooperating peers when they are not, and against Sybil attacks (TODO
ref background section for explanation).

\section{Assumptions and Prerequisites}
The system described in this chapter assumes that, for any pair of peers, there
is a reliable way for them to communicate at all times. In practice, the likely
possibility under which this is given is that each peer is able to establish and
maintain a TCP connection with any other peer. Messages may be delayed e.g. due
to congestion, but must arrive eventually. Issues arising from e.g. NAT that
would make it impossible to contact a peer directly, but also complete link
failures that cause connections to time out are considered out of scope.

While network delays can be dealt with, jitter should be low enough that future
delays can be predicted from past ones with reasonable accuracy.

Peers are assumed to have clocks, all of which are synchronized. Minor
variations are tolerable, though (typical accuracy of NTP is definitely enough).

Access to the \ac{DHT} is assumed to be equally distributed. This includes
firstly that the records are equally popular, i.e. each of them has the same
likelihood of being queried for. Secondly, all peers send the same volume of
queries. Sections~\ref{sec:desc_load_balancing}
and~\ref{sec:desc_credit_only_vs_credit_debit} discuss the consequences of these
assumptions not holding.

- TODO not all parts described in organization have been implemented, this has
  to be pointed out consistently

\subsection{Types of Peers}
\subsubsection{Selfish}
Selfish peers are the central type of peer this thesis focuses on, and the one
discussions are about unless otherwise indicated. They are lazy
and unwilling to expend any resources (network bandwidth, CPU time, memory)
unless there is a benefit in it for them.

As long as they see a benefit, they are willing to cooperate with other peers,
but when they don't, they stop taking actions, even if a protocol should require
them to. Finding incentives for this type of peer is a central part of this
thesis, and achieved by threatening to lower the quality of service.
Section~\ref{sec:desc_rationale} goes into more detail on the motivation of
selfish peers.

\subsubsection{Generous}
\label{sec:desc_generous_peers}
Generous peers are in a way the opposite of selfish peers. They don't care about
their own benefit and strive to make others' lives easier, or at least they
behave in such a way. This can actually become a problem, since it reduces the
incentive for other peers to do work, thus counteracting the goal of providing
many peers to interact with. The system has to take measures against this kind
of behavior. Section~\ref{sec:desc_rep_system} goes into more detail.

Generous peers may actually be malicious, trying to attract a lot of traffic in
order to build profiles. This is briefly discussed in
Section~\ref{sec:desc_attacks_generous}, but otherwise out of the scope of this
thesis.

\subsubsection{Contented}
\label{sec:desc_contented_peers}
Contented peers (as in satisfied) don't care about the quality of service they
receive and thus can't be incentivized via it. These could e.g. be running on
slow devices without direct user interaction. If they appear in large numbers,
they may cause problems.
Section~\ref{sec:rep_avail_selection_rep_sorted_contented} discusses one such
problem, but other than that, this type of peers is considered out of scope.

\subsubsection{Colluding/Sybils}
Colluding peers are multiple selfish peers working together in order to do less
work while enjoying the same level of service. Sybils (peers performing a Sybil
attack) (TODO cite sybil or ref background) are single peers acting under
multiple identities trying to achieve the same thing.

These are out of scope of the implementation, but some ideas are presented in
Section (TODO ref) on how they may be dealt with.

\subsubsection{Sabotaging}
Sabotaging peers are motivated by causing harm to others. They are willing to
invest resources into disrupting the system. A subtype are vengeful peers, who
seek to cause harm to particular peers who have wronged them. Both of these are
out of scope.

\section{Organization}
\subsection{Overview}
- TODO style: in the description of the protocol i'm not consistent in "the peer
  does x" vs. "the peer must do x"

The underlying \ac{DHT} for the system is one in the style of Kademlia or P-Grid
(TODO ref background chapter). Every peer has a unique \emph{ID} that he chooses
himself, and every \emph{record} in the \ac{DHT} has a unique ID of the same
length. In fact, each peer may store exactly one record under its own ID, giving
reachability information. As mentioned previously, this constraint was added for
simplicity. Under these circumstances, querying for a record stored in the
\ac{DHT} is done in order to learn how to contact a peer. For this reason, the
phrase "querying for a peer (with some ID)" is frequently used in this thesis to
mean querying for a record in the \ac{DHT} with that peer's ID (which contains
information on how to contact the peer.

These IDs are bit strings of fixed length and routing takes place
along a binary tree formed by them. It may be possible to adapt the system to
work with \acp{DHT} using a different routing scheme, such as Chord. See
Section~\ref{sec:desc_querying} for details on how querying is handled in the
\ac{DHT} used by the system.

Every peer stores a number of records of the \ac{DHT}, he is \emph{responsible}
for these records. Other peers are also responsible for these same records, and
the peer stays in contact with them in a \emph{sync group}. In these groups,
peers synchronize their information about stored records, like the creation of a
new one, or the update of an existing one. See
Section~\ref{sec:desc_sync_groups} for details on sync groups.

The system implements a reputation system in which peers have a
\emph{reputation} that is tracked within \emph{query groups}. A peer has a
separate non-negative, real-valued reputation value in each such group that he
is a member of. This value is initialized to 0 upon entering the group. The
other members of these groups are the peers \emph{query peers}, and every one of
them knows the current reputation of each of the other query peers. They are the
ones to whom the peer forwards queries he can't answer himself. The way query
groups are created, how they are managed, and what dynamics arise from this is
not examined in this thesis, but there proposals and more details on query
groups in Section~\ref{sec:desc_query_groups}.

A peer receives a \emph{reward}, i.e. an increase of his reputation in a query
group if he performs a \emph{cooperative action} towards a peer in that group.
Conversely, he receives a \emph{penalty}, i.e. a decrease of his reputation in a
query group, if he performs a \emph{defecting action} towards a peer in that
group. However, reputation is non-negative, and so penalties have no effect if
the peer has 0 reputation. The terms \emph{cooperative} and \emph{defecting} are
chosen following the terms from game theory. An example (in fact the only
example relevant for the implementation considered in this thesis) of a
cooperative action is responding to a query with the correct answer within the
time frame demanded by the system's \emph{rules} (the set of parameters).
Examples of a defecting action are not responding to a query, responding with an
answer that states that the query could not be resolved, or responding too early
or too late. See Section~\ref{sec:desc_rep_management} for details on how
reputation is tracked in query groups.

The reputation peers earn is used to determine the quality of service they
receive. Peers responding to queries are mandated to apply a \emph{penalty
delay} (not to be confused with a \emph{penalty}, described above)(TODO find a
better term?) appropriate to the reputation of the querying peer. This
potentially includes deliberately delaying the answer. The rules of the system
define a \emph{penalty threshold reputation} (or just \emph{penalty threshold}),
which is the amount of reputation a peer must be at or above in order to be
entitled to receive responses without penalty delay from query peers in the
corresponding query group. If the querying peer's reputation is below the
penalty threshold, the penalty delay is chosen proportional to the distance of
the reputation to the penalty threshold (up to a maximum, when the peer's
reputation is 0). Note that failing to apply the penalty delay is also a
defecting action on the part of the responding peer. Reputation is credit-only
(TODO ref credit-only vs. credit-debit in background), i.e. once a peer has
passed the penalty threshold he may send as many queries as he likes and is
entitled to delay-free service, he doesn't have to "pay" with having his
reputation reduced for a query.

Above the penalty threshold, \emph{reputation attenuation} makes it more
difficult for peers to gain further reputation. A distinction is made between
\emph{raw reputation} and \emph{effective reputation}. The former is affected by
rewards according to the rules, however, penalties are applied directly to the
latter. In order to gain an amount of effective reputation, it takes a modified,
usually greater, amount of raw reputation when attenuation is in effect. How the
raw reputation is attenuated is determined by the attenuation stretegy. See
Section~\ref{sec:attenuation} for a more detailed description and an examination
of different strategies.

During operation, pairs of peers exchange messages without other peers knowing
their content or even being aware communication is taking place at all.
Disagreements may occur between peers about whether the other's behavior was
within the rules of the system. For example, a querying peer may claim that the
responding peer took too long to respond and should receive a penalty, while the
responding peer insists he responded in time and should receive a reward.
Neither of the peers is able to prove his claim. A distributed \emph{complaint
system} needs to be in place that can make rulings in these situations. Such a
system is not part of the implementation in this thesis, and there may be very
complex dynamics arising from such a system that are not foreseeable without a
simulation. However, a proposal (TODO multiple proposals?) is presented in
Section~\ref{sec:desc_complaints}.

At many points, the system relies on messages being authenticated. It is assumed
that all messages exchanged between peers are signed to prevent one peer
impersonating another. Since peers only store one record in the \ac{DHT} with
their own ID as key, these records are also signed. Peer IDs are chosen to be
fingerprints of the peer's public key, so that there doesn't have to be a
mapping from peer ID to public key (TODO cite something about Tor hidden
services). The full public key can be distributed once a peer learns about
another peer for the first time (TODO this isn't entirely clear, it would have
to be the id->ip mapping records, but wouldn't that make them too big? we're
assuming "small" records).

The different message types and their possible layout is presented in chapter
(TODO ref performance chapter).

Bootstrapping the system is examined in
Section~\ref{sec:rep_avail_bootstrapping}, as is the ease with which new peers
can establishe themselves in an already running system. (TODO is this paragraph
necessary? ref bootstrapping section in this chapter as well)

\subsection{Sync Groups}
\label{sec:desc_sync_groups}
Every peer has a \emph{routing prefix}, which is a prefix of the peer's ID. The
length of the routing prefix is fixed in the rules of the system and the same
for all peers. (TODO ref a further work section and explain p-grid's approach of
making peer ID and routing prefix separate and allowing to advertise multiple
routing prefixes, even ones of different lengths) A peer is responsible for all
records whose ID starts with his routing prefix, in short, he is responsible for
the routing prefix. He maintains contact with other peers who have the same
routing prefix; these peers make up his sync group and are called his \emph{sync
peers}.

Sync peers keep each other apprised of changes to their record\footnote{As a
reminder, in this simplified version, peers store exactly one record in the
\ac{DHT}, where the key is their own ID.}. When it needs to be updated, they
broadcast the new record to all their sync peers, who then update their local
copy\footnote{Instead of broadcasting, gossipping could possibly be used (TODO
ref background chapter). However, it's difficult to incentivize peers to
participate in it (TODO ref rationale chapter).}.

Since every peer stores exactly one record in the \ac{DHT} under his own ID,
creating a new record to be maintained in the sync group is equivalent to
joining the group. To do this, the peer needs to know at least one other peer
who is already a member of the sync group and gives him the contact information
of all other sync peers. Then he broadcasts his new record to all the members.

Conversely, deleting a record is equivalent to leaving the group. In this case,
the peer simply broadcasts a notification to his sync peers.

Sync groups are relevant to replication within the \ac{DHT}. Every peer's record
is replicated as many times as there are peers in the sync group. The mean
replication factor in the \ac{DHT} is $\frac{n}{2^{l_{prefix}}}$, where $n$ is
the total number of peers participating in the \ac{DHT}, and $l_{prefix}$ is the
length of the routing prefix in bits.

No fixed replication factor can be guaranteed in this way, since the routing
prefixes are determined by the peer IDs, which peers choose themselves. It may
happen that one routing prefix is chosen by many fewer peers than another,
giving their records much lower replication. Advanced techniques exist to
address this problem (TODO ref background/p-grid multiple routing prefixes, load
balancing, or the rationale section here that should also discuss this), but are
considered out of scope for this thesis.

\subsection{Querying}
\label{sec:desc_querying}
- TODO make clear that this is about peer selection, link peer selection section
If a peer wants to get hold of a record (the key of which is called the
\emph{target ID}), he must send a query to another peer, the recipient. He
queries a peer known to him who is \emph{closer to} the target ID. This is the
case if the overlap of the recipient's routing prefix with the target ID is
greater than the overlap of the peer's own routing prefix with the target ID.
The overlap $o(a, b)$ of a bit string $a_i, i \in \{1, \ldots, n\}$ of length
$n$ and a bit string $b_j, j \in \{1, \ldots, m\}$ of length $m$ is defined as
\[o(a, b) = \max_{k \in \{0, \ldots, \min(n, m)\}} a_i = b_i \forall i \in \{1,
\ldots, k\}.\]

Note that, in case the peer's routing prefix overlaps entirely with the target
ID, i.e. it is also a prefix of the target ID, this implies the peer himself is
responsible for the record and should know it and not need to query anyone.

In case the recipient doesn't know the record himself, further querying is
necessary. There are multiple possibilities for handling this. The simplest of
them, and the one implemented in the simulation, is a recursive query, analogous
to recursive DNS queries, in which the recipient of the initial query does all
the work. He selects a peer closer to the target ID and sends him a query. This
peer again may need to query another peer. But this can only be the case a
finite number of times, since the next peer queried must always be closer to the
target ID, so eventually one is reached who is responsible for it. Once a
response arrives at one of the peers in this chain of queries, it is used to
respond to the peer who queried him. There are alternatives to this procedure
discussed in Section~\ref{sec:desc_recursive_vs_iterative}.

In this chain of queries, one of them may fail, e.g. because the recipient has
only recently joined the system and isn't yet sure he has all the records he is
responsible for. He may also just ignore the query out of laziness, in which
case the querying peer should, after a timeout, consider it failed. He should
then retry the query with a different suitable peer. If he has exhausted all the
possibilities, he should send a \emph{fail response}, indicating he is unable to
resolve the query.

Peers may elect to send multiple queries at once instead of waiting for one to
fail first. This reduces the risk of letting the querying peer wait
unnecessarily, especially if a timeout has to be waited out.

There may not exist a record under the target ID. In that case, the recipient of
the query should respond with a message that indicates this. However, a peer who
isn't responsible for the routing prefix of the target ID isn't guaranteed to be
aware of changes to records with the routing prefix, e.g., a new one may
recently have been added. Only peers in the sync group responsible for that
routing prefix are synchronizing information about relevant records. The sync
group therefore is an \emph{authority} for the routing prefix, and so are its
sync peers. The responding peer must include a signature from an authoritative
sync peer in the message that indicates no record exists for the target ID.

Instead of querying for a complete ID, the target ID can be a prefix of an ID of
arbitrary length (it can even be empty). A correct response to such a query is
any record whose ID starts with the target ID. This is useful for learning about
peers to complete one's subprefix coverage (see
Section~\ref{sec:desc_query_groups}). A response stating that no record with the
target ID exists in this context means that no record exists in the entire
\ac{DHT} whose ID starts with the target ID.

Additionally, when querying for a prefix, a querying peer may specify a number
of IDs or prefixes of IDs the record for which should not be the response. This
is useful for peers looking to complete their subprefix coverage: They can list
the IDs of peers they already know in order to learn about peers they don't yet.
If they then receive a response stating that no record with the target ID
exists, it means that they already know all peers matching the target ID.

- TODO make a graphic, or use the one from p-grid

\subsection{Query Groups}
\label{sec:desc_query_groups}
Query groups are a second kind of group employed by the system. Every peer may
be in multiple query groups, other members are called his \emph{query peers}.
Query peers are the pool of potential recipients of queries peers need to send.

Query groups are the unit within which reputation is tracked. Every peer has a
real-valued, non-negative reputation value for every query group he is in.
Reputation is synchronized within a query group, so that every peer is aware of
each of his query peer's reputations. In order for this to remain scalable,
query groups should be kept reasonably small.
Section~\ref{sec:desc_rep_management} contains details on how the reputation
record is kept up to date.

\subsubsection{Rewards, Penalties, and Penalty Delays}
The reason peers should only send queries to query peers is that their
reputation determines the quality of service they're entitled to receive. When
receiving a query, peers are required to consider the reputation of the querying
peer and apply a penalty delay, if applicable, i.e. to purposefully delay the
response. Failing to do so is itself a defecting action and incurs a penalty.
The appropriate penalty delay is $\max(0, p_{max} - r)$, where $p_{max}$ is the
configurable maximum penalty delay and $r$ is the peer's reputation.

$p_{max}$ is part of the rules of the system, and thus known to everyone, and
$r$ is known to all query peers. This means the sender of the query knows what
penalty delay is appropriate for his query. If a response has not arrived $d +
l$ after the query has been sent, where $l$ is another parameter in the rules of
the system to account for network latency, the response is late and the querying
peer must apply a penalty for a timeout. If the response arrives after less than
$d$ time has passed, the response is early and the querying peer must apply a
penalty for an early query.

Otherwise, the response has arrived within the correct time frame. In that case,
the querying peer must either apply a penalty for a failed response if the
response indicates that the recipient is unable to answer the query, or apply a
reward for a successful query if the response appears to contain the correct
information.

Furthermore, the querying peer may, through one or more additional queries, find
that the response was incorrect. This can be either because it was outdated, or
because it falsely claimed that no record existed for the target ID, when in
fact it did. In that case, the querying peer must apply a penalty for an
incorrect response.\footnote{To detect whether a record is outdated, they
contain a timestamp giving the last time they were changed. There should be some
leeway given, though. No penalty should be applied if the record has only just
been updated.}

\subsubsection{Subprefix Coverage}
Section~\ref{sec:desc_querying} described the process of sending a query and
stated that a peer wishing to send a query must send it to a recipient who is
closer to the target ID of the query. This requires the peer to know such a
recipient for every possible target ID. This is the case if the peer has
complete \emph{subprefix coverage}.

A bit string is a peer's \emph{subprefix} iff it is no longer than the peer's
routing prefix, all of it but the final bit is a prefix of the peer's routing
prefix, but the whole bit string is not a prefix of the routing prefix.

Formally, a bit string $s_i, i \in \{1, \ldots, n\}$ of length $n$ is a
subprefix of a peer with routing prefix $p_j, j \in \{1, \ldots, m\}$ of length
$m$ iff $n \leq m \land s_i = p_i, i \in \{1, \ldots, n - 1\} \land s_n \neq
p_n$. In other words, $s$ is a subprefix of $p$ iff $s$ is no longer than $p$
and the bitwise XOR of $s$ and an appropriately truncated $p$ takes the form
$00\ldots001$. Each peer therefore has $m$ different subprefixes, one each of
every length in $\{1, \ldots, m\}$.

A peer has complete subprefix coverage iff for each of his subprefixes he knows
at least one query peer for whose ID that subprefix is a prefix. No matter what
target ID he wishes to query for, he always knows a query peer closer to it: He
simply selects the query peer corresponding to the subprefix that has the
largest overlap with the target ID. That peer is at least one bit closer, namely
the last bit of the subprefix that is different for the peer sending the query.
In fact, there is a 50\% chance that the query peer's routing prefix overlaps by
at least 2 bits (in general, a $\frac{1}{n}$ chance it overlaps by at least $n$
bits), just by luck. This means steps in the routing are "skipped", an
authoritative peer for the target ID is reached sooner (TODO confirm this is
what kademlia builds its routing tables on (have more peers further away to
increase that chance) and mention it).

- TODO make or steal a tree graphic visualizing this

\subsection{Query Group Creation and Maintenance}
- TODO mention here that this has not been implemented, is only a proposal with
  potentially unforeseeable dynamics, and considered out of scope?

Query groups need to be created and disbanded, and peers need to be able to join
and leave them. There is no mechanism actually implemented for this in the
simulation included in this thesis. This is an outline of a protocol, but there
are likely to be complex dynamics involved causing unforeseeable problems.

\subsubsection{Leaving and Disbanding}
Leaving a query group can be achieved by simply broadcasting a message
indicating this to all members (TODO similar concurrency problem as the one
described in rep management: if a new member joins and gets the old state of the
group where the one leaving was still included, requires special handling, or a
confirmation that the current state is correct).

A query group is considered disbanded if it only contains one peer, i.e. once
the second-to-last member leaves.

\subsubsection{Creation}
A simple way to create a query group is for one peer to ask another to start
one, thus creating a query group containing 2 peers. This group would then grow
by new peers joining it. However, this assumes that it is reasonably easy for
one peer to find another with whom he is interested to share a query group and
who shares that interest. This may not be the case; a peer can send queries for
prefixes he is interested in, but the peers in the responses may not be
interested in him.

Peers are interested in others so that they can extend their subprefix coverage,
or get some redundancy into their coverage. For any pair of peers A and B, they
are either responsible for the same routing prefix, or A is responsible for
exactly one of B's subprefixes, namely the one of length $i$, where $i$ is the
one-based index of the first 1 in the bitwise XOR of the peer's respective
routing prefixes. So every peer can serve some subprefix for any other peer, and
the subprefix relationship between two peers is even symmetric, i.e. the
subprefix A can serve for B has the same length as the one B can serve for A
(they differ in the final bit). But that doesn't necessitate B being interested
in sharing a query group with A: he may already know enough peers for that
subprefix, or just not trust A for some reason.

A more complex way of creating a query group could involve the initiating peer
considering which subprefixes other peers are even interested in. This could be
announced in a peer's record that's stored in the DHT. The initiating peer would
check that the other peer is interested in extending his coverage to a subprefix
he is responsible for.

A query group could also be initiated with more than just 2 peers. The
initiating peer would contact all peers he wants to invite and propose to start
a group, attaching a list of members. If all agree, the initiating peer
broadcasts reachability information of all members. This allows for advanced
strategies in finding peers, where peers aren't pairwise interested in each
other, but circularly. E.g. A is interested in B, B in C, and C again in A.
Having more peers can also be helpful since it means more traffic (in small
groups, it can be hard to gain reputation for lack of queries).

\subsubsection{Usefulness}
\label{sec:desc_usefulness}
In this context it is convenient to introduce the notion of \emph{usefulness}:
it is the value peer A places on knowing peer B, with regards to his subprefix
coverage. It is a policy decision, not something the rules of the system can
dictate, and may depend on such things as the number of peers peer A already
knows for the subprefix, or whether he trusts peer B (maybe he feels like he is
being followed).

These factors are not visible to peer B, the length of the subprefix he can
serve is, though. All else being equal, peers serving longer prefixes are more
useful than those serving shorter ones. That's because peers for shorter
subprefixes are easier to find: consider a subprefix of length 1, which can be
served by all peers whose routing prefix begins with that single bit. Assuming
evenly distributed IDs, that's half the population. Longer subprefixes are also
closer to a peer's own routing prefix, and the peer is likely to get more
queries for it, making more redundancy a handy thing to have.

The length of subprefix a peer can serve for another peer could be a heuristic
for the peer to estimate how useful he is to the other peer. And, more
crucially, how useful he would be in a query group, by calculating the sum of
the lengths of the subprefixes he can serve for each peer (or e.g. the sum of
squares, to give longer subprefixes a higher weight). This can serve as an
estimator for how many queries he is likely to receive in that group, and thus
how easily he can gain reputation (see chapter~\ref{chap:rep_avail},
particularly Section~\ref{sec:rep_avail_group_reeval}) (TODO is that section
still in?).

\subsubsection{Inviting and Joining}
Joining a query group must be done via an \emph{entry peer}, a current member of
the group who handles the joining process. The membership candidate may ask the
entry peer to join, or receive a proactive invitation from him. Since a new
member affects everyone in the group, the entry peer should not be allowed to
make the decision by himself.

The simplest way is a majority vote: The entry peer presents the candidate to
the current members, and if more than half of them agree that he may join, he is
accepted. A modification of this is to require a unanimous vote. The members may
use their own judgement in this decision, e.g. considering how useful the
candidate would be to them, what impact he might have on the total traffic in
the group (especially with regards to reputation updates), or whether they
distrust him from previous experience.

If a candidate is accepted that not all current members agree with, these
members may leave the group, thus the candidate threatens the integrity of the
group. Therefore, it may be helpful to hold the vote in a way that asks the
current members whether they would leave the group if the candidate was added.
Again, this vote is up to peer policy. They may consider the peer untrustworthy,
not useful to them, or simply deem the group to big after his addition.
This would have to be done in multiple rounds: Peers announcing they'd leave in
the first round may prompt additional peers to want to leave, who were there
for connection to those peers. This is repeated until no additional peers
announce their intention to leave.

After all peer intentions have been collected, the decision to accept or reject
the candidate has to be made. It could either be made by the entry peer alone,
assuming that everyone who isn't going to leave is alright with the new peer
joining, or again by a vote (majority or unanimous) of the members who would
remain after the cadidate joins.

This approach assumes peers always have the option to leave a query group. But
some may depend on the group because their subprefix coverage would become
incomplete if they left. These peers would be pressured to stay and have no real
say in accepting or refusing the candidate.

In a possible extension, those peers that leave the group after that candidate
joins may automatically form a new query group. Candidates looking to join
groups then act as a sort of catalyst to split groups into smaller parts.

The process of deciding on a candidate's membership is likely to entail complex
dynamics. Both the simple voting and the other approach with peers announcing
they'd leave is highly dependent on peer decisions, and neither is clear to
yield a system that allows a good distribution of peers in query groups of
useful size (i.e. not just 2 peers per group).

- TODO find and cite consensus algos?

\subsubsection{Finding Groups}
In order to request to join a query group, a peer first needs to find one he
would like to join. After all, there is no public record of all available query
groups.

To do this, the query mechanism could be extended to allow queries not just for
peers (i.e. their record in the \ac{DHT}), but for query groups. The peer
looking for a group would send a query to some of his current query peers
stating he is looking for a group. In that query, he lists one or more of his
subprefixes for which there should be responsible peers in the group. If the
recipient of the query is in a group matching these criteria, he may offer to
act as the entry peer and receive a reward in the current query group for a
successfully answered query.

In addition to listing subprefixes that should be covered, the peer looking for
a group could request that his usefulness in the prospective group be calculated
or estimated, and only groups offered where it exceeds a threshold. This may be
necessary to ensure the new peer even gets any queries and can gain reputation
in order to make use of his new group.

A peer completely new to the system needs some way of entering his first query
group. He needs to know at least one peer already in the system to get started
(see Section~\ref{sec:desc_bootstrapping}), who must act as entry peer to at
least one query group.

Again, it is not at all clear that this proposal would work in practice. If this
process is used alone, it may well happen that the network becomes fragmented,
with some parts of it being entirely unaware of others, and thus unable to
access records stored there. There is also no solution for the problem that a
peer new to the system is rejected in all query groups he is offered in the
beginning.

- TODO concurrency issues all over the place

\subsection{Reputation Management}
\label{sec:desc_rep_management}
Penalty delays can only be correctly applied if every peer in a query group is
aware of all of his query peer's reputations. To achieve this, every peer stores
all relevant reputations locally, and receives updates for them from query peers
in the same query group.

The reputation of a query peer is initialized to 0 upon his joining the group.
It is subsequently modified through \emph{reputation updates} shared in the
group. A reputation update can either contain a reward, increasing the
reputation of the peer whom it is regarding, or a penalty, decreasing it.

Rewards and penalties are awarded in response to an interaction with another
peer or peers. In case the two peers share more than one query group, they are
applied in all of those groups (letting the peer choose in which group he would
like to receive the reputation update is open to abuse: he could choose to
receive a penalty in a group he doesn't really care about).

Rewards and penalties are handled in different ways for incentive reasons
(Section~\ref{sec:desc_rationale} goes into more detail on peer incentives).
After a peer performs a cooperative action (one deserving a reward, e.g.
successfully responding to a query) on another peer, the other peer sends him a
\emph{cooperation confirmation}, which is a statement signed by the other peer
confirming that the peer performed the action. The peer then broadcasts this
statement to all peers in the query group, who then increase the reputation they
have stored for the peer by the amount the rules of the system dictate should be
awarded for the action.

After a peer performs a defecting action (one deserving a penalty, e.g. failing
to respond to a query in time) on another peer, it is the other peer that
broadcasts the message stating what the first peer did. The query peers again
update their reputation record. For penalties, the peer who was wronged must
broadcast the message, since the one about to receive a penalty clearly has no
incentive to do it himself. Strictly speaking, the other peer doesn't either,
since there is no immediate benefit for him doing so (peers are selfish, not
vengeful). So broadcasting the reputation update must also give the broadcaster
a (small) reward. (TODO this is not done in the sim. mention it here? the
broadcaster may be saturated and just not spread the update. this may lead to
problems in the system, and hasn't been tested.)

The reputations stored at the members of a query group of course can't be
perfectly synchronized due to network delays. So whenever a peer has to make a
decision that depends on another peer's or his own reputation, he must give some
leeway to account for this. He should accept any action of another peer's as
appropriate that is within the rules according to any of the reputation values
up until a small timeout, dictated by the rules of the system, in the recent
past.

- TODO not done in the sim, but works anyway. mention this here?

- TODO make a timing graphic illustrating this?

Instead of broadcasting the reputation updates, it may be possible to use
gossipping to spread the traffic load. However, all the peers sending messages
then have to have incentive to do so. If they are given rewards, depending on
the gossipping protocol, peers low on reputation may opt to send many messages
to gain lots of reputation, leading to redundant traffic. On the other hand, if
lots of peers already have enough reputation, they may opt not to send anything
and gossipping fails. Therefore, this is considered out of scope.

Reputation updates must include a timestamp at which they should be applied in
order to correct for different arrival orders. Otherwise, the reputation record
may diverge at different peers if they receive 2 updates in a different order,
where one of them set the reputation to 0. Consider a peer with 1 reputation and
2 updates arrive: one subtracting 2, the next adding 1. Subtracting first yields
1 reputation, adding first yields 0. In case 2 updates specify the same time,
use the uint value of the ID of the peer who sent the update (this implies that
peers must not send two reputation updates with the same timestamp).

The complete reputation record must be made available to a new peer joining a
query group, it is simply copied from the entry peer. But this opens up the
possibility for a race condition if the new peer joins right after a current
member of the group sends a reputation update. The new peer will not be sent the
update, but the update may not have arrived in time at the entry peer for him to
give a current copy to the new member.

To address this, every peer must include a sequence number with every reputation
update he sends within a query group. This number is incremented by 1 with every
update he sends. Every peer maintains the last sequence number received by every
peer in the group. When joining a query group, the new peer receives a copy of
the current group state from one of the peers, which includes the sequence
numbers. Then, after he is established in the group, the new peer can check with
another peer whether he is up to date by giving his current state of sequence
numbers, potentially receiving a correction. To incentivize another peer's
cooperation in this, a reward is applied, but only once per new peer.

\subsection{Bootstrapping}
\label{sec:desc_bootstrapping}
- TODO necessary here, or just link rep avail chapter?
- routing prefix length must be variable, must start at 0 (see limitations on
  empty sync groups)
- need to know at least one other peer
- does that other peer have to be generous, giving plenty of his knowledge
  without delay?
- how to find first query group (see query group management section)

\subsection{Complaint System}
\label{sec:desc_complaints}
In certain situations, particularly regarding rewards or penalties being applied
or not, a peer may treat one of his query peers unfairly, but deny it. Usually,
no one but the two peers involved know what actually happened, and there is no
way for the wronged peer to prove it. The two most important examples are:
\begin{itemize}
\item Peer A queries peer B, B responds successfully. But then A fails to send
      the cooperation confirmation that allows B to claim a reward.
\item Peer A sends a reputation update containing a penalty for a peer B (thus
      receiving a small reward himself), but B didn't actually do anything
      wrong.
\end{itemize}

There may be multiple reasons for a peer to misbehave in such a way. In the
first example, it may be laziness (sending the confirmationyields no benefit to
the peer). The second example may be due to greed, since the peer receives a
small reward for broadcasting a reputation update. But it needn't necessarily be
selfishness, it could also be due to a tremble: Some part of one peer's network
stack at the lower layers may have failed, or a message may have been lost or
delayed in the network. Then the protocol implementation at both peers may
rightfully assume that, given the information available to it, it is in the
right.

In both examples, only the two involved peers know what happened, and neither
can prove it to anyone else: In the first case, B has no way of proving that A
didn't send the cooperation confirmation, not even that A sent a query in the
first place, much less that he answered it successfully (and in time). In the
second case, peer B can't prove he didn't do anything wrong. Peer A may for
example have queried him with no reaction from B, and again B can't prove that
he didn't receive a query.

To resolve situations like these, there must be a complaint system within each
query group, via which one peer can \emph{complain about} one of his query
peers. Such a complaint may be accepted, resulting in a penalty for the subject
of the complaint and a reward for the complaining peer (as compensation), or it
may be rejected. Of course, this system can't determine the truth either, but it
should not be continually exploitable, i.e. a peer should not be able to
continually misbehave without receiving a penalty through the complaint system,
and a peer should not be able to continually lodge false complaints against
another peer that are accepted.

- needs to ensure that it's a dominant strategy not to lodge false complaints,
  and to cooperate

As with query group creation, no complaint system has been implemented in the
simulation covered in later chapters. These are proposals (TODO plural?) for a
protocol that may work, but needs closer examination and possibly modification
due to the complex dynamics that will likely arise.

- TODO possibly leave this out, just state that tit-for-tat is a dominant strat?
  but that doesn't take into account trembles, one of those and a peer
  relationship breaks down
- w/o the system, peers could just retaliate: A applies penalty to B, B to A
  etc., the query group would break down (even though the peers playing
  tit-for-tat should be dominant)
- complaint system must offer a better alternative to retaliation
- maybe a complaint means the involved peers don't have to do anything for each
  other, can't get rewards or penalties from each other?

- majority vote, could be abused up to a point
- automatically believe an indictment if enough (n/2+1) indictments heap up.
  believe all following indictments as well?
- no chance if colluding peers
- at some point, it's best for a peer whose complaints keep getting rejected, to
  leave the group
- complex dynamics

\subsection{Load Balancing}
\label{sec:desc_load_balancing}
In the system as described, every peer is responsible for his routing prefix,
which is just a prefix of his own ID. Assuming IDs are evenly distributed (which
is the case if they are fingerprints of public keys and peers don't make
attempts to generate a particular key), this implies the expected value of the
number of peers responsible for a record is equal for all records. This is fine
under the assumption that all records are queried for with the same likelihood
(which is in fact what's happening in the simulation, where no load balancing is
necessary).

A more realistic record access pattern follows a power law distribution, with
very few records forming a large share of the total number of queries (TODO cite
something). Taking this into consideration, distributing the load of responding
to all the queries to the popular record becomes necessary, or else those peers
who happen to have a routing prefix matching one of those popular records
receive an unfairly disproportionate amount of queries. They may be inclined to
generate a new ID that doesn't suffer from this.

One of the central tenets of P-Grid involves separating a peer's routing prefix
from the records he is responsible for. Instead of a peer's ID determining which
records he has to store, each peer explicitly advertises one or more prefixes
which he is responsible for (this can become part of their record stored in the
\ac{DHT}). These new prefixes don't even all have to be the same length,
allowing for finer grained control. Being responsible for more prefixes means
peers have to join more sync groups and completing their subprefix coverave may
require more connections.

With this system in place, it becomes possible for more peers to take over
responsibility for prefixes containing popular records, thus balancing the load.
In P-Grid, this is done because peers are well-intentioned and willing to
cooperate (TODO cite). Under the assumptions made by the system in this thesis,
peers need an incentive to do so. They may have such an incentive because they
are unable to gain reputation quickly enough because they aren't getting enough
queries. Taking on responsibility for a popular prefix can address this.

A new kind of query could help peers decide which new prefixes to cover, in
which they query for a prefix which receives a lot of traffic and the peers
responsible for it would like the load lightened. Ideally, this system would be
self-organizing, like P-Grid is, but via economical incentives. But this has not
been implemented for this thesis, and doing so is likely to bring to light
complex dynamics not foreseen here.

\section{Rationale}
\label{sec:desc_rationale}
- TODO a lot of this is repetition from various previous parts. that ok?

The goal of the system is to extend a \ac{DHT} storing reachability information
so that users have a wide variety of peers to choose from for privacy reasons
(TODO ref introduction/motivation). This should be achieved under the assumption
that on the one hand users want to use the system for this privacy benefit with
a good quality of service, but all participants are selfish and unwilling to
expend resources that they don't have to. Records in the \ac{DHT} are fairly
small, but queries should return quickly.

The approach taken by the system is to exploit the users' desire to use the
system in order to get them to become peers contributing to the \ac{DHT},
thereby cultivating the wide variety of peers to choose from. To achieve this,
every action the system requires the peers to do must have an incentive, i.e.
not carrying out the action must carry the threat of being less likely to use
the system with a good quality of service.

It should be considered, though, that just because an action is rewarded, there
is no guarantee that it will be carried out. If a peer considers his current
amount of reputation sufficient, he may opt to not carry out such an action. In
that case, the peer is said to be \emph{reputation saturated}.

This notion of a hard cutoff is actually a simplified view on a selfish peer's
motivation. In a more realistic setting, whether a peer carries out an action
that is rewarded depends on the magnitude of the reward and the cost of the
action that he weighs against one another. Peers would have a valuation function
that assigns a value to each possible amount of reputation the peer could have,
as well as a cost function determining how costly the peer considers an action
to be. The peer would perform the action iff the value gained according to the
valuation function exceeds the cost according to the cost function.

In the simplification in the implementation, peers always aim to be cooperative
until their reputation reaches a threshold determined by the penalty threshold
and the reputation buffer described in (TODO ref implementation chapter). This
acts as a proxy for their weighing cost of action agains valuation of
reputation.

\subsection{The Reputation System}
\label{sec:desc_rep_system}
This is the purpose of the reputation system. Without reputation, peers only
receive a poor level of service, taking the form of artificial delays. Looking
up someone's reachability information is assumedly done right before initiating
communication with the other party. Artificial delays in this situation would be
annoying to the user and are therefore a good incentive for peers to gain
reputation.

Rewards and penalties can then be used to incentivize desirable and
disincentivize undesirable behavior. Rewards are applied for properly responding
to a query, i.e. within the right time frame, and with the correct record.
Penalties are applied for not responding at all, responding too early or too
late, or not being able to resolve the target ID to the correct record
(responding with a wrong or outdated record is discussed separately in TODO
ref).

Of course, late queries must be penalized in order to keep up the promise
of good service for high reputation. But, a little less intuitively, early
queries must also be penalized. Without any incentive to apply the correct
penalty delay, peers would be inclined to respond at the next convenient time.
In many cases, that might be right away, so that the peer doesn't have to keep
the state of the interaction stored for longer than necessary. Delaying the
response also requires the programmer of the implementation to do extra work.
The peer might conceivably also just respond whenever network resources are
free.

The problem with being indifferent to peers responding with no delay at all is
that it reduces or removes the threat of not participating. If there are
generous peers responding without delay independent of reputation, there is less
or no incentive to gain reputation. If the system's goal was availability alone,
this may be fine, but it is also supposed to offer peers a wide variety of peers
as query targets to choose from, so that profile building becomes more
challenging. Reducing the incentive to gain reputation, and thereby to respond
to queries, hinders this. The generous peers would attract a lot of queries and
have it easier to build profiles, counteracting one of the goals of the system.

\subsection{Query Groups}
Reputation gained by responding to one peer entitles a peer to better service
from other peers. This makes sense, since one of the goals of the system is to
allow peers to spread their queries to many peers. It has to be practical for
any pair of peers to have very few interactions but still use the reputation
system. Having every peer keep a local reputation for his communication partners
is therefore not an option, as many peers wouldn't be able to gain sufficient
reputation over these few interactions to be entitled to a reasonable quality of
service.

Tracking reputation globally doesn't scale to large networks. Peers can't
practically store every other peer's reputation locally and receive updates for
all of them. Moreover, spreading these updates throughout the network to so many
recipients would be challenging and require incentives. Storing reputation in a
decentralized structure like the \ac{DHT} the system is supporting, or a wholly
separate one, would add additional delays. Before responding, each peer would
have to do a \ac{DHT} lookup to even determine when to send the response. This
would have to happen at every hop in the query. Not to mention that this
potentially includes another \ac{DHT}, the maintenance of which requires
incentives itself.

The solution offered in this thesis are query groups, which are small enough
that each peer can store locally the reputations of the query peers, and updates
can be broadcast. At the same time, peers don't have to cultivate a trusting
relationship with a peer before they can query him, only with the query group as
a whole. Query groups become the unit in which peers build trust in each other's
willingness to cooperate.

This of course segments the network into many small groups, and the ideal goal
of being able to query any other peer is not reached. But peers wishing for more
privacy can join more query groups than is necessary to complete their subprefix
coverage, thereby increasing the pool of query peers. Of course, this comes at a
higher management cost.

Furthermore, query groups can be changed frequently, so that peers in them don't
become so familiar with one's querying habits that they can build a meaningful
profile. Before leaving an old group, it makes sense to join a replacement group
and gain reputation above the penalty threshold to ensure a smooth transition.
Doing so more frequently again adds cost at the benefit of higher privacy.

\subsection{Incentives for Repuation Management}
\label{sec:desc_incentives_rep_mgmt}
There must also be incentives for reputation updates. After receiving a
successful response to a query, the querying peer sends a cooperation
confirmation to the responder which entitles him to his reward. He then
broadcasts a reputation update benefitting himself. The incentive for this is
evident.

Broadcasting a penalty for another peer also has to give a small reward to the
one sending it, as otherwise there would be no incentive to do so. This is the
case even if the penalty is applied for defecting on the sender of the penalty,
like failing to respond in time, since we're only assuming peers are selfish,
not vengeful. They do not consider hurting someone who hurt them a benefit.

For this to work, however, the complaint system is crucial. If the querying peer
doesn't send the cooperation confirmation, and there is no recourse and no
threat of penalty for it, selfish peers will not do it. Similarly, it must not
be possible to apply penalties to other peers without cause, just for the
benefit of the reward broadcasting it brings.

The complaint system is sadly out of scope for this thesis, and likely to have
to deal with complex dynamics arising from the interaction of peers. If it
doesn't work, the entire system doesn't work.

Reputation should not be so plentiful that all peers in a query group can become
reputation saturated. If that should happen, no one has any incentive to respond
to queries, but no one has an incentive to broadcast penalties anymore, either.
Everyone would remain at their current, saturated reputation without any work
actually getting done. Section~\ref{sec:attenuation} describes how to adjust the
parameters of \emph{reputation attenuation} in order to control how often peers
become saturated.

A possible modification to defuse the situation should it happen anyway is to
let reputation decay at a slow rate. At a set interval, a fixed amount of
reputation is subtracted from each peer in every query group. This can be
implemented without actual reputation updates by viewing reputation relative to
the group's creation time: Every time a peer accesses a query peer's reputation,
he has to subtract a fraction of the group's age from the value stored in the
reputation record. This requires reasonably well-synchronized clocks.
\emph{Reputation decay} was implemented in the simulation as the first approach
to keep peers interested in participating, but the threat of penalty for failing
to respond turned out to be enough (TODO ref impl or eval).

Besides broadcasting reputation updates, peers receiving them also need to have
an incentive to update their local reputation record. After all, simply ignoring
an incoming message is easier than processing it. The incentive is indirect and
results from the fact that peers need to know the accurate reputation of their
query peers in order to apply the correct penalty delays and thus avoid
penalties.

\subsection{Incentives in Sync Groups}
\label{sec:desc_incentives_sync}
Incentives are also needed in sync groups, where peers stay up to date with the
records they're responsible for. Updates to records are broadcast. One
simplification the system assumes is that each peer stores exactly one record
with his own ID as record ID. This means each sync group consists of the owners
of the records the sync group is responsible for. The incentive for a peer to
broadcast updates to his own record is simply that other peers can reach him.

Of course, this only works as long as the owner of each record is in the sync
group responsible for the record. In a more general \ac{DHT} where peers store
records with arbitrary IDs, there needs to be some other mechanism. But that
case is out of scope for this thesis.

But there also needs to be an incentive to even listen to and apply updates
within the sync group; it would be easier to just dismiss them and respond with
the old record when queried, or even claim that, no matter the target ID, no
record exists for it.

The incentive here is indirect: If they respond with an outdated record or claim
no record exists even though it does, the querying peer may apply a penalty for
an incorrect response if he notices. This is not guaranteed to happen, since it
requires the querying peer to send at least one redundant query. It does not
require the complaint system, though: The querying peer has the (signed)
response containing incorrect information, and a response from another peer with
the correct information (records themselves are signed by the owner, so can't be
forged), and can therefore prove the defecting behavior.

The penalty is not guaranteed to be applied for another reason: Even if the
querying peer sends a second query, the other responder may also give an
incorrect response. In the extreme case, all members of a sync group respond "no
such record" to every query and no one can ever receive a penalty for an
incorrect query. This would be an equilibrium for this particular part of the
system, but an unfortunate one, since none of the records of the peers in the
sync group ever make their way to an outside peer.

But if even one peer properly applies the updates, there is a threat of
penalty\footnote{It actually takes at least two peers. With just one, that one
peer is safe to switch to responding "no such record" without fear of penalty.}:
Say there are $n$ peers in the sync group, $k$ of which respond correctly ($k <
n$), a chance $p_{retry}$ a querying peer sends a second query to a different
peer in the sync group, and a chance $p_{sat}$ the querying peer is reputation
saturated. Further assume he sends no more than two queries, knows all of the
sync peers and chooses the other peer randomly. Then there is a chance of
$p_{retry} \cdot (1 - p_{sat}) \cdot \frac{k}{n - 1}$ that the querying peer
applies a penalty. Depending on a peer's current reputation and the magnitude of
the penalty, this may seem an acceptable risk. The penalty should therefore be
set sufficiently high to ensure that it does so rarely. If that is the case,
peers need to apply updates in order to even be able to respond correctly when
they are low on reputation. If they have the current information anyway when a
query arrives, the cost of responding correctly (looking up the record in
memory) shouldn't be much higher than just responding "no such record".

A peer who is new to the system must join his sync group, i.e. learn about all
peers that share his routing prefix and tell them about him so that they can
send him updates. There is no apparent way to incentivize any sync peer to tell
him about all the other sync peers, though: once the first sync peer knows the
new peer, he will receive updates from him and thus there is no threat of
penalty anymore. But the new peer can use the ordinary query mechanism, querying
for his routing prefix while specifying the sync peers he already knows, until
he knows all of them. To shorten this process, there could be a special query
for all records matching a prefix, which also gives a multiple of the usual
query reward. A current sync peer would even be incentivized to act as entry
peer for the new peer into one of his query groups so that he can collect this
reward.

In case gossipping should be used for the updates to records within sync groups,
finding incentives becomes more difficult. Gossipping spreads the load of
disseminating the update to multiple peers, and removes the need for all sync
peers to know one another. One might imagine a separate reputation system for
sync groups, but that's out of scope for this thesis.

Incentives in sync groups are important for the system to function.
Unfortunately, they are not part of the simulation implemented with this thesis
(TODO ref impl chapter), so no investigation has been done into them.

\subsection{Relative Amounts of Rewards and Penalties}
While not tested in experiment, some general rules for the relative proportion
of different rewards and penalties can be derived by theoretical consideration.

The penalty for letting a query time out should be greater than that for a fail
response (one which states the target ID couldn't be resolved). The latter is
more useful to the querying peer, since it takes less time and allows him to
e.g. send a retry sooner.

The reward for broadcasting a penalty should probably be less than any penalty,
otherwise it becomes easy for colluding peers to "penalize" each other, thus
gaining arbitrary reputation. Technically, the complaint system is supposed to
catch this, and there are considerations on how to limit collusion (TODO ref
section), but it seems ill-advised to make this kind of behavior possible in the
first place.

The penalty for an incorrect response should be set sufficiently high to
disincentivize giving them even if the likelihood of actually receiving it is
low (see Section~\ref{sec:desc_incentives_sync}).

\subsection{Recursive vs. Iterative Queries}
\label{sec:desc_recursive_vs_iterative}
Section~\ref{sec:desc_querying} described recursive querying, in which the
recipient of a query is expected to respond with either a successful or a fail
response. If he isn't responsible for the target ID and thus doesn't have a copy
of the record, he has to query a suitable peer himself and wait for the response
in order to pass it on to the peer who queried him. This is the simplest way to
handle the situation, and the most convenient for the querying peer, since all
the work is done by the recipient.

\subsubsection{The Recursive Query Problem}
\label{sec:desc_rec_query_prob}
But recursive queries can lead to the \emph{recursive query problem}, which is a
peer's bad reputation in one query group having adverse effects on his
reputation in another. Say peer A receives a query from peer B, and they share
query group X. A needs to send the query on to one of his query peers in order
to answer it, and determines that peer C is the most suitable recipient, whom he
knows from query group Y. A is expected to respond to B within a time window
determined by B's reputation in group X. Similarly, C is expected to respond to
A within a time window determined by A's reputation in group Y. If minimum delay
of the latter time window is greater than the maximum delay of the former, A has
no chance of responding in time (that is assuming C observes the expected
delay) and will be penalized for it.

- TODO make a graphic for the recursive query problem

Under these circumstances, A's bad reputation in group Y leads to a penalty in
group X. This is potentially a serious hindrance to a peer's ability to gain
reputation. Under unfortunate circumstances, one could imagine low reputation
values in two query groups keeping each other down. This problem can also be
observed in the simulation results, e.g. in
figure~\ref{fig:selection_overlap_peer_reps} in
Section~\label{sec:rep_avail_selection_overlap} (TODO confirm it's the rec query
prob).

While the recursive query problem can impede peers' ability to gain reputation,
there is an argument that it can be a good thing once peers have sufficient
reputation. It forces them to maintain good subprefix coverage with a good
quality of service, even if they may not care about parts of the ID namespace
personally. But whether this effect is useful or even noticeable is beyond the
scope of the simulation and this thesis.

\subsubsection{Iterative Queries with Vouchers}
\label{sec:desc_iterative_vouchers}
Instead of recursive queries, iterative queries could be used. With this method,
instead of the recipient of a query handling all the work and returning the
final response, he is only expected to find a suitable next peer to query, i.e.
one who is closer to the target ID, and respond with this peer's reachability
information. The querying peer must then query that peer himself. This solves
the recursive query problem, because the recipient of a query only needs to do a
local lookup and is not dependent on another peer.

It introduces a new problem, though. There is no guarantee, and with growing
network size it becomes increasingly unlikely, that the querying peer shares a
query group with the peer the first recipient said to contact next. That peer
has no incentive to respond to a peer with whom he shares no query group.

A possible solution to this problem is for the responder to pass on his level of
service. Say peer A has queried peer B, who responds with reachability
information for peer C. The response now also contains a \emph{query voucher},
which is a statement signed by B that allows A to send a query to C with the
quality of service that B would have gotten. A can then query C, and C must
treat his as though he were B. This entitles B to a reward for a successful
response, even if the level of service ultimately received by A is bad (because
B has low reputation in the query group he shares with C). After all, B
demonstrated his best effort to answer the query.

Query vouchers must be single use to prevent peers from collecting vouchers for
all their subprefixes and then leeching off of another peer's good reputation.
First off all, this means they have to contain some unique ID. Secondly, it is
either required that the peer's with whom the voucher is redeemed keep a history
of all vouchers that have been redeemed, or the peer who issues the voucher must
authorize the particular voucher for one use.

They also need to have a (relatively short) expiry time to prevent peers from
stockpiling vouchers and then leeching off of the issuers good standing. This
again requires an expiry time on the voucher, as well as clocks within the
network to be reasonably in sync.

Query vouchers also require the complaint system to be extended. Using peers A,
B and C from the example, A needs to have a threat of penalty for not sending a
cooperation confirmation to C after a successfuly query. But C can only complain
about B, so there must be a new complaint type that allows B to complain about A
for having received a complaint after use of a voucher. Then C complains about
B, and B complains about A.

\subsubsection{Explicitly Advertised Quality of Service}
Another alternative becomes possible if the extension described in
Section~\ref{sec:desc_load_balancing} is implemented. In it, the records a peer
is responsible for is independent of the peer's ID. Peers are free to choose
which prefixes they are responsible for and advertise this to other peers.

These advertisements could specify the level of service peers are able to
deliver for each prefix, i.e. how long a delay they have to impart at a minimum.
Querying peers are then only allowed to penalize them if they fail to meet the
quality of service that they advertised.

Using this method increases the management overhead somewhat, as prefixes may
have to be split so that differing levels of service for different subprefixes
can be advertised. More importantly though, it may make it harder for peers with
low reputation in some query group to gain reputation. Other peers will be less
likely to query them if they are only promised a slow response. This is not part
of the simulation implemented for this thesis.

\subsection{Credit-only vs. Credit-Debit Reputation}
\label{sec:desc_credit_only_vs_credit_debit}
The system uses credit-only reputation, i.e. no reputation is deducted for using
the system (e.g. per query) (TODO ref background). Once peers have gathered at
least the penalty threshold reputation, they are entitled to delay-free service.
They only lose this privilege if they incur a penalty, most notably for failing
to respond properly to a query. As long as this doesn't happen, they can send as
many queries as they like. With this method, having reputation is seen as proof
that a peer is generally willing to contribute, to play a part of the whole. It
is unnecessary to deduct this reputation, as querying someone else doesn't
change that assessment.

In the alternative, credit-debit, each query a peer gets successfully answered
costs him a little reputation as a sort of payment. This could be incorporated
into the system quite easily, by incorporating this penalty into the reputation
update containing the reward for the responding peer. This method views
reputation more like currency, to trade work done for other peers against work
demanded of other peers.

Credit-only is the simpler of the two possibilities and works fine if we assume
peers have comparable query behavior, as the simulation does (TODO ref
implementation). But if there are some power users sending many more queries
than other peers, the relationship of work put in vs. work consumed becomes
unfairly skewed towards those peers not querying very much. After all, there are
more queries in total that need to be answered (not responding is not an option
since it will yield penalties), but they are not getting anything more out of
it.

Credit-debit could address this: Peers using the service more heavily have to
contribute more, in effect taking load off of moderate consumers. But heavy
users must also be given the opportunity to earn more reputation by increasing
their contribution. Load balancing, described in
Section~\ref{sec:desc_load_balancing} has to be used to address this. Heavy
users have to take responsibility for more records in order to receive more
queries, thus gaining the additional reputation they need in order to use the
service to the extent that they wish.

\section{Definitions}
- TODO where the hell do i put definitions? before or after the description?

- TODO some of these seem unnecessary

\begin{description}
\item[The system:] Short to reference the protocol, procedures and rules meant
to incentivize participation in a \ac{DHT} that is the topic of this thesis.
\item[Rules of the system:] The set of parameters of the protocol that are fixed
and known to everyone. Includes e.g. the magnitudes of rewards and penalties,
and the penalty threshold reputation.
\item[Record:] The data stored in a \ac{DHT} under one key (or \emph{record
ID}). Also contains a timestamp giving the time of last change.
\item[Peer:] One participant of the system, has one unique \emph{ID}; commonly
called \emph{node} in the context of \acp{DHT}.
\item[ID, Peer ID:] A bit string the length of which is fixed and prescribed by
the \emph{rules of the system}. Uniquely identifies a peer.
\item[Record ID:] A bit string of the same length as a \emph{peer ID}. Uniquely
identifies the record belonging to the peer with the same ID. Technically the
key of the record, but the word ID is used to stress that it's of the same kind
as a peer ID.
\item[Target ID:] A bit string that is the subject of a \emph{query}. Its length
is less than or equal to that of a \emph{peer ID}. If less, the query is asking
for any peer whose ID starts with the target ID.
\item[Matching an ID:] Two ID's match iff one is a prefix of the other. E.g., a
\emph{target ID} matches a \emph{peer ID} (or a \emph{record ID}) if the target
ID is a prefix of the peer ID, i.e., the peer ID is a valid response for a query
asking for the target ID.
\item[Overlap $o(a, b)$:] The length of the longest common prefix of two bit
strings. Defined in Section~\ref{sec:desc_query_groups} (TODO ref the
definition itself as an equation? otherwise make sure it's actually defined in
that section).
\item[Being closer to an ID:] ID A is closer to ID X than ID B iff the overlap
of A and X is greater than that of B and X.
\item[Reachability information:] The abstract notion of information stored in
records in the \ac{DHT} that is sufficient to communicate with a peer. More
concretely, an IP address.
\item[Routing prefix:] The prefix of a \emph{peer ID} of a length that is equal
for all peers and prescribed by the \emph{rules of the system}. Determines which
\emph{sync group} a peer is in.
\item[Being responsible for a routing prefix:] A peer is responsible for a
routing prefix iff the routing prefix is a prefix of his ID, i.e. he is in the
\emph{sync group} associated with that routing prefix.
\item[Sync group:] A group of peers who are \emph{responsible for} the same
routing prefix. Maintain contact in order to exchange updates regarding the
records they store.
\item[Sync peer:] Peers A and B are sync peers of one another iff they are in
the same \emph{sync gruop}.
\item[Authority (with regards to a sync group/peer):] Sync groups or peers are
an authority, or authoritative, for their routing prefix because they are
responsible for keeping current with updates to its records.
\item[Query group:] A group of peers that may send each other \emph{queries}.
Maintain contact in order to track each other's reputation within the group.
\item[Query peer:] Peers A and B are query peers iff they are in the same
\emph{query group}.
\item[Subprefix (TODO better term?):] A prefix of a \emph{routing prefix} where
the last bit is flipped. Each peer has a number of subprefixes equal to the
length of the routing prefix.
\item[Serving a subprefix:] Peer A serves subprefix P for peer B iff P is a
subprefix of B's and a prefix of A's routing prefix.
\item[Subprefix coverage:] The state of a peer's ability to forward queries,
which is dependent on having query peers serving subprefixes. The subprefix
coverage is complete iff, for each of his subprefixes, the peer has a query peer
serving it.
\item[Request (for a target ID):] External stimulus to the system prompting it
to resolve the target ID. Simulates interaction of e.g. a user with the protocol
stack. Not to be confused with a \emph{query}.
\item[Query (for a target ID):] Message sent from one peer (the querying peer)
to another (the recipient) requesting the \emph{record} for the target ID.
\item[Response, fail response:] Message sent from one peer (the responding peer)
to another to answer a \emph{query}. The responder may claim to have included
the correct record, making the response successful, or that he is unable to
resolve the correct record, making it a fail response.
\item[Usefulness:] A measure of how useful or important peer B is to peer A to
complete subprefix coverage. Based on the length of subprefix B can serve for A,
but can also incorporate the trust in A or how many other peers A knows also
serving that subprefix. Used by B to estimate how many queries he is likely to
get from A, or from an entire group of peers (see
Section~\ref{sec:desc_usefulness}).
\item[Reputation:] A non-negative, real number stored in every query group for
each of its members. Earned by peers for better quality of service.
\item[Reputation record:] The copy of all of a peer's query peers' reputation
stored locally.
\item[Reputation decay:] Process that subtracts reputation from every peer in a
query group at a steady rate.
\item[Reward, penalty:] A change in a peer's reputation respectively increasing
or decreasing it.
\item[Cooperative action:] An action performed by a peer that warrants a reward.
\item[Defecting action:] An action performed by a peer that warrants a penalty.
\item[Tremble:] A defecting action that was not intended by the peer, but e.g.
caused by a congested network. (TODO also cite the paper introducing the term)
\item[Cooperation confirmation:] A signed statement given to another peer
confirming that he performed a cooperative action.
\item[Reputation attenuation:] A reduction of the magnitude of rewards that
takes effect once a peer's reputation passes a minimum (see
Section~\ref{sec:attenuation}).
\item[Raw and effective reputation:] Reputation before and after
\emph{reputation attenuation} takes effect.
\item[Recursive query:] A query in which the recipient performs all further
querying on behalf of the sender and returns the final result.
\item[Iterative query:] A query in which the recipient only indicates the next
suitable peer to query, the sender must perform the query himself. Requires
\emph{query vouchers} to provide a good level of service (see
Section~\ref{sec:desc_iterative_vouchers}).
\item[Query voucher:] A one-time-use statement entitling one peer to the level
of service of another. Used for iterative queries.
\item[Hop:] One step in a query.
\item[Routing loop:] Failure in looking up a record in the \ac{DHT} caused by at
least one peer in the chain of queries forwarding the query to a peer who is not
\emph{closer to} the \emph{target ID}. The query is sent back to peers who
originally sent it. See also Section~\ref{sec:selection}.
\item[Entry peer:] A peer who is a member of a query group and facilitates the
joining of another peer into that group.
\item[Recursive query problem:] An issue in which a peer's low reputation in one
query group negatively affects his reputation in another (see
Section~\ref{sec:desc_rec_query_prob}).
\item[Penalty delay:] The delay that is deliberately imposed before responding
to a query to penalize the querying peer.
\item[Penalty threshold reputation (short: penalty threshold):] The minimum
reputation a peer must have in order for the penalty delay to be 0.
\item[Saturation reputation, being saturated:] The amount of reputation at which
a peer considers himself to have enough and isn't motivated to do work in order
to earn more.
\item[Broadcast:] The spreading of the same message to all peers in some group.
Explicitly \textbf{not} necessarily an actual broadcast on the network layer,
may also be executed as multiple unicasts. The term is used for simplicity of
discussion.
\end{description}

- TODO game theory stuff: utility, cooperating, defecting, social value (probably
  more)

\section{Limitations and Unresolved Issues}
There are some situations which the system can't handle.

\subsection{Network-Optimized Queries}
There are no provisions in the system to choose recipients in a way that is
optimized with regards to network topology or geography. As it stands, every hop
in a chain necessary to resolve one initial query may be transmitted to another
continent, making queries potentially run very long.

Even if considering their own interest, peers may not have an incentive to query
a peer close to them in the network if they have a bad reputation in the query
group they share with him.

If such an extension is desired, peers need a way to find new query peers
(serving a specific subprefix) that are close to them in the network.

Then again, being able to spread out ones queries to a large pool of peers is a
design goal of the system, it actually happening a consequence.

\subsection{Empty Sync Groups}
\label{sec:desc_empty_sync_groups}
There may be a routing prefix for which there is not peer, i.e. the
corresponding sync group is empty. In that case, there is no one who can
authoritatively answer queries for this prefix (stating that no peer exists).
Such queries are still legitimate, e.g. because there used to be a peer with the
routing prefix in question. But more importantly, they occur when the empty
routing prefix is the subprefix of some peer. That peer will send a prefix query
to complete his subprefix coverage but never receive an authoritative response.
Peers close to the empty sync group (sync peers of the peer, whose routing
prefix is equal to the empty one except in the final bit) could respond that
there probably doesn't exist a peer, but this isn't ideal.

The only workaround that can be offered is to choose the routing prefix length
in such a way that empty sync groups don't occur. It must be variable during
operation anyway, in order to support changes in network size. This can be
achieved e.g. through a system like the one described in
Section~\ref{sec:desc_load_balancing}, but maybe also a simpler one, in which
the routing prefix length is the same for all peers, but can be changed at
runtime.

Besides empty sync groups, groups with only very few peers cause problems as
well, as they make peers' ability to gain reputation brittle, as described in
Section (TODO ref).

\subsection{Going Offline}
The system so far doesn't consider that peers may want to go offline. If they
do, they are still expected to respond to queries. Consequently, their
reputation will crash once they do, and they have to work their way back up from
0 when they come back online.

What's missing to make going offline convenient is a way to mark oneself as
inactive, saving reputation until later. Such a mechanism must be proofed
against abuse, or peers would mark themselves inactive almost all the time. It
is also not ideal for peers whose connection may be disrupted unexpectedly, like
ones on mobile connections.

\subsection{Circular Queries}
There is nothing in the system in principle that prevents peers from querying a
peer who is not closer to the target ID than they are themselves. If they do,
circular queries can occur, in which peer A queries peer B, who queries peer C,
who queries peer A again. Without any way of detecting such loops, peer A may
even think that he can use the response to his initial query to answer the query
he received from C, and not query anyone else. Then all queries will time out
and everyone (including peer A) receive a penalty, even if all involved were
trying to cooperate.

There are no incentives specifically in place to get peers to only query peers
closer to the target ID. Usually of course, it is better for a peer to query the
closest peer since that reduces the number of hops required and thus the
response time. But if the peer has low reputation in all the query groups he
shares with peers closer to the target ID, it is the rational choice to instead
query one who is not closer from a group in which he has high reputation. The
absence of the penalty delay is likely to make up for the additional hops (that
is assuming there is no routing loop).

Forbidding such queries entirely isn't an option either, since they are
sometimes necessary. In particular for new peers looking to complete their
subprefix coverage. They have to query peers not closer to the subprefix they're
querying for, since they don't know anyone closer (that's why they're doing it,
after all). It might be an option to respond to such queries, but give a large
penalty for posing them. That wouldn't affect new peers, who have no reputation
yet anyway, while disincentivizing the behavior for established peers.

\subsection{Queries from Non-Query Peers}
There is also nothing in the system keeping peers from querying peers they don't
share query groups with. They aren't supposed to do this because it steps
outside the reputation system, which is essential to the system. The recipients
of the queries also don't have any incentive to even answer.

However, having the option of querying a non-query peer could be useful for new
peers trying to complete their subprefix coverage. Especially if they don't know
many peers to start with.

- TODO maybe just join query groups with them? is this even really a problem
  that belongs here?

\section{Possible Attacks}
This section describes a few vectors attacks on the system that have been
considered, but doesn't claim to be exhaustive. Some of them can be dealt with,
other less so with the system as it has been proposed. Two kinds of attack can
very broadly and clearly be distinguished:
\begin{itemize}
\item Trying to cheat the system into granting a better quality of service than
the peer is intended to receive based on his cooperation. This includes tricks
to avoid penalties after defecting behavior, and ones to receive rewards where
they're technically correctly applied, but not in the spirit of the system (TODO
do i even have examples of this?). The perpetrators are selfish peers, their
capabilities can be considered to be limited.
\item Attempting to nullify the privacy gains by tracking or building profiles
on either particular, multiple, or even all peers. The attackers are the
adversaries from the initial motivation for the system (TODO ref?): data
collectors ranging from private user behavior analysts to government agencies.
They should be considered high-tech threats.
\end{itemize}

In this thesis, the former are the primary focus. However, the latter should be
kept in mind, since they are ultimately the reason the system even exists.

\subsection{Free Riding and Whitewashing}
Free riding, or leeching, is peers using the service without contributing to it.
Preventing it is the center of the reputation system and works, under the
assumption that peers are interested in delay-free service. Contented peers (see
Section~\ref{sec:desc_contented_peers} don't meet that assumption, and could
become a problem, since the promise of a good level of service is the only
leverage the system has over peers.

Whitewashing is a peer leaving and then rejoining the system with a different ID
in order to shed previous bad reputation. This attack is viable only in systems
where new users are given some trust upon joining, which they can subsequently
lose. In the system proposed here, peers joining a query group start at the very
bottom, namely 0 reputation (and reputation is non-negative). Rejoining thus
doesn't afford them any benefit.

- TODO previous work has shown that it makes sense to trust new users to some
  degree, though
(TODO ref previous work, which should cite some stuff on this?)

\subsection{Collusion/Sybil Attacks}
\label{sec:desc_collusion_sybil_attacks}
The system, as it is, enables a very simple attack by a pair of colluding peers
(irrespective of whether they're actually separate peers or the same peer using
different IDs). The two peers join a query group together and give each other
rewards for successfully answered queries (they just need to pass each other
cooperation confirmations, no queries need to actually take place).

One quick fix to Sybil attacks is to require rewards to not just come from a
separate ID, but from a separate network address as well. However, this causes
problems for legitimately separate peers sharing a connection, and isn't
effective considering IPv6.

Possible solutions also targeting separate colluding peers are more complex. The
peers' ability to continually reward each other may be impeded by limiting the
volume of reward a peer can give any other given peer in a given time. The
reputation record would include who gave whom which rewards in the recent past
(this would require $\mathcal{O}(n^2 \cdot m)$ space in a query group with $n$
peers, each giving $m$ rewards to each other peer in this time frame). Instead
of completely disabling rewards, they could also just become smaller. After some
time of not applying rewards, peers would regain this ability.

With the time frame properly configured, this would already make it a good deal
harder for peers to collude. They can't simply ignore other peers' queries since
they don't have an unlimited supply of free reputation.

The colluders may then find more peers to collude with, to at least increase
that supply. But at some point, the colluders make up a large shared of the
peers in the query group. The other peers, whom the colluders have come to
exploit, are likely to leave if they continue to not have their queries
answered.

Implementing this solution would have side effects. The overhead of storing the
timeouts for every pair of peers has been mentioned already. Additionally, peers
receiving queries would lose their incentive to respond if they have received
the maximum number of rewards from the querying peer already. Everyone would
have to have more query peers so to be able to continue querying. This means
either larger query groups, or membership in more query groups than otherwise,
in either case a larger management overhead.

In a more extreme variant of the previous solution, the volume of reward any
peer can apply to each other peer could be restricted absolutely. Once it's been
used, that peer can't be rewarded anymore and thus has no incentive to respond.
If the relationship with that peer was important, the peer may consider the
query group to be "used up", and leave. Of course, there would have to be a way
to prevent a peer from leaving and immediately rejoining in order to reset the
counter.

This can even be combined with the first solution and should be even more
effective at combating collusion. It also brings the added side effect that all
peers are forced to switch query groups frequently.

Another approach could target colluding peers' ability to join the same query
group. Since there is no way of telling whether two peers are colluding, it
would have to be aimed at preventing pairs of peers from joining in close
temporal proximity. Whether a peer is accepted into the query group is a policy
decision by the group's members, so they have to consider it in their interest
to prevent collusion.

They could choose not to accept a peer into the group shortly after a new member
has been accepted, as would be the case with two colluding peers. With a more
complex policy, they could choose to only accept an applicant if the newest
member has already gained some reputation (and thus proven himself to be not
completely useless, as a peer waiting for his colluder would be). They could
also choose to refuse an applicant if the newest member of the group is acting
as entry peer, which is the easiest way for colluding peers to share a group.

None of these proposed solutions offer a way to detect or deter colluders with
certainty, but they may make it harder for them, possibly hard enough that it
isn't worth the effort. Whether this is the case is beyond the scope of this
thesis.

\subsection{Limiting Responsibility}
If sync group sizes are very disparate, a peer may choose one that is very small
(by generating new IDs until the routing prefix matches). Thus, he is
responsible for only a small amount of records. While this makes it harder to
gain reputation initially, once he has done so, maintaining it requires less
effort.

If all peers are willing to generate IDs in order to join the smallest sync
group, this problem really fixes itself, performing load balancing. Otherwise, a
credit-debit system (see Section~\ref{sec:desc_credit_only_vs_credit_debit}) or
reputation decay (see Section~\ref{sec:desc_incentives_rep_mgmt}), together with
explicitly advertised prefixes (see Section~\ref{sec:desc_load_balancing}) may
help.

\subsection{Abusing the Complaint System}
- TODO selfishness: abusing the complaint system (TODO)

\subsection{Generous Peers}
\label{sec:desc_attacks_generous}
Generous peers have been described in Section~\ref{sec:desc_generous_peers}.
The motivation for their generosity may not be selfless. Firstly, responding
without delay regardless of the querying peer's reputation attracts traffic from
peers with low reputation. This counteracts one of the motivations for creating
the system, as the generous peer is better able to build a user profile of these
low-reputation peers sending him queries.

Secondly, it lowers the usefulness of having good reputation and thus lowers
peers' incentive to even respond to queries. While regarding the first point one
could argue that it's up to the peers to not query the same peer repeatedly,
this consequence makes just that harder. With all of those peers who don't care
very much about privacy using the generous peer's service but not contributing
themselves, those peers who do have fewer options of peers to query.

Placing a large number of generous peers may act as part of a strategy by a
high-tech attacker aimed at attracting enough traffic to track users again.

\subsection{Stalking}
\label{sec:desc_stalking}
The motivation for using the system is that it makes tracking users and profile
building harder. Attackers may try to do it anyway by following peers. In order
to learn anything about a target peer's query behavior, the attacker must share
at least one, and should probably aim to share all, of the target peer's query
groups.

The attacker has to find and join query groups the target peer is in. He can do
this by querying for the target peer's routing prefix, pretending that he is
trying to complete his subprefix coverage. If he wants to avoid arousing
suspicion, he needs to have the right routing prefix, one which has the target
peer's routing prefix as a subprefix (i.e. it is that subprefix with the final
bit flipped). Assuming the routing prefix has a length of $p$, he can expect to
have to generate $2^{p-1}$ random IDs to get one that matches. This process
should eventually find at least some of the target's query groups.

A sufficiently paranoid peer should keep a list of query peers he has interacted
with recently, so to avoid them for a while and make it hard for anyone to track
him. So the attacker needs to generate a new ID for every query group the target
joins, and likely a new network address as well (these are known within a query
group and can also be temporarily avoided by the target).

\subsubsection{Passive Observation}
Once the attacker shares a query group with the target, his simplest option is
to passively listen. He knows when the target sends a query by observing the
reward the target applies to the query's recipient, or the complaint of the
recipient about the missing cooperation confirmation. By observing rewards
applied to the target in other query groups, he can infer whether that query was
for the target himself, or done on behalf of another peer. He doesn't know
exactly what the target queried for, but it's probably something the recipient
of the query is closer to.

\subsubsection{Actively Attracting Queries}
The attacker can also be more active and try to attract queries from the target.
To make this more likely, he must be useful to him, meaning he needs more
observation peers with different IDs (serving the target's subprefixes) in the
query group. The attacker's first peer can act as entry peer for them. If the
query peer selection strategy \emph{overlap $\rightarrow$ reputation sorted}
(see Section~\ref{sec:rep_avail_selection_overlap_rep_sorted}) is mandated, the
attacker should aim for the observation peers to always be the lowest reputation
peer for the given subprefix to make it likely to receive queries from the
target\footnote{It's still not guaranteed since there may be another peer with a
longer overlap. The attacker may choose to use even more observation peers, in
the extreme one mirroring the routing prefix for each existing peer in the
group. This would likely be suspicious to a paranoid peer, though.}. For all
queries sent to one of the observation peers, the attacker knows exactly what
the target is querying for.

Getting all of the observation peers into the query group can be difficult
already if the collusion countermeasures described in
Section~\ref{sec:desc_collusion_sybil_attacks} are implemented which make it
harder to get multiple peers into a group at once. In that case, a paranoid
target could make the attacker's life harder by switching query groups quickly
so the attacker never gets a good foothold in any of them.

The target may also elect to never query a peer who joined the group after him.
This would disable that attack vector unless the attacker joins query groups
preemptively.

In another possibile attack not vulnerable to that countermeasure, the attacker
could attract queries from the target's query peers instead of from the target
himself. Any query received from one of them that is followed by either a reward
from the target, or a complaint about the target because of a missing
cooperation confirmation, was likely done on behalf of the target. Thus the
attacker knows what the target queried for. This attack requires joining many
more query groups though, as each of the target's query peers will be in groups
he doesn't share with the target.

\subsubsection{Total Surveillance}
The most extreme attack is to strive to have a presence in every query group.
The attacker can then observe query chains through different query groups down
to sync group precision (i.e. know the routing prefix of the target ID of any
given query) by correlating query rewards or complaints about missing
cooperation confirmations.

\subsubsection{Cost}
The cost for these attacks mostly comes down to generating the required IDs and
possibly acquiring the needed network addresses. In the following, $p$ denotes
the (fixed) length of the routing prefix, $m$ the number of members in each
query group, $g$ the average number of query groups each peer is in, and $n$ the
total number of query groups in the system.

To join all of the target's query groups, the attacker must generate
$\mathcal{O}(g)$ specific IDs.

To attract queries from the target directly, with one observation peer per
subprefix, the attacker must generate $\mathcal{O}(g \cdot p)$ specific IDs.

To do this indirectly, attracting queries from the target's query peers, with
one observation peer per subprefix, the attacker must generate $\mathcal{O}(g
\cdot p \cdot m)$ specific IDs.

For total surveillance, the attacker must generate $\mathcal{O}(n \cdot m)$
specific IDs.

- TODO calculate how many random IDs in total until the attacker can expect to
  have all the needed IDs

For each of the generated IDs, the attacker needs a separate network address,
assuming the target is paranoid enough to check for this.

To make generating all these IDs more time consuming, the length of the routing
prefix could be extended, thus increasing the number of IDs required. However,
this causes problems in the system, as explained in
Section~\ref{sec:desc_empty_sync_groups}. It also means everyone needs more
query peers for a complete subprefix coverage.

It could also be made harder to generate an ID itself. The ID is assumed to be
the hash of the peer's public key. This hash function could be chosen as one
with adjustable work factor. Paranoid peers could elect not to query peers whose
ID has a low work factor, or even leave query groups if such a peer joins to
avoid passive observation.

The attacks described here are likely too difficult to be practical for low-tech
attackers, given paranoid and vigilant targets, especially if they are not aimed
at just a few specific peers. But they are probably in reach of high-tech
attackers, especially if the attack is targeted on a specific peer, even though
they are likely not trivial.
