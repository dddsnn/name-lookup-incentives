\chapter{Performance}
\label{chap:performance}
- TODO call this "performance potential"? we're not measuring real performance,
  just theoretically assessing the potential performance, basically just looking
  at how many hops lookups take
- TODO just call this "comparison with alternatives"?
- TODO this is basically just the performance of the dht, has nothing to do with
  the penalty system (until we get to timeouts)

One of the requirements of the system is to provide name resolution with a
latency low enough to be useful for real-time interaction, e.g. in the chat
system given as an example in the introduction.

The central performance characteristic to consider is the total latency of a
query a peer sends. It is dependent on a number of factors:

\begin{itemize}
\item The distance of the peer's routing prefix to the target ID (in the extreme
case, the peer is responsible for the target ID himself and can resolve the
query locally).
\item The network delay between the peers involved in the query chain.
\item The length of the routing prefix. The longer it is, the higher the average
number of recursive steps necessary for a resolution.
\item The subprefix coverage of all involved peers. The more peers there are per
subprefix, the higher the chance that a step in the query chance can be skipped
(by being able to select a peer who is not just one bit closer to the target ID,
but two, or even more). If the subprefix coverage is incomplete at some steps,
the query may not be resolvable at all without violating the rule that a query
recipient must be closer to the target ID.
\item The reputation of the involved peers in their query groups. Firstly, this
determines whether a penalty delay will be applied, but if a recipient is
reputation saturated, he will let the query time out.
\item Whether only one or multiple queries are sent. When sending more than one,
the probability that both fail or time out is lower than if only one is sent.
\end{itemize}

- TODO measure performance of simulation?

\section{Theoretical Latency}
The system has not been fully implemented, so it is not possible to measure
latencies from a real system. But theoretical considerations can be done. A good
starting point is to be optimisitic about the factors influencing latency.

\paragraph{Simple Case}
Assume the following:

\begin{itemize}
\item The network round-trip time between any two peers is
$t_{RTT}$\footnote{Of course it's very unrealistic to assume it's the same for
any pair of peers. But it's useful as a first approximation to make statements
about expected query times.}.
\item The routing prefix is $l_{r}$ bits long.
\item Every peer in the network has complete subprefix coverage and knows
exactly $n_{qp}$ query peers for every subprefix.
\item All peers are above the penalty threshold reputation in all query groups,
but none are saturated.
\item Peers only send one query at time.
\item Local lookups are instantaneous.
\item There are no packet drops of failures in the network.
\item Connections between query peers are already set up, no need for
handshakes.
\item Peer IDs are equally distributed.
\end{itemize}

If a query is sent under these circumstances, all peers will be cooperative and
no penalty delays will be imposed.

Now suppose a query is sent for a target ID $ID_{target}$ that is $k$ steps away
from the querying peer's routing prefix $r$, i.e. $l_{r} - o(ID_{target}, r) =
k$. At every but the last step in the query chain, a peer has to be selected to
forward the query to. Suppose the selected peer's routing prefix is exactly one
bit closer to the target ID. Then the query will take $k \cdot t_{RTT}$.

\paragraph{Skipping Steps}
Now consider that, out of the $n_{qp}$ query peers each peer in the query chain
has to choose from to forward the query to, some may be more than one step
closer to the target ID. Thus it may be possible to, for example, forward the
query to a peer two steps closer, skipping one step in the query chain. The
probability to be able to skip at least $n_{skip}$ steps is $1 - (1 -
2^{-n_{skip}})^{n_{qp}}$ (assuming, of course, that there even are that many
steps left).

To get the expected value $E(n_{qp})$ of steps that can be skipped given a
number of query peers, set this probability to 0.5. Solving for $n_{skip}$
yields

\[E(n_{qp}) = n_{skip} = \log_2\left(\frac{1}{1 -
0.5^{\frac{1}{n_{qp}}}}\right).\]

So for example, with 4 query peers, the expected number of steps that can be
skipped by selecting the right peer is about 2.65. To be precise of course, this
has to be capped by the distance to the target ID.

Taking this into account, the expected query time drops to $\lceil\frac{k}{1 +
E(n_{qp})}\rceil \cdot t_{RTT}$.

\paragraph{Timeouts}
- TODO necessary?
Responses can time out if the recipient of a query is reputation saturated, or
is unable to resolve the query in the required time. Let $p_{timeout}$ be the
probability that a query times out, and let $t_{timeout}$ be the time after
which peers consider a query timed out. Further assume that timeouts can happen
independently at any point in a query chain.

Since $t_{timeout}$ is the same for all peers, one timeout means the query times
out at every step and has to be retried from scratch.

- then consider probability peers are saturated and don't answer, (adjust
  attenuation)
- quantify latency increase due to response failures
    - consider multiple concurrent queries

\paragraph{Concrete Values}
Plugging in some roughly realistic values for the variables to get a feel for
the kind of latency the \ac{DHT} has.

Set $t_{RTT} = \SI{50}{\milli\second}$, and $n_{qp} = 3$. Set the length of the
routing prefix $l_r = 16$. Supposing a network of $2^{22}$, or about 4 million,
peers, this implies average sync group sizes of 64 peers.

In the simple case where no steps are skipped, a query that has to go the full
$l_r$ steps (i.e. where the querying peer has no overlap with the target ID)
takes $16 \cdot \SI{50}{\milli\second} = \SI{800}{\milli\second}$.

If we do consider skipping steps, the average query time drops to
$\lceil\frac{16}{1 + E(3)}\rceil \cdot \SI{50}{\milli\second} =
\SI{250}{\milli\second}$.

\section{Message Sizes (TODO maybe call this overhead? include the query group
management overhead etc)}
- TODO necessary?
- define messages, estimate sizes, id lengths etc.
    - remember signatures for size
    - remember public key for size of a record (or use a separate record to map
      key fingerprint to full key?)
    - remember last changed time (to determine if a received record is very old
      and to give a penalty in that case)
    - remember (possibly?) ttl
    - reputation update regarding responses have to state which ID the query was
      for if peer selection overlap-rep-sorted is used, so that the lowest-rep
      peer can detect that he should have been sent a query

- size of messages also a concern
- sync group maintenance overhead
- query group, rep management
- quantify the overhead from responding to queries necessary to maintain rep
    - assume all peers query at the same rate, so basically need to answer one
      query for every query sent

- TODO performance depends on
    - routing prefix length, indirectly on sync group size
    - reputation saturation
    - multiple queries? could add to the load

\section{Comparison with Plain DNS}
The obvious first candidate to compare against is plain DNS. Firstly, the system
described in this thesis is designed to address the privacy issues typical
usage of DNS entails. DNS also uses a hierarchical namespace, as opposed to the
flat one employed by the system's \ac{DHT}.

Measuring DNS performance in real systems is
difficult~\cite{liston2002diversity}, and there would be no real-life data to
compare it against. Instead, as above, look at the number of round-trip times
needed to resolve a query.

- TODO

- sync group size <-> num records per domain
- caching
    - gives performance, but only works with a power-law dist, i've assumed i
      don't have that
- in my system, everyone has a record in the dht

\subsection{Comparison with a Simple Proxy}
A simple extension of just using DNS is using a simple proxy for DNS lookups.
This can provide some privacy protection by making profile building more
difficult. There are two ways for it: either there are a lot of other users
using the proxy for DNS lookups at the same DNS server. Their traffic can then
act as cover traffic. Alternatively, proxies can be switched often to "reset"
the data that has been accumulated.

Performance-wise, this should behave like plain DNS, with the added round-trip
time to the proxy server (provided the proxy is fast enough).

\section{Comparison with Tor}
- TODO tor in background? not central

When talking about privacy-enhacing technology, a comparison with Tor is
basically mandatory. Of course, Tor aims to fully anonymize all TCP traffic, not
just make it less practical to collect data for one particular service. It can
be used just to anonymize DNS, though.

If the user wants to use Tor for anonymization anyway, a separate DNS lookup is
not even necessary. He just specifies the domain name of the target and the exit
node does the lookup.

Plain DNS lookups are actually not possible via Tor, since they are done via
UDP, but Tor only works with TCP. But DNS over TLS or DNS over HTTPS can be
used.

Alternatively, Tor has a remote hostname lookup which instructs the exit node to
resolve a domain name and send back the A or AAAA record. However, this uses a
simple format rather than forwarding the DNS server response. It doesn't include
any DNSSEC records, so the user has to trust the exit node to send the correct
response.

If we stick to a very simplified view for a look at the performance, DNS queries
via Tor essentially add the 3 round-trip times between the 3 Tor nodes to the
time the DNS query takes.

- latency is what's important to us, ignore tor's establishing a circuit and our
  finding query groups, that can be done in the background
- but circuit creation is costly, also with respect to computation (crypto)
- overhead of circuit creation dependent on user's paranoia, how many queries
  he's willing to use a circuit for. in the extreme case, only one query per
  circuit. see
  https://github.com/torproject/torspec/blob/master/proposals/216-ntor-handshake.txt
  for cost
- queries not optimizable for geography, worst case every hop towards the other
  end of the planet
- ignore possible tcp delay from dropped packets, hol blocking, assume network
  not congested?
    - also delay from nagle's, delayed acks
- fixed-size cells of 512 bytes, can probably fit the entire response (but not
  necessarily. don't know what happens in that case, don't think it's possible
  to receive 2 cells as response)
- remote hostname lookup: 512 bytes fixed-length cell, query and response each
- or let the exit node do the lookup and include the first payload. with a small
  http request, lookup is basically free, same as a dns query
- but tor has no incentives (cite tor perf improvements paper)
- use end-to-end latency from performance.torproject.org, gives the rtt of a
  http request/response to a destination server, use this as an approximation
  for the rtt to a dns server, then add the actual dns query time
    - but exitnode to specific destination server should be expected to be at
      least as long as exitnode to any dns server. exitnode uses his own dns
      server, is likely to be close and have low rtt, so the values from
      metrics.tor... are probably a little higher than what can be expected for
      relay\_resolve
- %https://2019.www.torproject.org/docs/faq.html.en#SendPadding tor sends
  padding on idle circuits, up to 103b/s
  (https://github.com/torproject/torspec/blob/master/padding-spec.txt)

- TODO compare with tit-for-tat?
