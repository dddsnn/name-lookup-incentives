\chapter{Performance}
\label{chap:performance}
- define messages, estimate sizes, id lengths etc.
    - remember signatures for size
    - remember public key for size of a record (or use a separate record to map
      key fingerprint to full key?)
    - remember last changed time (to determine if a received record is very old
      and to give a penalty in that case)
    - remember (possibly?) ttl
    - reputation update regarding responses have to state which ID the query was
      for if peer selection overlap-rep-sorted is used, so that the lowest-rep
      peer can detect that he should have been sent a query
- reputation updates could potentially be batched to save overhead
- compare with tit-for-tat
\section{Performance Indices}
TODO
\section{Comparison with Plain DNS}
\section{Comparison with a Simple Proxy}
- tor with a single relay?
\section{Comparison with Tor}
- tor fully anonymizes tcp, stronger than what we do here
    - specify hostname when establishing a connection, exit nodes does dns
      lookup
    - Remote hostname lookup is basically dns over the circuit, (but only A and
      AAAA records)
- latency is what's important to us, ignore tor's establishing a circuit and our
  finding query groups, that can be done in the background
- but circuit creation is costly, also with respect to computation (crypto)
- overhead of circuit creation dependent on user's paranoia, how many queries
  he's willing to use a circuit for. in the extreme case, only one query per
  circuit. see
  https://github.com/torproject/torspec/blob/master/proposals/216-ntor-handshake.txt
  for cost
- queries not optimizable for geography, worst case every hop towards the other
  end of the planet
- ignore possible tcp delay from dropped packets, hol blocking, assume network
  not congested?
- fixed-size cells of 512 bytes, can probably fit the entire response (but not
  necessarily. don't know what happens in that case, don't think it's possible
  to receive 2 cells as response)
- assuming link protocol version 4
- remote hostname lookup: 512 bytes fixed-length cell, query and response each
- or let the exit node do the lookup and include the first payload. with a small
  http request, lookup is basically free, same as a dns query
- but tor has no incentives (cite tor perf improvements paper)
- use end-to-end latency from performance.torproject.org, gives the rtt of a
  http request/response to a destination server, use this as an approximation
  for the rtt to a dns server, then add the actual dns query time
    - but exitnode to specific destination server should be expected to be at
      least as long as exitnode to any dns server. exitnode uses his own dns
      server, is likely to be close and have low rtt, so the values from
      metrics.tor... are probably a little higher than what can be expected for
      relay\_resolve
- TODO what is typical dns query time?
- %https://2019.www.torproject.org/docs/faq.html.en#SendPadding tor sends
  padding on idle circuits, up to 103b/s
  (https://github.com/torproject/torspec/blob/master/padding-spec.txt)
- relay\_resolve doesn't include any dnssec or anything, so we have to trust the
  exitnode. a simple udp proxy is more useful than tor in this case. figure
  that.
    - alternative: do DoT via Tor
TODO
