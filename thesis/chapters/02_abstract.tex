\cleardoublepage

\begin{center}
\paragraph{Abstract}
\hrulefill
\end{center}
Name-lookup services like the domain name system (DNS) are essential to any
large network. Letting queries accumulate at a single point is a privacy issue.
A possible way to address it is to spread queries out to many nodes in a
distributed hash table (DHT). To ensure that there is a large number of these
nodes available, incentives for their operators are required.

This thesis proposes an incentive system that encourages the users of the
service themselves to provide the resources to operate the DHT's nodes. It does
so by punishing those who don't with delayed responses. Users are divided into
small groups in which their level of cooperation is tracked via a reputation
score. These have the benefit that they are scalable and the reputation score is
always known to everyone in the group, so no latency is added looking it up.

The ability of good peers to gain reputation is the main focus of the thesis. A
network of peers implementing a simplified version of the incentive system has
been simulated. Some modifications necessary to the system have been identified.
The results are promising, showing peers who cooperate being able to get to and
maintain the reputation needed to enjoy unpenalized service.

The performance impact of the incentive system is roughly estimated and shows
the DHT being slower that plain DNS, but possible faster than DNS over Tor.

Further work is necessary to complete the system, particularly regarding the
management of the user groups, as well as a mechanism to deal with
disagreements between users. Open questions are identified, potential issues
described, and possible solutions discussed.
