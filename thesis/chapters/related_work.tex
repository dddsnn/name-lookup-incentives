\chapter{Related Work}
\label{chap:related_work}
- TODO

\paragraph{Private Information Retrieval}
The very initial motivation of this thesis was to address the privacy problem a
user has when he sends all his name resolution requests to a single server. This
problem goes away entirely if the server answers the requests satisfactorily but
is unable to tell which name the user queried for. This exactly is the essence
of private information retrieval (PIR)\cite{gasarch2004survey}.

Trivially, PIR is achieved if the server responds with the entire database, no
matter the query. More complex PIR schemes offer less prohibitively expensive
communication complexity, even more so computationally private information
retrieval (cPIR), a relaxation using the assumption attackers can only perform
polynomial computations. An example is the work of Cachin et
al.\cite{cachin1999computationally}. However, even they are impractical for the
purpose of this thesis, since the server must process the entire database for
every query. If he didn't, he could infer that the record the user was looking
for was in the part he didn't process.

\paragraph{DNS Extensions}
DNS over HTTPS (DoH)\cite{RFC8484} principally encrypts DNS traffic, but also
has the potential to address the problem of user data accumulating, at least in
part. In conjunction with HTTP/2's server push\cite{RFC7540}, a web server
serving a web page can proactively include DNS resolutions for the domain names
linked in the page. The user then doesn't need to resolve them himself. But this
only really works for web browsing or similar applications, where the server can
reasonably anticipate which DNS resolutions a client needs. It doesn't transfer
to the chat system example from earlier.

DNS over TLS (DoT)\cite{RFC7858} only encrypts DNS traffic and isn't by itself
useful to us. But it can be used to allow DNS lookups via
Tor\cite{dingledine2004tor} (which doesn't support UDP connections).

\cite{cox2002serving}
Serving DNS Using a Peer-to-Peer Lookup-Service
- TODO

\paragraph{DHTs}
The \ac{DHT} used in this thesis is a simplified variant of
P-Grid\cite{aberer2001pgrid}. One of the central promises of P-Grid is its
self-organizing load balancing, achieved by making peer identifiers independent
of the records a peer stores (the peer's path), as well as a cooperative
distributed algorithm extending or shrinking a peer's responsibility. This
functionality is not used in this thesis, but it is potentially useful in future
work.

The better known Kademlia\cite{maymounkov2002kademlia} is of a similar style, in
that routing takes place along a binary tree.

\paragraph{User Behavior}
McGee and Sk{\aa}geby\cite{mcgee2004gifting} survey the motivation of users of a
real file sharing system to contribute in a way that doesn't promise an
immediate benefit to them, which they call \emph{gifting}. They classify the
motivations, as well as two other dimensions of gifting they found, openness and
proactivity. One kind of openness, restrictive gifters, may choose not to upload
to certain peers, but do so for the good of the system. For example, some may
restrict uploading only to users who contact them personally, to foster social
interaction within the community and thereby increase cooperation. A similar
kind of logic, deliberately treating others badly to enforce better behavior, is
systematized in this thesis.

Ripeanu et al.\cite{ripeanu2006gifting} examine the gifting behavior and
freeriding (not contributing at all, only consuming the service provided by
others) in different BitTorrent communities. They find a very low amount of
freeriders compared to other file sharing applications and attribute this
firstly to user's perception that BitTorrent has effective penalties for
freeriders, and secondly to the fact that most BitTorrent clients are not easily
configured for freeriding. They also find contribution to be very unevenly
distributed (few users uploading a lot), and can't explain this by difference in
bandwidth alone. They see this as evidence that incentives in BitTorrent are not
the only influence on how much a user contributes.

\paragraph{Reputation and Incentives}
- TODO much of the existing work proposes reputation tracking that helps answer
  the question "is this peer going to be good to me", or incentive systems that
  make sure peers are good to others. but i also need to detect or incentivize
  them to be bad to bad peers
- TODO some of these (or all?) are also not applicable to a system where peers
  can't necessarily reciprocate. in bittorrent, everyone in a swarm wants the
  same file, so you can swap pieces. but in a dht, a wants a record from b, but
  b is probably not interested in any of a's records (certainly not right then)

Obreiter and Nimis\cite{obreiter2003taxonomy} firstly present a taxonomy of
uncooperative behavior, which is more closely explained in a previous
paper\cite{obreiter2003stimulating}. It crucially distinguishes profitable from
malicious misbehavior, the former of which grants the one employing it some
advantage, while the other does not. This thesis is principally concerned with
the former. They go on to give a taxonomy of incentive patterns. The one used in
this thesis most closely resembles, in their
terminology, the community pattern.

The NICE system\cite{sherwood2006cooperative} is a distributed trust inference
system for peer-to-peer applications in general. It views them as a set of
two-party interactions, after each of which the two peers exchange a signed
statment, called cookie, about the quality of the interaction, basically giving
a rating. These can later be used by peers to prove to others that they had a
satisfactory interaction before and should therefore trust them now. If no
direct interaction took place, a chain of trust through mutual acquaintances can
also be presented as evidence of trustworthiness.

The purpose of NICE is to ensure that each peer only interacts with other
cooperative peers, whereas this thesis requires the system to also make sure
there are many cooperative peers to choose from. Otherwise, in an extreme case,
one peer could serve all queries and collect everyone's data.

\cite{gupta2003reputation}
A Reputation System for Peer-to-Peer Networks
- TODO (in lit\_todo)
- i think this is the one assuming a central authority i mentioned in the
  initial

\cite{zhou2007gossip}
Gossip-based reputation management for unstructured peer-to-peer networks
- each node holds a reputation vector containing a score for each other node.
  the scores of other nodes are integrated into it according to the node's
  policy.
- nodes get a hold of others' reputation vectors through gossiping
  (probabilistic guarantees)
- bloom filter for compression
- possibly useful, but doesn't offer exactly matching reputation scores. so the
  system must allow for some leeway
- i'm not sure about the degree of collusion resistance

\cite{moreton2003trading}
Trading in trust, tokens, and stamps
- compares trust, token and stamp based incentive techniques
- trust: peers deny service if some trust value is too low
- token: peers require payment in the form of tokens to provide service
- stamp: peers mint their own tokens (stamps) and trade them for service
- paper just argues that stamp is a generalization of both trust and token

\cite{papaioannou2005optimizing}
Optimizing an Incentives’ Mechanism for Truthful Feedback in Virtual Communities
- approach to deal with lying:
- after a transaction, both peers must submit feedback about it (e.g. ok, or
  delayed)
- if they don't match, both peers are punished
- punishment is a little more involved
- apparantyl leads to a stable system

\cite{moreton2003enforcing}
Trust Management 2003 - Enforcing Collaboration in Peer-to-Peer Routing Services
- starts at page 255
- tries to incentivize participation in routing in kademlia
- makes a distinction between trust that a node participates and trust that it
  makes good recommendations (rates other nodes well)
- has a good list of possible attacks
- recommendation trust should be useful, it's similar to my ideas on how to deal
  with nodes being to generous
- trust grows between adjacent nodes because their connection persists for a
  long time (kademlia keeps the longest existing connections around, since
  statistically, they have the highest chance of staying around even longer).
  this is similar to query groups, where trust grows.

\cite{feldman2004robust}
Robust Incentive Techniques for Peer-to-Peer Networks
- TODO
- has a sort of complaint system (section 4.2.2)
- this is the paper that says its sane to be kind to newcomers if newcomers
  recently have been cooperative (section 4.3), strategy is called
  stranger-adaptive

\cite{hassen2012secure}
Secure Reputation Framework for BitTorrent
- TODO (in lit\_todo) i think assumes a central server?

\cite{yonezawa2006novel}
A Novel Protocol for Communicating Reputation in P2P networks
- TODO (in lit\_todo, Trust Management 2006, p.423)
- also uses a central server, but uses statistics somehow, potentially relevant

Shin et al. propose an incentive scheme for file sharing they call
\emph{treat-before-trick (TBeT)}\cite{shin2009treat}, in which files for
download are encrypted with $(t, n)$ threshold secret sharing. There are $n$
subkeys, and any $t$ of them are sufficient to decrypt the file. Peers barter
subkeys, receiving one as payment for a successfully uploaded file chunk from
the peer they uploaded to. This system is fairly unique and interesting, but not
suitable to a latency-sensitive application, because it pushes the mechanism
that prevents freeriding until after the decision is made which file the peer is
interested in. Peers then need to find others to barter for subkeys with, which
is likely to take quite a bit of time, if it is possible at all. Splitting the
small records of the \ac{DHT} into chunks also seems counterproductive.
with only after the decision

\paragraph{Exploitation}
\cite{douceur2002sybil}
The Sybil Attack
- without a validating external authority, it is infeasible to determine whether
  2 distinct identities belong to 2 distinct entities
- 2 identities can prove that they belong to 2 different entities by each
  solving a challenge that only 1 entity couldn't solve by itself. this could be
  testing computational capacity (solving a hard challenge in a time bound) or
  storage capacity (storing a large amount of data)
- but this becomes impossible as the network gets bigger, or at least
  prohibitively impractical
- very relevant, should cite, as it's the most dangerous freerider kind of
  attack

\cite{friedman2001social}
The Social Cost of Cheap Pseudonyms
- TODO
- also considers paying for a new identity
- conclusion says no strategy can do better than punishing all newcomers, but i
  know there was a paper saying it's sane to assume newcomers cooperate at the
  same level as recent newcomers, and treat them accordingly

\cite{feldman2006freeriding}
Free-Riding and Whitewasing in rep systems
- TODO

\cite{li2012collusion}
Collusion Detection in Reputation Systems for Peer-to-Peer Networks
- use either centralized reputation system, or decentralized, but with a number
  of reputation managers
- looks at suspicious behavior (e.g. colluding nodes will rate each other
  frequently) to detect collusion


- TODO the paper with trembles
