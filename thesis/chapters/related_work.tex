\chapter{Related Work}
\label{chap:related_work}
- TODO

The very initial motivation of this thesis was to address the privacy problem a
user has when he sends all his name resolution requests to a single server. This
problem goes away entirely if the server answers the requests satisfactorily but
is unable to tell which name the user queried for. This is the essence of
private information retrieval (PIR), \cite{gasarch2004survey} gives an overview.

Trivially, PIR is achieved if the server responds with the entire database, no
matter the query. More complex PIR schemes offer less prohibitively expensive
communication complexity, even more so computationally private information
retrieval (cPIR) like \cite{cachin1999computationally}, a relaxation using the
assumption attackers can only perform polynomial computations. However, even
they are impractical for the purpose of this thesis, since the server must
process the entire database for every query. If he didn't he could infer that
the record the user was looking for was in the part he didn't process.

DNS over TLS (DoT)\cite{RFC7858} encrypts traffic to DNS servers. As stated
previously, this doesn't solve the problem of privacy-sensitive data
accumulating in a central location.

DNS over HTTPS (DoH)\cite{RFC8484} does this as well, but has the potential to
also address the other problem. In conjunction with HTTP/2's server
push\cite{RFC7540}, a web server serving a web page can proactively include DNS
resolutions for the domain names linked in the page. The user then doesn't need
to resolve them himself. But this only really works for web browsing or similar
applications, where the server can reasonably anticipate which DNS resolutions a
client needs. It doesn't transfer to the chat system example from earlier.

The \ac{DHT} used in this thesis is a simplified variant of
P-Grid\cite{aberer2001pgrid}. One of the central promises of P-Grid is its
self-organizing load balancing, achieved by making peer identifiers independent
of the records a peer stores (the peer's path), as well as a cooperative
distributed algorithm extending or shrinking a peer's responsibility. This
functionality is not used in this thesis, but it is potentially useful in future
work.

The better known Kademlia\cite{maymounkov2002kademlia} is of a similar style, in
that routing takes place along a binary tree.

- various trust management
- various incentives
- freeriding, whitewashing, sybil attack etc.
- peers with utility functions that have a price for privacy
- that paper that found not all peers to be selfish
- that paper that looked at patterns of collusion
- that paper that found trusting peers initially, based on the trustworthiness
  of recent joiners, is sane

The NICE system presented in \cite{sherwood2006cooperative} is a distributed
trust inference system for peer-to-peer applications in general. It views them
as a set of two-party interactions, after each of which the two peers exchange a
signed statment, called cookie, about the quality of the interaction, basically
giving a rating. These can later be used by peers to prove to others that they
had a satisfactory interaction before and should therefore trust them now. If no
direct interaction took place, a chain of trust through mutual acquaintances can
also be presented as evidence of trustworthiness.

The purpose of NICE is to ensure that each peer only interacts with other
cooperative peers, whereas this thesis requires the system to also make sure
there are many cooperative peers to choose from. Otherwise, in an extreme case,
one peer could serve all queries and collect everyone's data.

- treat-before-trick
- gifting technologies paper (already mentioned above)
