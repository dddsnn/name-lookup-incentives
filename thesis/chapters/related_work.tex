\chapter{Related Work}
\label{chap:related_work}
\paragraph{Private Information Retrieval}
The very initial motivation of this thesis was to address the privacy problem a
user has when he sends all his name resolution requests to a single server. This
problem goes away entirely if the server answers the requests satisfactorily but
is unable to tell which name the user queried for. This exactly is the essence
of private information retrieval (PIR)\cite{gasarch2004survey}.

Trivially, PIR is achieved if the server responds with the entire database, no
matter the query. More complex PIR schemes offer less prohibitively expensive
communication complexity, even more so computationally private information
retrieval (cPIR), a relaxation using the assumption attackers can only perform
polynomial computations. An example is the work of Cachin et
al.\cite{cachin1999computationally}. However, even they are impractical for the
purpose of this thesis, since the server must process the entire database for
every query. If he didn't, he could infer that the record the user was looking
for was in the part he didn't process.

\paragraph{DNS Extensions}
DNS over HTTPS (DoH)\cite{RFC8484} principally encrypts DNS traffic, but also
has the potential to address the problem of user data accumulating, at least in
part. In conjunction with HTTP/2's server push\cite{RFC7540}, a web server
serving a web page can proactively include DNS resolutions for the domain names
linked in the page. The user then doesn't need to resolve them himself. But this
only really works for web browsing or similar applications, where the server can
reasonably anticipate which DNS resolutions a client needs. It doesn't transfer
to the chat system example from earlier.

Cox et al.\cite{cox2002serving} explored using a DHT based on Chord to serve
DNS records. They found DNS to be superior to the DHT with regard to latencies.
This is in agreement with the findings in Chapter~\ref{chap:performance}.

\paragraph{DHTs}
The DHT used in this thesis is a simplified variant of
P-Grid\cite{aberer2001pgrid}. One of the central promises of P-Grid is its
self-organizing load balancing, achieved by making peer identifiers independent
of the records a peer stores (the peer's path), as well as a cooperative
distributed algorithm extending or shrinking a peer's responsibility. This
functionality is not used in this thesis, but it is potentially useful in future
work.

The better known Kademlia\cite{maymounkov2002kademlia} is of a similar style, in
that routing takes place along a binary tree.

\paragraph{User Behavior}
McGee and Sk{\aa}geby\cite{mcgee2004gifting} survey the motivation of users of a
real file sharing system to contribute in a way that doesn't promise an
immediate benefit to them, which they call \emph{gifting}. They classify the
motivations, as well as two other dimensions of gifting they found, openness and
proactivity. One kind of openness, restrictive gifters, may choose not to upload
to certain peers, but do so for the good of the system. For example, some may
restrict uploading only to users who contact them personally, to foster social
interaction within the community and thereby increase cooperation. A similar
kind of logic, deliberately treating others badly to enforce better behavior, is
systematized in this thesis.

Ripeanu et al.\cite{ripeanu2006gifting} examine the gifting behavior and
free-riding (not contributing at all, only consuming the service provided by
others) in different BitTorrent communities. They find a very low amount of
free-riders compared to other file sharing applications and attribute this
firstly to user's perception that BitTorrent has effective penalties for
free-riders, and secondly to the fact that most BitTorrent clients are not
easily configured for free-riding. They also find contribution to be very
unevenly distributed (few users uploading a lot), and can't explain this by
difference in bandwidth alone. They see this as evidence that incentives in
BitTorrent are not the only influence on how much a user contributes.

\paragraph{Reputation and Incentives}
Previous work on reputation tracking and incentive techniques exists. These are
usually intended to increase the performance of the system they support, e.g. by
getting users to give the best possible service. Increasing the number of
participants for privacy reasons is not the primary purpose of the incentives.

Most of them are also done with file sharing in mind and not perfectly
applicable to situations in which peers can't directly reciprocate. In a
BitTorrent swarm, file chunks can be swapped, but in the DHT containing
reachability information in this thesis, one peer queries another, but the other
may never want to query the first.

Obreiter and Nimis\cite{obreiter2003taxonomy} firstly present a taxonomy of
uncooperative behavior, which is more closely explained in a previous
paper\cite{obreiter2003stimulating}. It crucially distinguishes profitable from
malicious misbehavior, the former of which grants the one employing it some
advantage, while the other does not. This thesis is principally concerned with
the former. They go on to give a taxonomy of incentive patterns. The one used in
this thesis most closely resembles, in their terminology, the community pattern.

Moreton and Twigg\cite{moreton2003trading} compare incentive techniques, which
they broadly categorize into trust-based and token-based schemes. In the former,
service is denied if a trust value is too low. In the latter, service is bought
with tokens. This is a distinction similar to what others call debit-only and
debit-credit reputation systems. The authors argue that there is a
generalization of both categories, which they call a stamp-based scheme. In it,
peers mint their own tokens which they trade for service. The value of these
tokens is dependent on the trust others have in them.

The NICE system\cite{sherwood2006cooperative} is a distributed trust inference
system for peer-to-peer applications in general. It views them as a set of
two-party interactions, after each of which the two peers exchange a signed
statement, called cookie, about the quality of the interaction, basically giving
a rating. These can later be used by peers to prove to others that they had a
satisfactory interaction before and should therefore trust them now. If no
direct interaction took place, a chain of trust through mutual acquaintances can
also be presented as evidence of trustworthiness.

The purpose of NICE is to ensure that each peer only interacts with other
cooperative peers, whereas this thesis requires the system to also make sure
there are many cooperative peers to choose from. Otherwise, in an extreme case,
one peer could serve all queries and collect everyone's data.

Gupta et al.\cite{gupta2003reputation} created a reputation tracking system for
peer-to-peer networks, with file sharing in mind. They propose two schemes,
credit-only and credit-debit. In the former, users gain reputation to prove
themselves trustworthy and use the underlying service. In the latter, reputation
is deducted for using the service (i.e. downloading). Their system uses a
central reputation computation agent, making it not fully decentralized.

Zhou et al.\cite{zhou2007gossip} propose a reputation management system that
uses gossipping to spread updates. Nodes in the system maintain a reputation
score for each other node. They get hold of other nodes' ratings via gossipping
and integrate them into their own. The authors claim the reputation scores can
be compressed, but it's not clear how updates can be done on the compressed
scores. Maintaining a score for every node and receiving updates for them also
probably doesn't scale to very large networks.

Papaioannou and Stamoulis\cite{papaioannou2005optimizing} propose a mechanism
incentivizing peers to give truthful feedback in situations where it can't be
proven that they're lying. In their system, peers must give feedback about any
transaction with another peer. If feedback of both peers is not in agreement,
both of them are punished. This procedure is effective at discovering liars.  A
mechanism like this is also necessary for the complaint system described in this
thesis in Section~\ref{sec:desc_complaints}.

Moreton and Twigg\cite{moreton2003enforcing} present an incentive system for
Kademlia. In it, nodes rate other nodes based on experience, and share these
ratings with others. A distinction is made between trusting that a node
participates well, and trusting that he makes good recommendations. The result
is a sort of web of trust.

Feldman et al.\cite{feldman2004robust} investigate incentives for cooperation
in peer-to-peer networks with a game theoretic approach. Among other things,
they consider the problem of whitewashing, an exploit performed by peers
unwilling to contribute: Once they have lost all their reputation, they leave
and rejoin with a new identity to start with a clean slate. Interestingly, the
solution is not to always be distrustful of newcomers, but to employ an adaptive
strategy, in which, essentially, newcomers are trusted if newcomers have proven
trustworthy in the recent past.

Shin et al. propose an incentive scheme for file sharing they call
\emph{treat-before-trick (TBeT)}\cite{shin2009treat}, in which files for
download are encrypted with $(t, n)$ threshold secret sharing. There are $n$
subkeys, and any $t$ of them are sufficient to decrypt the file. Peers barter
subkeys, receiving one as payment for a successfully uploaded file chunk from
the peer they uploaded to. This system is fairly unique and interesting, but not
suitable to a latency-sensitive application, because it pushes the mechanism
that prevents free-riding until after the decision is made which file the peer
is interested in. Peers then need to find others to barter for subkeys with,
which is likely to take quite a bit of time, if it is possible at all. Splitting
the small records of the DHT into chunks also seems counterproductive.  with
only after the decision

\paragraph{Exploitation}
Feldman et al.\cite{feldman2006freeriding} investigate free-riding and
whitewashing in peer-to-peer networks with a game theoretic approach. They model
the generosity level of the peers, which determines whether the peer chooses to
contribute or free-ride given the cost of contribution, even without any
incentive. So in their model, peers are not necessarily entirely selfish. They
find that, if the generosity level is low, penalizing free-riders can improve
the system performance. Further, whitewashing can be discouraged by penalizing
newcomers. These two findings are implemented in this thesis.

Friedman and Resnick\cite{friedman2001social} examine the consequences of being
able to easily create new identities more generally, also in a game theoretic
way. They introduce the notion of \emph{trembles}, unintentional mistakes that
appear to others as uncooperative behavior. These can e.g. be caused by failures
in the network, and will be useful for the discussion in this thesis. They also
find that newcomers need to be distrusted.

Li et al.\cite{li2012collusion} examine collusion in real-world reputation
systems. Collusion in a reputation tracking system is an exploit performed by
two or more peers aimed at giving each other good ratings in order to quickly
gain and maintain high reputation.  The authors use the observations to propose
a method that detects colluding peers by their typical behavior.

Douceur\cite{douceur2002sybil} describes the Sybil attack: Without an external
authority, it is impractical to determine whether two distinct identities
represent two distinct entities, or are in fact the same person. This kind of
attack is relevant to reputation systems since it allows one peer using multiple
identities to give himself rewards. It is essentially one peer colluding with
himself.
