\chapter{Implementation}
\label{chap:implementation}
A simulation has been implemented using the Simpy discrete event simulation
framework (TODO put on github and link). It simulates a network of selfish peers
providing and accessing a \ac{DHT} implementing the system described in the
previous chapter.

Parameters of the simulation are set via a \texttt{.settings} file, which is
supposed to make results of simulation runs reproducible\footnote{The simulation
uses floating-point arithmetic, which is a possible source of indeterminism
between different systems\cite{dawson2013determinism}. On the same system,
results should be reproducible.}. Parameters for the default settings, along
with some explanation, can be found in Section~\ref{sec:app_default_settings}.
They are chosen in such a way that the dimensionless time units of the
framework can be viewed as seconds.

\section{Initialization}
Before the simulation starts, the environment is initialized. This involves
generating IDs for all the peers, creating the peer objects, and introducing
each peer to a few others.

The IDs are generated randomly, although there are some options that can
influence this by forcing all sync groups to be non-empty, or even have the same
sizes (sync groups are determined by the routing prefix, which is a part of the
ID).

Once all peers are created, they are introduced to other peers, i.e. given their
ID and network address such that they can communicate with them. Each peer is
introduced to all of his sync peers as well as a number of randomly selected
peers with which he may from query groups (the random sample may also contain
sync peers and himself). Afterwards, every peer's process to complete the
subprefix coverage is queued up to run once the simulation starts.

Finally, another process is started that facilitates traffic. Every second, one
request for a full ID is generated at every peer. A request represents a user
interacting with the protocol stack. It may lead to the peer needing to send out
a query in order to resolve it, but it may not (if the peer already knows the
answer, e.g. if the requested ID is in the same sync group). The requested ID is
randomly chosen with equal distribution from all existing peer IDs and may be
the peer's own ID.

\section{Peer Behavior Model}
\subsection{Completing Subprefix Coverage}
Peers are introduced to a number of other peers once during initialization, but
every time they receive a successful response they are introduced to the peer
whose information is contained in the response.

When a peer is introduced to another one he doesn't already know, he checks
which of his subprefixes the other peer serves. If he already knows enough peers
serving that subprefix, he doesn't do anything. Otherwise, he will join a query
group with the new peer.

This process of joining a query group is simplified: All peers have access to a
list of all query groups. The peer will first check if there is a query group
the other peer is in that is not full. If so, he will add himself to that group.
Otherwise, he will check if any of his own query groups is not full and, if so,
add the other peer to it. Otherwise, he will create a new query group containing
just himself and the other peer. A query group is considered full once its size
reaches a number adjustable in the settings, which is the same for all peers.
There is no further mechanism to prevent query groups from becoming too big.

Each peer repeatedly, after a fixed interval, runs a process that attempts to
complete the peer's subprefix coverage. The peer for which of his subprefixes he
knows fewer peers than he'd like and sends queries for these subprefixes. Once a
successful response arrives, the peer information contained in it is introduced
and a query group can be joined. In those queries, the peer specifies which
peers covering the subprefix he already knows, so they will not occur in the
response. The queries are also special in that peers can select recipients that
are further away from the target ID. This is usually disallowed to prevent
routing loops. However, when completing subprefix coverage, they may be
necessary: if the peer doesn't know anyone closer to the desired subprefix (and
that is the problem he is trying to solve), he must necessarily query someone
who is not closer. There is no mechanism to prevent routing loops implemented.

\section{Cooperative/Defecting Behavior}
\label{sec:impl_coop_defect_behavior}
Peers are modeled to be selfish with regards to their willingness to cooperate.
They behave cooperatively and do their best to respond to incoming queries
within the correct time window while they are not yet reputation saturated.
They are saturated once their reputation in the relevant query group (the one
shared with the peer sending a query) is at least the penalty threshold
reputation plus the \emph{reputation buffer}. If they share more than one query
group with the querying peer, they consider their minimum reputation, since they
will receive rewards and penalties in all those groups.

If they receive a query and are reputation saturated, they will ignore it
(eventually causing a timeout at the sender), but expect to receive a penalty
for doing this. That means they store an expectation in all the relevant query
groups that their reputation there will be reduced by a penalty in the future.
Once the penalty is actually applied, they delete the record of the expectation.
When determining whether they are reputation saturated, they take into account
the expectations they have stored. In effect, once they ignore a query, they
expect a penalty and thus are immediately not saturated anymore.

The reputation buffer is a configurable parameter determining how long peers
stay cooperative before they defect for the first time. It is a simplification
that acts as a proxy both for the peers' valuation of reputation, as well as the
cost of cooperating. Essentially, peers give reputation infinitely high value
while they have less than the saturation reputation, and no value otherwise.
This makes it impossible for them to factor the cost of the action requested of
them into their decision.

When peers are trying to resolve a target ID, they only send one query at a
time. Only once that query has failed do they send a query to the next viable
peer. For each target ID they are trying to resolve, they note which peers they
have already queried and don't try them again. If all viable peers have been
tried, they consider the query to have failed.

No complaint system has been implemented, so simplified reputation updates are
used: Peers simply broadcast reputation updates faithfully and honestly, not
considering any incentive for doing so. When a response arrives or a timeout
occurs, they send an update to each member in all the query groups they share
with the responder. Contrary to the system description, even for successful
responses it is the peer who sent the query that broadcasts the reputation
update, not the one who responded successfully.

\section{Simplifications}
The implementation contains a number of simplifications compared to the system
described in Chapter~\ref{chap:system_description}.

\begin{itemize}
\item As already mentioned, there is no complaint system. Any actions that may
require it, like reputation updates, are always executed faithfully.

\item Query group creation and joining is simplified in that peers can simply
add themselves to any query group they like, and even add other peers to an
existing query group without that peer having any say in this. For this purpose,
peers have access to a complete list of all existing query groups that is shared
among all peers. Creating a new query group is done by placing a new object in
that list.

\item There is no way for peers to join query groups other than through the
mechanism to complete their subprefix coverage. There is also no way at all to
leave a query group. This leads to some very small groups, usually a few with
just two peers. In these groups, reputation gains are very slow because of the
low amount of traffic.

\item Peers never go offline, change their addresses or update the record they
store in the \ac{DHT}.

\item Peers respond to all queries, even from peers they don't share a query
group with. For those, they apply the maximum penalty delay.

\item Peers never respond too early, and hence there is no mechanism implemented
to penalize this.

\item Peers don't give any leeway when checking another peer's reputation. When
deciding the penalty delay, the current reputation is used, even if e.g. the
sender of the query had more reputation when he sent it and is thus expecting a
faster response.

\item No encryption or signing of messages is actually taking place.
\end{itemize}

\section{Analysis Module}
When the simulation is running, a log is kept of all interesting events like
queries or responses being sent or timing out. Once the simulation is done, this
log is written to a file that allows to examine the behavior of the system.

The \texttt{analyze} module is capable of creating various plots concerning the
system behavior.
\begin{description}
\item[Reputation percentiles] Illustrates the development of reputation of the
members of one query group over time. The reputation of the 5th, 25th, 50th,
75th and 95th percentile of reputation is plotted with one graph each. Note that
in many cases, query groups have 16 members, and so the 5th and the 95th
percentile represent only one peer. The penalty threshold reputation as well as
the saturation reputation are indicated by horizontal lines.
\item[Peer reputations] Illustrates the reputation values of one peer in all of
the query groups he is in over time. The reputation for each group is shown as a
separate graph. Optionally, a bar graph in the background shows the fraction of
the peer's responses that suffers from the recursive query problem (i.e. the
peer tried to respond, but once he was able, it was too late already). This plot
also includes horizontal lines indicating the penalty threshold reputation and
the saturation reputation.
\item[Response statuses] Illustrates the status of responses within the
simulation over time. The statuses \emph{success}, \emph{failure} and
\emph{timeout} should be self-explanatory. A response can be \emph{unmatched} if
peer A sends a query to peer B, the query then times out at A, but B responds
(late) anyway. By that time, A has deleted the record of having sent a query to
B and thus can't match the response to an outgoing query. This frequently occurs
with the recursive query problem. The status \emph{returned\_excluded\_peer}
denotes the situation where a query was sent where certain IDs were excluded,
i.e.  unwanted as a response, but returned in the response anyway. This doesn't
happen in the simulation.
\end{description}
